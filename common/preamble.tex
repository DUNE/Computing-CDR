% stuff before document begins.  

% Most packages are required through dune.cls.  Put hyper-ref related here to assure it's last.
%%% xr won't play nice
% \usepackage{xr-hyper}
\usepackage[pdftex,bookmarks,hidelinks]{hyperref}

\graphicspath{ {graphics/} }

% make some "if"s for DP and SP.
% They should be set true inside DP or SP volume or chapter mains.
% set try like \dptrue or \sptrue.
% they can be referenced with \ifdp or \ifsp and terminated with \fi.
\newif\ifdp
\newif\ifsp




% This holds definitions of macros to enforce consistency in names.

% This file is the sole location for such definitions.  Check here to
% learn what there is and add new ones only here.  

% also see units.tex for units.  Units can be used here.

%%% Common terms

% Check here first, don't reinvent existing ones, add any novel ones.
% Use \xspace.

%%%%% Anne adding macros for referencing TDR volumes and chapters May 2019 %%%%%
\def\expshort{DUNE\xspace}
\def\dune{\expshort}
\def\explong{The Deep Underground Neutrino Experiment\xspace}

\def\thedocsubtitle{Deep Underground Neutrino Experiment (DUNE)} 
\def\tdrtitle{Technical Design Report}
% All volume titles and numbers in one place.
\def\voltitleexec{Introduction to DUNE\xspace}
\def\volnumberexec{I}
% Note structure of definition name: 
% vol=intro, ch for chapter, short id for chap, e.g., es=exec summary -->
% e.g.,  intro ch es --> introches --> \introches
\def\introches{Volume~\volnumberexec{}, \voltitleexec{}, Chapter~1\xspace}
\def\introchphys{Volume~\volnumberexec{}, \voltitleexec{}, Chapter~2\xspace}
\def\introchsp{Volume~\volnumberexec{}, \voltitleexec{}, Chapter~3\xspace}
\def\introchdp{Volume~\volnumberexec{}, \voltitleexec{}, Chapter~4\xspace}
\def\introchnd{Volume~\volnumberexec{}, \voltitleexec{}, Chapter~5\xspace}
\def\introchcomp{Volume~\volnumberexec{}, \voltitleexec{}, Chapter~6\xspace}
\def\introchtc{Volume~\volnumberexec{}, \voltitleexec{}, Chapter~7\xspace}

\def\voltitlephysics{DUNE Physics\xspace}
\def\volnumberphysics{II}
\def\physches{Volume~\volnumberphysics{}, \voltitlephysics{}, Chapter~1\xspace}
\def\physchproj{Volume~\volnumberphysics{}, \voltitlephysics{}, Chapter~2\xspace}
\def\physchland{Volume~\volnumberphysics{}, \voltitlephysics{}, Chapter~3\xspace}
\def\physchtools{Volume~\volnumberphysics{}, \voltitlephysics{}, Chapter~4\xspace}
\def\physchlbl{Volume~\volnumberphysics{}, \voltitlephysics{}, Chapter~5\xspace}
\def\physchndk{Volume~\volnumberphysics{}, \voltitlephysics{}, Chapter~6\xspace}
\def\physchsnb{Volume~\volnumberphysics{}, \voltitlephysics{}, Chapter~7\xspace}
\def\physchbsm{Volume~\volnumberphysics{}, \voltitlephysics{}, Chapter~8\xspace}
\def\physchconcl{Volume~\volnumberphysics{}, \voltitlephysics{}, Chapter~9\xspace}

\def\voltitletc{DUNE Far Detector Technical Coordination\xspace}
\def\volnumbertc{III}
\def\tcches{Volume~\volnumbertc{}, \voltitletc{}, Chapter~1\xspace}
\def\tcchproj{Volume~\volnumbertc{}, \voltitletc{}, Chapter~2\xspace}
\def\tcchdesorg{Volume~\volnumbertc{}, \voltitletc{}, Chapter~3\xspace}
\def\tcchjpo{Volume~\volnumbertc{}, \voltitletc{}, Chapter~4\xspace}
\def\tcchfac{Volume~\volnumbertc{}, \voltitletc{}, Chapter~5\xspace}
\def\tcchdet{Volume~\volnumbertc{}, \voltitletc{}, Chapter~6\xspace}
\def\tcchie{Volume~\volnumbertc{}, \voltitletc{}, Chapter~7\xspace}
\def\tcchrev{Volume~\volnumbertc{}, \voltitletc{}, Chapter~8\xspace}
\def\tcchqa{Volume~\volnumbertc{}, \voltitletc{}, Chapter~9\xspace}
\def\tcchesh{Volume~\volnumbertc{}, \voltitletc{}, Chapter~10\xspace}
\def\tcchappx{Volume~\volnumbertc{}, \voltitletc{}, Chapter~11\xspace}

\def\voltitlesp{The DUNE Far Detector Single-Phase Technology\xspace}
\def\volnumbersp{IV}
\def\spches{Volume~\volnumbersp{}, \voltitlesp{}, Chapter~1\xspace}
\def\spchapa{Volume~\volnumbersp{}, \voltitlesp{}, Chapter~2\xspace}
\def\spchhv{Volume~\volnumbersp{}, \voltitlesp{}, Chapter~3\xspace}
\def\spchtpcelec{Volume~\volnumbersp{}, \voltitlesp{}, Chapter~4\xspace}
\def\spchpds{Volume~\volnumbersp{}, \voltitlesp{}, Chapter~5\xspace}
\def\spchcalib{Volume~\volnumbersp{}, \voltitlesp{}, Chapter~6\xspace}
\def\spchdaq{Volume~\volnumbersp{}, \voltitlesp{}, Chapter~7\xspace}
\def\spchcisc{Volume~\volnumbersp{}, \voltitlesp{}, Chapter~8\xspace}
\def\spchinstall{Volume~\volnumbersp{}, \voltitlesp{}, Chapter~9\xspace}

\def\voltitledp{The DUNE Far Detector Dual-Phase Technology\xspace}
\def\volnumberdp{V}
\def\dpches{Volume~\volnumberdp{}, \voltitledp{}, Chapter~1\xspace}
\def\dpchcrp{Volume~\volnumberdp{}, \voltitledp{}, Chapter~2\xspace}
\def\dpchhv{Volume~\volnumberdp{}, \voltitledp{}, Chapter~3\xspace}
\def\dpchtpcelec{Volume~\volnumberdp{}, \voltitledp{}, Chapter~4\xspace}
\def\dpchpds{Volume~\volnumberdp{}, \voltitledp{}, Chapter~5\xspace}
\def\dpchcalib{Volume~\volnumberdp{}, \voltitledp{}, Chapter~6\xspace}
\def\dpchdaq{Volume~\volnumberdp{}, \voltitledp{}, Chapter~7\xspace}
\def\dpchcisc{Volume~\volnumberdp{}, \voltitledp{}, Chapter~6\xspace}
\def\dpchinstall{Volume~\volnumberdp{}, \voltitledp{}, Chapter~9\xspace}


% This one used for testing only - SWC volume is not included in TDR
\def\voltitleswc{DUNE SC\xspace}
\def\volnumberswc{22}
\def\voltitlend{DUNE SC\xspace}
\def\volnumbernd{21}

% see~\refsec{exec}{2.3}
\newcommand{\refsec}[2]{Volume~\csname volnumber#1\endcsname \xspace Section~#2}
% see~\refch{exec}{2}
\newcommand{\refch}[2]{Volume~\csname volnumber#1\endcsname \xspace Chapter~#2}
% see Table~\refinch{exec}{1.2}
\newcommand{\refinch}[2]{#2 in Volume~\csname volnumber#1\endcsname \xspace}

\newcommand{\bigo}[1]{\ensuremath{\mathcal{O}(#1)}}


% Things about oscillation
%
\newcommand{\numu}{\ensuremath{\nu_\mu}\xspace}
\newcommand{\nue}{\ensuremath{\nu_e}\xspace}
\newcommand{\nutau}{\ensuremath{\nu_\tau}\xspace}

\newcommand{\anumu}{\ensuremath{\bar\nu_\mu}\xspace}
\newcommand{\anue}{\ensuremath{\bar\nu_e}\xspace}
\newcommand{\anutau}{\ensuremath{\bar\nu_\tau}\xspace}

\newcommand{\dm}[1]{\ensuremath{\Delta m^2_{#1}}\xspace} % example: \dm{12}

\newcommand{\sinst}[1]{\ensuremath{\sin^2\theta_{#1}}\xspace} % example \sinst{12}
\newcommand{\sinstt}[1]{\ensuremath{\sin^22\theta_{#1}}\xspace}  % example \sinstt{12}

\newcommand{\deltacp}{\ensuremath{\delta_{\rm CP}}\xspace}   % example \deltacp
\newcommand{\mdeltacp}{\ensuremath{\delta_{\rm CP}}}   %%%%%%%%%%  <--- missing something; what's the m for?

\newcommand{\nuxtonux}[2]{\ensuremath{\nu_{#1} \to \nu_{#2}}\xspace}  % example \nuxtonux23 (no {...} )
\newcommand{\numutonumu}{\nuxtonux{\mu}{\mu}}
\newcommand{\numutonue}{\nuxtonux{\mu}{e}}
% Add chi sqd MH?  avg delta chi sqd?

\newcommand{\numubartonumubar}{
\ensuremath{\overline{\numu}\rightarrow\overline{\numu}}\xspace
}

\newcommand{\numubartonuebar}{
\ensuremath{\overline{\numu}\rightarrow\overline{\nue}}\xspace
}
% atmospheric neutrinos and PDK
\newcommand{\ptoknubar}{\ensuremath{p\rightarrow K^+ \overline{\nu}}\xspace}
\newcommand{\ptoepizero}{\ensuremath{p \rightarrow e^+ \pi^0}\xspace}
\newcommand{\ntoek}{\ensuremath{n\rightarrow e^{-}K^{+}}\xspace}
\newcommand{\nnbar}{\ensuremath{n-\bar{n}}\xspace}

% FD parameters (as newcommand for glossary)
\newcommand{\nominalmodsize}{\SI{10}{kt}\xspace} % nominal module size 10 kt
\newcommand{\fdfiducialmass}{\SI{40}{\kt}\xspace} 

% Isotopes - stay here
\def\argon40{${}^{40}$Ar}       
\def\Ar39{$^{39}$Ar}
\def\Cl40{$^{40}$Cl}
\def\K40{$^{40}$K}
\def\B8{$^{8}$B}
\newcommand\isotope[2]{\textsuperscript{#2}#1} % use as, e.g.,: \isotope{Si}{28}

% Parameters common to SP DP
\def\ndfromtarget{\SI{574}{\meter}\xspace} % ND from target
%\def\fdfiducialmass{\SI{40}{\kt}\xspace}
\def\driftvelocity{\SI{1.6}{\milli\meter/\micro\second}\xspace} % same for sp and dp?
\def\lartemp{\SI{88}\,K\xspace}
\def\larmass{\SI{17.5}{\kt}\xspace} % full mass in cryostat

% from https://edms.cern.ch/ui/file/1834156/1/CENBNFCR0075.pdf 12/11/19:
\def\cryostatht{\SI{17.8}{\meter}\xspace} % outer height of cryostat (Jim Stewart 5/2/19)
\def\cryostatlen{\SI{65.8}{\meter}\xspace} % outer length of cryostat (Jim Stewart 5/2/19)
\def\cryostatwdth{\SI{18.9}{\meter}\xspace} % outer width of cryostat (Jim Stewart 5/2/19)

\def\cryostathtinner{\SI{14.0}{\meter}\xspace} % inner height of cryostat (Jim Stewart 5/2/19)
\def\cryostatleninner{\SI{62.0}{\meter}\xspace} % inner length of cryostat (Jim Stewart 5/2/19)
\def\cryostatwdthinner{\SI{15.1}{\meter}\xspace} % inner width of cryostat (Jim Stewart 5/2/19)

%\def\nominalmodsize{\SI{10}{kt}\xspace} % nominal module size 10 kt
\def\dunelifetime{\SI{20}{years}\xspace} % nominal operational life time of DUNE experiment
\def\pipiibeampower{\SI{1.2}{MW}\xspace} 
\def\cooldown{cool-down\xspace} % standardize w/ or w/o space or hyphen

% Parameters SP
\def\spmaxfield{\SI{500}{\volt/\centi\meter}\xspace} % SPfield strength
\def\spactivelarmass{\SI{10}{\kt}\xspace} % active mass in cryostat
\def\spmaxdrift{\SI{3.5}{\m}\xspace}
\def\tpcheight{\SI{12.0}{\meter}\xspace} % height of SP TPC, APA, CPA and of DP TPC
\def\sptpclen{\SI{58.2}{\meter}\xspace} % length of SP TPC, APA, CPA
\def\apacpapitch{\SI{2.3}{\meter}\xspace} % pitch of SP CPAs and APAs
\def\spfcmodlen{\SI{3.5}{\m}} % length of SP FC module
\def\spnumch{\num{384000}\xspace} % total number of APA readout channels 
\def\spnumpdch{\num{6000}\xspace} % total number of PD readout channels 
\def\planespace{\SI{4.8}{\milli\meter}\xspace}
\def\sptargetdriftvolt{$-\SI{180}{\kilo\volt}$\xspace} % target drift voltage - positive
\def\sptargetdriftvoltpos{\SI{180}{\kilo\volt}\xspace} % target drift voltage - positive
\def\coldbox{cold box\xspace} % standardize w/ or w/o space or hyphen
\def\Coldbox{Cold box\xspace} % standardize w/ or w/o space or hyphen
\def\endwall{end wall\xspace} % standardize w/ or w/o space

% Parameters DP
\def\dpactivelarmass{\SI{12.1}{\kt}\xspace} % active mass in cryostat
\def\dpfidlarmass{\SI{10.6}{\kt}\xspace} % fiducial mass in cryostat
\def\dpmaxdrift{\SI{12.0}{\m}\xspace} % max drift length
\def\dptpclen{\SI{60.0}{\meter}\xspace} % length of TPC
\def\dptpcwdth{\SI{12.0}{\meter}\xspace} % width of TPC
\def\dpswchpercrp{\num{36}\xspace} % number of anode/lem sandwiches per CRP 
\def\dpnumswch{\num{2880}\xspace} % total number of anode sandwiches in module
\def\dptotcrp{\num{80}\xspace} % total number of CRPs in module
\def\dpchpercrp{\num{1920}\xspace} %  channels per CRP
\def\dpnumcrpch{\num{153600}\xspace} % total number of CRP channels in module
\def\dpchperchimney{\num{640}\xspace} %  channels per chimney  --CRP channels?
\def\dpnumpmtch{\num{720}\xspace} % number of PMT channels
\def\dpstrippitch{\SI{3.1}{\milli\meter}\xspace} % pitch of anode strips
\def\dpnumfcmod{\num{244}\xspace} % number of FC modules
\def\dpnumfcres{\num{240}\xspace} % number of FC resistors
\def\dpnumfcrings{\num{60}\xspace} % number of FC rings
\def\dpnominaldriftfield{\SI{500}{\volt/\cm}\xspace} % nominal drift voltage per cm
\def\dptargetdriftvoltpos{\SI{600}{\kV}\xspace} % target drift voltage - positive
\def\dptargetdriftvoltneg{\SI{-600}{\kV}\xspace} % target drift voltage - negative

% Nominal readout window time
%% SP has 2.25ms drift time.  The readout is 2*dt + 20%*dt extra.
\def\spreadout{\SI{5.4}{\ms}\xspace}
%% DP has 7.5 ms drift time.  The same (over generous) rule gives 16.5ms
\def\dpreadout{\SI{7.5}{\ms}\xspace}
% Supernova Neutrino Burst buffer and readout window time
\def\snbtime{\SI{100}{\s}\xspace}
% interesting amount of time we might have SNB neutrinos but not yet
% enough to trigger.
\def\snbpretime{\SI{10}{\s}\xspace}
% SP SNB dump size. MUST KEEP THIS MANUALLY IN SYNC 1.5 TB/s * \snbtime
\def\spsnbsize{\SI{45}{\TB}\xspace}

% available power in the CUC and for DAQ
\def\cucpower{\SI[inter-unit-product =$\cdot$]{500}{\kilo\volt\ampere}\xspace}
\def\daqpower{\SI[inter-unit-product =$\cdot$]{500}{\kilo\volt\ampere}\xspace}
\def\surfdaqpower{\SI[inter-unit-product =$\cdot$]{50}{\kilo\volt\ampere}\xspace}

% available racks in the CUC and for DAQ.
\def\cucracks{\SI{60}{racks}\xspace}
\def\daqracks{\SI{56}{racks}\xspace}
\def\surfdaqracks{\SI{8}{racks}\xspace}

% keep these three numerically in sync
\def\offsitepbpy{\SI{30}{\PB/\year}\xspace}
\def\offsitegbyteps{\SI{1}{\GB/\s}\xspace}
\def\offsitegbps{\SI{8}{\Gbps}\xspace}
\def\surffnalbw{\SI{100}{\Gbps}\xspace}



% New from Anne March/April 2018
%physics terms
\newcommand{\efield}{E field\xspace}
\newcommand{\Lbl}{Long-baseline\xspace}
\newcommand{\rms}{RMS\xspace} % Might want this small caps?
\newcommand{\threed}{3D\xspace}
\newcommand{\twod}{2D\xspace}
\newcommand{\fdth}{feedthrough\xspace} % ok not in gloss
\newcommand{\phel}{photoelectron\xspace} % ok not in gloss
\newcommand{\frfour}{FR-4\xspace} % used in gloss and sp hv chap



% Top-level requirements and specifications
% 1 Minimum drift field
\def\mindriftfield{\SI{250}{\volt/\cm}\xspace}
\def\mindriftfieldgoal{\SI{500}{\volt/\cm}\xspace}
% 2 FE elec noise
\def\elecnoisefe{ \SI{1000}{e$^-$}\xspace}
% 3 light yield
\def\lightyield{\SI{0.5}{pe/\MeV}\xspace}
\def\lightyieldgoal{\SI{5}{pe/\MeV}\xspace}
% 4 time resolution
\def\timeres{\SI{1}{\micro/\second}\xspace}
\def\timeresgoal{\SI{100}{ns}\xspace}
% 5 LAr purity
\def\larpurity{\SI{100}{ppt}\xspace}
\def\larpuritygoal{\SI{30}{ppt}\xspace}
% 6 APA gaps
\def\apagapsame{\SI{15}{\milli\m}\xspace}
\def\apagapdiff{\SI{30}{\milli\m}\xspace}
% 7 drift field uniformity (from component positioning)
\def\fielduniformity{\SI{1}{\%}\xspace}
% 8a APA collection wire angle
\def\apacollwireangle{$\SI{0}{^\circ}$\xspace}
% 8b APA induction wire angle
\def\apainducwireangle{$\pm\SI{35.7}{^\circ}$\xspace}
% 9a APA wire pitch - U,V
\def\uvpitch{\SI{4.7}{\milli\meter}\xspace}
% 9b APA wire pitch - X, G
\def\xgpitch{\SI{4.8}{\milli\meter}\xspace}
% 10 APA wire position tolerance
\def\wirepitchtol{$\pm$\SI{0.5}{\milli\meter}\xspace}
% 11 drift field uniformity (from HVS)
\def\fielduniformityhv{\SI{1}{\%}\xspace}
% 12 HV PS ripple contrib to noise
\def\hvripplenoise{\SI{100}{e$^-$}\xspace}
% 13 FE peaking time
\def\fepeaktime{\SI{1}{\micro\second}\xspace}
% 14 signal saturation level (SP)
\def\spsignalsat{\num{500000} electrons\xspace}
% 15 LAr N contamination
\def\nitrogencontam{\SI{25}{ppm}\xspace}
% 16 detector dead time
\def\deadtime{\SI{0.5}{\%}\xspace}
% Engineering
% 17 Cathode resistivity
\def\cathodemegohm{\SI{1}{\mega\ohm/square}\xspace}
\def\cathodegigohm{\SI{1}{\giga\ohm/square}\xspace}
% 
%\def\{\xspace}
% 19 ADC sampling frequency
\def\samplingfreq{\SI{2}{\mega\hertz}\xspace}
% 20 ADC dynamic range
\def\adcdynrange{\num{12} bits}  %{3000}:\num{1}\xspace}
\def\adcdynrangegoal{\num{13} bits} %{4070}:\num{1}\xspace}
% 21 CE power consumption (SP)
\def\cepower{\SI{50}{mW/channel}\xspace}
% 22 data to tape
\def\dataratetotape{\SI{30}{PB/year}\xspace}
% 23 SNB trigger
\def\snbtriggereff{90\% efficiency\xspace}
%\def\snbtriggervisenergy{90\% efficiency \xspace}
% 24 local E fields
\def\localefield{\SI{30}{\kV/\cm}\xspace}
% 25 non-FE noise contributions
\def\elecnoisenonfe{$<<$ \SI{1000}{e$^-$}\xspace}
% 26 impurity contrib from components
\def\larpuritycomps{$<<$ \SI{30}{ppt}\xspace}
% 28 dead channels
\def\deadchannels{\SI{1}{\%}\xspace}



% The following from phys ch-bsm 1/3/19 (was in their cls file)
\newcommand{\lsim}{{\;\raise0.3ex\hbox{$<$\kern-0.75em\raise-1.1ex\hbox{$\sim$}}\;}}
\newcommand{\gsim}{{\;\raise0.3ex\hbox{$>$\kern-0.75em\raise-1.1ex\hbox{$\sim$}}\;}}
\newcommand{\beq}{\begin{equation}}
\newcommand{\eeq}{\end{equation}}
\newcommand{\bea}{\begin{eqnarray}}
\newcommand{\eea}{\end{eqnarray}}
\newcommand{\DF}{\Delta_{4}}
\mathchardef\minus="002D
\newcommand{\dk}[1]{\textcolor{red}{#1}}
\newcommand{\dkc}[1]{\textbf{\textcolor{red}{(#1 --DK)}}}
\newcommand{\dd}[1]{\textcolor{blue}{#1}}

%Milestones (from Eric's talk Mar 12, 2019} https://indico.fnal.gov/event/20149/contribution/0/material/slides/1.pdf
\newcommand{\startpduneiispinstall}{March 2021\xspace}% Start of ProtoDUNE-II (SP) Installation: March 2021
\newcommand{\startpduneiidpinstall}{March 2022\xspace}%Start of ProtoDUNE-II (DP) Installation: March 2022
\newcommand{\sdlwavailable}{April 2022\xspace}%South Dakota Logistics Warehouse Available: April 2022
\newcommand{\cucbenocc}{October 2022\xspace}% Beneficial Occupancy of Cavern 1/CUC: October 2022
\newcommand{\accesscuccountrm}{April  2023\xspace}% CUC Counting Room Accessible: April 2023
\newcommand{\accesstopfirstcryo}{January 2024\xspace}%Top of Far Detector #1 Cryostat Accessible: January 2024
\newcommand{\startfirsttpcinstall}{August 2024\xspace}%Start of Far Detector #1 TPC Installation: August 2024
\newcommand{\accesstopsecondcryo}{January 2025\xspace}%Top of Far Detector #2 Accessible: January 2025
\newcommand{\firsttpcinstallend}{May 2025\xspace}% End of Far Detector #1 TPC Installation: May 2025
\newcommand{\startsecondtpcinstall}{August 2025\xspace}%Start of Far Detector #2 TPC Installation: August 2025
\newcommand{\secondtpcinstallend}{May 2026\xspace}%End of Far Detector #2 TPC Installation: May 2026

\newcommand{\maincavernstartexc}{(get date)\xspace}% Start exc of detector cavern 1. needed for exec summ

%Mike Kordosky: the command below is used to refer to
% planned entries in the requirements table.
%\newcommand{\rrt}[1]{\ifthenelse{\equal{#1}{}}{[RT:TBD]}{[RT:#1]}}
% remove these entries
\newcommand{\rrt}[1]{}

\newcommand{\beamturnon}{fix beam turn on date in defs.tex\xspace}

%%%%%%% Everything below here is DEPRECATED (4/30/19 AH) %%%%%%%%%%

% Names of expts or detectors -- all these go into glossary DEPRECATED
% in use
\newcommand{\cherenkov}{Cherenkov\xspace}  %nonaccel
\newcommand{\kamland}{KamLAND\xspace} %cisc dp
\newcommand{\superk}{Super--Kamiokande\xspace} %nonaccel, bsm
\newcommand{\hyperk}{Hyper--Kamiokande\xspace} %nonaccel
\newcommand{\microboone}{MicroBooNE\xspace} %used in cisc sp, hv, daq
\newcommand{\minerva}{MINERvA\xspace} %tools/meth, nuosc11
\newcommand{\nova}{NOvA\xspace} %lots
\newcommand{\lariat}{LArIAT\xspace} % calib,dppds
\newcommand{\argoneut}{ArgoNeuT\xspace} %nonaccel
%not in use
\newcommand{\kkande}{Kamiokande\xspace}  
\newcommand{\miniboone}{MiniBooNE\xspace}
\newcommand{\numi}{NuMI\xspace}
\newcommand{\larnd}{LAr ND\xspace}

% usage not checked for the rest 4/30/19
% Random -- all these go into glossary DEPRECATED
\newcommand{\lartpc}{LArTPC\xspace}
\newcommand{\globes}{GLoBES\xspace}
\newcommand{\larsoft}{LArSoft\xspace}
\newcommand{\snowglobes}{SNOwGLoBES\xspace}
\newcommand{\docdb}{DUNE DocDB\xspace}
\newcommand{\lbl}{long-baseline\xspace} %DEPRECATED

% also have in glossary; Glossary also has CERN, PSL, other big labs, etc. DEPRECATED
\newcommand{\fnal}{Fermilab\xspace} 
\newcommand{\surf}{SURF\xspace} 
\newcommand{\bnl}{BNL\xspace}
\newcommand{\anl}{ANL\xspace}
 
 %detectors and modules
% also have in glossary; THE FOLLOWING NINE TERMS ARE DEPRECATED 4/30/19
\newcommand{\detmodule}{detector module\xspace}
\newcommand{\dual}{DP\xspace}
\newcommand{\Dual}{DP\xspace}
\newcommand{\single}{SP\xspace}
\newcommand{\Single}{SP\xspace}
\newcommand{\dpmod}{DP detector module\xspace}
\newcommand{\spmod}{SP detector module\xspace}
\newcommand{\lar}{LAr\xspace}
\newcommand{\lntwo}{LN$_2$\xspace}  %used in sp-tpcelec 

%detector components SP and DP -- need to be in gloss; THESE 14 ITEMS DEPRECATED:
\newcommand{\dss}{DSS\xspace}
\newcommand{\hv}{high voltage\xspace}
\newcommand{\fcage}{field cage\xspace}
\newcommand{\fc}{FC\xspace}
\newcommand{\fcmod}{FC module\xspace}  %%%   don't need?
\newcommand{\topfc}{top FC\xspace}
\newcommand{\botfc}{bottom FC\xspace}
\newcommand{\ewfc}{endwall FC\xspace}
\newcommand{\pdsys}{PD system\xspace}
\newcommand{\phdet}{photon detector\xspace}
\newcommand{\sipm}{SiPM\xspace}
\newcommand{\pmt}{PMT\xspace}
\newcommand{\pwrsupp}{power supply\xspace}
\newcommand{\pwrsupps}{power supplies\xspace}

% Add Computing specific definitions that aren't in defs.tex

% Example
% \def\tdrtitle{Technical Design Report}

\def\testanne{Testing by Anne}

\newcommand{\hideme}[1]{{\it( #1)}}



% This holds definitions of macros to enforce consistency in units.

% This file is the sole location for such definitions.  Check here to
% learn what there is and add new ones only here.  

% also see defs.tex for names.


% see
%  http://ctan.org/pkg/siunitx
%  http://mirrors.ctan.org/macros/latex/contrib/siunitx/siunitx.pdf

% Examples:
%  % angles
%  \ang{1.5} off-axis
%
%  % just a unit
%  \si{\kilo\tonne}
%
%  % with a value:
%  \SI{10}{\mega\electronvolt}

%  range of values:
% \SIrange{60}{120}{\GeV}

% some shorthand notation
%\DeclareSIUnit \MBq {\mega\Bq}
\DeclareSIUnit \s {\second}
\DeclareSIUnit \MB {\mega\byte}
\DeclareSIUnit \GB {\giga\byte}
\DeclareSIUnit \TB {\tera\byte}
\DeclareSIUnit \PB {\peta\byte}
\DeclareSIUnit \Mbps {\mega\bit/\s}
\DeclareSIUnit \Gbps {\giga\bit/\s}
\DeclareSIUnit \Tbps {\tera\bit/\s}
\DeclareSIUnit \Pbps {\peta\bit/\s}
\DeclareSIUnit \kton {\kilo\tonne} % changed  back to kton
\DeclareSIUnit \kt {\kilo\tonne}
\DeclareSIUnit \Mt {\mega\tonne}
\DeclareSIUnit \eV {\electronvolt}
\DeclareSIUnit \keV {\kilo\electronvolt}
\DeclareSIUnit \MeV {\mega\electronvolt}
\DeclareSIUnit \GeV {\giga\electronvolt}
\DeclareSIUnit \m {\meter}
\DeclareSIUnit \cm {\centi\meter}
\DeclareSIUnit \in {\inchcommand}
\DeclareSIUnit \km {\kilo\meter}
\DeclareSIUnit \kV {\kilo\volt}
\DeclareSIUnit \kW {\kilo\watt}
\DeclareSIUnit \MW {\mega\watt}
\DeclareSIUnit \MHz {\mega\hertz}
\DeclareSIUnit \mrad {\milli\radian}
\DeclareSIUnit \year {year}
\DeclareSIUnit \POT {POT}
\DeclareSIUnit \sig {$\sigma$}
\DeclareSIUnit\parsec{pc}
\DeclareSIUnit\lightyear{ly}
\DeclareSIUnit\foot{ft}
\DeclareSIUnit\ft{ft}
\DeclareSIUnit \ppb{ppb}
\DeclareSIUnit \ppt{ppt}
\DeclareSIUnit \samples{S}

\sisetup{inter-unit-product = \ensuremath{{}\cdot{}}}
%\def\ktyr{\si[inter-unit-product=\ensuremath{{}\cdot{}}]{\kt\year}\xspace}
\newcommand{\ktyr}{\si{\kt\year}\xspace}
%\def\Mtyr{\si[inter-unit-product=\ensuremath{{}\cdot{}}]{\Mt\year}\xspace}
\newcommand{\Mtyr}{\si{\Mt\year}\xspace}
%\def\msr{\si[inter-unit-product=\ensuremath{{}\cdot{}}]{\meter\steradian}\xspace}
\newcommand{\msr}{\si{\meter\steradian}\xspace}
%\def\ktMWyr{\si[inter-unit-product=\ensuremath{{}\cdot{}}]{\kt\MW\year}\xspace}
\newcommand{\ktMWyr}{\si{\kt\MW\year}\xspace}
% used for hyphen, obsolete now: \newcommand{\SIadj}[2]{\SI[number-unit-product = -]{#1}{#2}}
% change command definition Nov 2017 in case people copy e.g., \ktadj from CDR text.
% E.g., \ktadj{10} now renders the same as \SI{10}{\kt}
\newcommand{\SIadj}[2]{\SI{#1}{#2}}

% Adjective form of some common units (Nov 2107 changed to be same as normal form, no hyphen)
% "the 10-kt detector"

\newcommand{\ktadj}[1]{\SIadj{#1}{\kt}}
% "the 1,300-km baseline"
\newcommand{\kmadj}[1]{\SIadj{#1}{\km}}
% "a 567-keV endpoint"
\newcommand{\keVadj}[1]{\SIadj{#1}{\keV}}
% "Typical 20-MeV event"
\newcommand{\MeVadj}[1]{\SIadj{#1}{\MeV}}
% "Typical 2-GeV event"
\newcommand{\GeVadj}[1]{\SIadj{#1}{\GeV}}
% "the 1.2-MW beam"
\newcommand{\MWadj}[1]{\SIadj{#1}{\MW}}
% "the 700-kW beam"
\newcommand{\kWadj}[1]{\SIadj{#1}{\kW}}
% "the 100-tonne beam"
\newcommand{\tonneadj}[1]{\SIadj{#1}{\tonne}}
% "the 4,850-foot depth beam"
\newcommand{\ftadj}[1]{\SIadj{#1}{\ft}}
%

% Mass exposure, people like to put dots between the units
% \newcommand{\ktyr}[1]{\SI[inter-unit-product=\ensuremath{{}\cdot{}}]{#1}{\kt\year}}
% must make usage of \ktyr above consistent with this one before turning on

% Beam x mass exposure, people like to put dots between the units
\newcommand{\ktmwyr}[1]{\SI[inter-unit-product=\ensuremath{{}\cdot{}}]{#1}{\kt\MW\year}}

\newcommand{\ndmuidbeamrate}{\SI[round-mode=places,round-precision=1]{20.0}{\hertz}\xspace}

\newcommand{\cosmicmuondatayear}{\SI[round-mode=places,round-precision=0]{20.4289491035}{\tera\byte}\xspace}

\newcommand{\tpcdriftdistance}{\SI[round-mode=places,round-precision=2]{3.6}{\meter}\xspace}

\newcommand{\gdaqchannelnumberfactor}{\num[round-mode=places,round-precision=1]{1.1}\xspace}

\newcommand{\daqsamplerate}{\SI[round-mode=places,round-precision=1]{2.0}{\mega\hertz}\xspace}

\newcommand{\beamspillcycle}{\SI[round-mode=places,round-precision=1]{1.2}{\second}\xspace}

\newcommand{\tpcapaperdriftcell}{\num[round-mode=places,round-precision=2]{0.75}\xspace}

\newcommand{\snbdatavolumezs}{\SI[round-mode=places,round-precision=0]{2.4e-05}{\giga\byte}\xspace}

\newcommand{\beamreprate}{\SI[round-mode=places,round-precision=2]{0.833333333333}{\hertz}\xspace}

\newcommand{\ndrockmuonrate}{\SI[round-mode=places,round-precision=1]{10.0}{\hertz}\xspace}

\newcommand{\snbdataratefs}{\SI[round-mode=places,round-precision=1]{17.5226192958}{\mega\byte\per\second}\xspace}

\newcommand{\dunefsreadoutsize}{\SI[round-mode=places,round-precision=1]{24.8832}{\giga\byte}\xspace}

\newcommand{\betarate}{\SI[round-mode=places,round-precision=1]{11.16}{\mega\hertz}\xspace}

\newcommand{\chargezsthreshold}{\SI[round-mode=places,round-precision=1]{0.5}{\MeV}\xspace}

\newcommand{\tpcmodulemass}{\SI[round-mode=places,round-precision=0]{10.0}{\kilo\tonne}\xspace}

\newcommand{\dunenumberchannels}{\num[round-mode=places,round-precision=0]{1536000.0}\xspace}

\newcommand{\chargemaxsignalnoiseratio}{\num[round-mode=places,round-precision=0]{74.0}\xspace}

\newcommand{\ndecalmass}{\SI[round-mode=places,round-precision=0]{93.0}{\tonne}\xspace}

\newcommand{\cosmicmuondatarate}{\SI[round-mode=places,round-precision=1]{647.36816}{\kilo\byte\per\second}\xspace}

\newcommand{\dunenumberapas}{\num[round-mode=places,round-precision=0]{600.0}\xspace}

\newcommand{\dunefsreadoutrate}{\SI[round-mode=places,round-precision=1]{4.608}{\tera\byte/\second}\xspace}

\newcommand{\ndecalbeamrate}{\SI[round-mode=places,round-precision=1]{18.6}{\hertz}\xspace}

\newcommand{\cosmicmuoneventsize}{\SI[round-mode=places,round-precision=1]{2.5}{\mega\byte}\xspace}

\newcommand{\tpcapapermodule}{\num[round-mode=places,round-precision=0]{150.0}\xspace}

\newcommand{\nddatarate}{\SI[round-mode=places,round-precision=1]{1.0048}{\mega\byte\per\second}\xspace}

\newcommand{\chargehethreshold}{\SI[round-mode=places,round-precision=1]{10.0}{\MeV}\xspace}

\newcommand{\beamhighdatayear}{\SI[round-mode=places,round-precision=0]{21.9254891928}{\giga\byte}\xspace}

\newcommand{\betaratedensity}{\SI[round-mode=places,round-precision=3]{0.279}{\hertz\per\kilo\gram}\xspace}

\newcommand{\cosmicmuoncorrection}{\num[round-mode=places,round-precision=2]{1.36}\xspace}

\newcommand{\snbeventratetpc}{\SI[round-mode=places,round-precision=0]{100.0}{\hertz}\xspace}

\newcommand{\snbcloserfactor}{\num[round-mode=places,round-precision=0]{100.0}\xspace}

\newcommand{\snbeventsize}{\num[round-mode=places,round-precision=0]{48.0}\xspace}

\newcommand{\beamrunfraction}{\num[round-mode=places,round-precision=3]{0.667}\xspace}

\newcommand{\beamdatayearfs}{\SI[round-mode=places,round-precision=0]{218.230533073}{\tera\byte}\xspace}

\newcommand{\tpcfulllength}{\SI[round-mode=places,round-precision=1]{58.0}{\meter}\xspace}

\newcommand{\snbcanddatayearzs}{\SI[round-mode=places,round-precision=0]{200.88}{\giga\byte}\xspace}

\newcommand{\snbdataratehighzs}{\SI[round-mode=places,round-precision=0]{2.4e-05}{\tera\byte/\second}\xspace}

\newcommand{\ndeventrate}{\SI[round-mode=places,round-precision=1]{50.24}{\hertz}\xspace}

\newcommand{\dunefsreadoutsizeyear}{\SI[round-mode=places,round-precision=1]{145.414314891}{\exa\byte}\xspace}

\newcommand{\tpcfullwidth}{\SI[round-mode=places,round-precision=1]{14.5}{\meter}\xspace}

\newcommand{\daqchannelsperapa}{\num[round-mode=places,round-precision=0]{2560.0}\xspace}

\newcommand{\tpcapaheight}{\SI[round-mode=places,round-precision=1]{6.0}{\meter}\xspace}

\newcommand{\ndbytesperchannel}{\SI[round-mode=places,round-precision=0]{5.0}{\byte}\xspace}

\newcommand{\ndmuidmass}{\SI[round-mode=places,round-precision=0]{100.0}{\tonne}\xspace}

\newcommand{\ndmuidchannels}{\num[round-mode=places,round-precision=0]{165888.0}\xspace}

\newcommand{\gdaqtrigsamplespersnhit}{\num[round-mode=places,round-precision=0]{48.0}\xspace}

\newcommand{\beamdataratefs}{\SI[round-mode=places,round-precision=0]{6.915456}{\mega\byte\per\second}\xspace}

\newcommand{\ndbeameventratedensity}{\SI[round-mode=places,round-precision=1]{0.2}{\hertz\per\tonne}\xspace}

\newcommand{\tpcapawidth}{\SI[round-mode=places,round-precision=1]{2.3}{\meter}\xspace}

\newcommand{\cosmicmuonrate}{\SI[round-mode=places,round-precision=3]{0.258947264}{\hertz}\xspace}

\newcommand{\ndsstbeamrate}{\SI[round-mode=places,round-precision=1]{1.6}{\hertz}\xspace}

\newcommand{\tpcdrifttime}{\SI[round-mode=places,round-precision=2]{2.25}{\milli\second}\xspace}

\newcommand{\snbeventsizefs}{\SI[round-mode=places,round-precision=1]{46.08}{\tera\byte}\xspace}

\newcommand{\chargeminsignalnoiseratio}{\num[round-mode=places,round-precision=0]{15.0}\xspace}

\newcommand{\beameventsizecompressed}{\SI[round-mode=places,round-precision=1]{0.1}{\mega\byte}\xspace}

\newcommand{\ndecalchannels}{\num[round-mode=places,round-precision=0]{52224.0}\xspace}

\newcommand{\ndcosmicmuonrate}{\SI[round-mode=places,round-precision=2]{0.04}{\hertz}\xspace}

\newcommand{\ndbeameventrate}{\SI[round-mode=places,round-precision=1]{40.2}{\hertz}\xspace}

\newcommand{\gdaqronetriggeredphysicsdatarate}{\SI[round-mode=places,round-precision=0]{0.8448}{\byte\per\second}\xspace}

\newcommand{\gdaqunit}{\SI[round-mode=places,round-precision=0]{0.825}{\byte}\xspace}

\newcommand{\gdaqbetahighrateAPA}{\SI[round-mode=places,round-precision=0]{500.0}{\kilo\hertz}\xspace}

\newcommand{\dunenumbermodules}{\num[round-mode=places,round-precision=0]{4.0}\xspace}

\newcommand{\daqreadoutchannelsamples}{\num[round-mode=places,round-precision=0]{10800.0}\xspace}

\newcommand{\beameventsize}{\SI[round-mode=places,round-precision=1]{2.5}{\mega\byte}\xspace}

\newcommand{\tpcdriftvelocity}{\SI[round-mode=places,round-precision=1]{1.6}{\milli\meter\per\micro\second}\xspace}

\newcommand{\gdaqronebeamphysicsdatarate}{\SI[round-mode=places,round-precision=0]{41250.0}{\kilo\byte\per\second}\xspace}

\newcommand{\gdaqronedatarate}{\SI[round-mode=places,round-precision=0]{41.5153208448}{\mega\byte\per\second}\xspace}

\newcommand{\ndsstmass}{\SI[round-mode=places,round-precision=0]{8.0}{\tonne}\xspace}

\newcommand{\beamfsdatarate}{\SI[round-mode=places,round-precision=0]{436.461066147}{\peta\byte\per\year}\xspace}

\newcommand{\snbreadouttime}{\SI[round-mode=places,round-precision=1]{10.0}{\second}\xspace}

\newcommand{\gdaqronebetadatarate}{\SI[round-mode=places,round-precision=1]{41.25}{\mega\byte\per\second}\xspace}

\newcommand{\snbcanddataratezs}{\SI[round-mode=places,round-precision=0]{6365.63904105}{\byte\per\second}\xspace}

\newcommand{\gdaqnoncosmicoverheadfactor}{\num[round-mode=places,round-precision=1]{1.5}\xspace}

\newcommand{\gdaqronelowenergyphysicsdatarate}{\SI[round-mode=places,round-precision=0]{7.92}{\kilo\byte\per\second}\xspace}

\newcommand{\gdaqcrrateAPA}{\SI[round-mode=places,round-precision=4]{0.0004}{\hertz}\xspace}

\newcommand{\beamonfraction}{\num[round-mode=places,round-precision=0]{0.0030015}\xspace}

\newcommand{\betadatayear}{\SI[round-mode=places,round-precision=0]{52.8262940816}{\peta\byte}\xspace}

\newcommand{\betainspillyear}{\SI[round-mode=places,round-precision=0]{158.558121686}{\tera\byte}\xspace}

\newcommand{\gdaqtrigcompression}{\num[round-mode=places,round-precision=0]{2.0}\xspace}

\newcommand{\gdaqhighrateAPAs}{\num[round-mode=places,round-precision=0]{100.0}\xspace}

\newcommand{\daqdriftsperreadout}{\num[round-mode=places,round-precision=1]{2.4}\xspace}

\newcommand{\snbdatavolumehighzs}{\SI[round-mode=places,round-precision=0]{0.00024}{\tera\byte}\xspace}

\newcommand{\ndsstchannels}{\num[round-mode=places,round-precision=0]{215040.0}\xspace}

\newcommand{\gdaqroneSNphysicsdatarate}{\SI[round-mode=places,round-precision=0]{257.4}{\kilo\byte\per\second}\xspace}

\newcommand{\beamhighdatarate}{\SI[round-mode=places,round-precision=2]{0.694791666667}{\kilo\byte\per\second}\xspace}

\newcommand{\dunefsreadoutsizeminute}{\SI[round-mode=places,round-precision=1]{276.48}{\tera\byte}\xspace}

\newcommand{\daqchannelspermodule}{\num[round-mode=places,round-precision=0]{384000.0}\xspace}

\newcommand{\gdaqtrigsamplesperbetahit}{\num[round-mode=places,round-precision=0]{24.0}\xspace}

\newcommand{\snbcandrate}{\SI[round-mode=places,round-precision=0]{12.0}{\per\year}\xspace}

\newcommand{\dunedetectormass}{\SI[round-mode=places,round-precision=0]{40.0}{\kilo\tonne}\xspace}

\newcommand{\gdaqAPAchannelspercosmic}{\num[round-mode=places,round-precision=0]{2560.0}\xspace}

\newcommand{\ndeventsize}{\num[round-mode=places,round-precision=0]{4000.0}\xspace}

\newcommand{\snbdatayearfs}{\SI[round-mode=places,round-precision=0]{552.96}{\tera\byte}\xspace}

\newcommand{\dunefsreadoutsizesecond}{\SI[round-mode=places,round-precision=1]{4.608}{\tera\byte}\xspace}

\newcommand{\betainbeamyear}{\SI[round-mode=places,round-precision=0]{79.279060843}{\giga\byte}\xspace}

\newcommand{\gdaqlphighrateAPA}{\SI[round-mode=places,round-precision=0]{200.0}{\hertz}\xspace}

\newcommand{\betaeventsize}{\num[round-mode=places,round-precision=1]{100.0}\xspace}

\newcommand{\betadatarate}{\SI[round-mode=places,round-precision=1]{1.674}{\giga\byte\per\second}\xspace}

\newcommand{\betareadoutsize}{\SI[round-mode=places,round-precision=0]{150.0}{\byte}\xspace}

\newcommand{\snbcandeventsizezs}{\SI[round-mode=places,round-precision=1]{16.74}{\giga\byte}\xspace}

\newcommand{\daqreadouttime}{\SI[round-mode=places,round-precision=1]{5.4}{\milli\second}\xspace}

\newcommand{\cosmicmuonflux}{\SI[round-mode=places,round-precision=2]{5.66e-09}{\hertz\per\centi\meter\squared}\xspace}

\newcommand{\snbdataratezs}{\SI[round-mode=places,round-precision=0]{2.4e-06}{\giga\byte\per\second}\xspace}

\newcommand{\tpcfullheight}{\SI[round-mode=places,round-precision=1]{12.0}{\meter}\xspace}

\newcommand{\gdaqsnhighrateAPA}{\SI[round-mode=places,round-precision=0]{6500.0}{\hertz}\xspace}

\newcommand{\beameventoccupancy}{\num[round-mode=places,round-precision=4]{0.0005}\xspace}

\newcommand{\beamrate}{\SI[round-mode=places,round-precision=0]{8770.19567714}{\per\year}\xspace}

\newcommand{\daqbytespersample}{\SI[round-mode=places,round-precision=1]{1.5}{\byte}\xspace}

%% generated file, do not edit% This file is generated, any edits may be lost.

% It defines macros which expand to corresponding
% specification values for subsystem SP-FD


\def\sp{}
\def\spadcnumberofbits{\num{12} bits}
\def\spadcsamplingfreq{$\sim\,\SI{2}{\mega\hertz}$}
\def\spapagaps{$<\,\SI{15}{mm}$ between APAs on same support beam; $<\,\SI{30}{mm}$ between APAs on different support beams}
\def\spapawireangles{\SI{0}{\degree} for collection wires, $\pm\,$\SI{35.7}{\degree} for induction wires}
\def\spapawirepostolerance{$\pm\,\SI{0.5}{mm}$}
\def\spapawirespacing{\SI{4.669}{mm} for U,V; \SI{4.790}{mm} for X,G}
\def\spcathoderesistivity{$>\,\SI{1}{\mega\ohm/square}$}
\def\spcepowerconsumption{$<\,\SI{50}{ mW/channel} $}
\def\spcryomonitordevices{}
\def\spdataratetotape{$<\,\SI{30}{PB/year}$}
\def\spdeadchannels{$<\,\SI{1}{\%}$}
\def\spdpdetmoduptime{$>\,$90\%}
\def\spdpdetuptime{$>\,$98\%}
\def\spfepeaktime{\SI{1}{\micro\second}}
\def\sphvpsripple{$<\,\SI{100}e^-$}
\def\sphvsfielduniformity{$<\,\SI{1}{\%}$ throughout volume}
\def\splarimpuritycontrib{$<<\,\SI{30}{ppt} $}
\def\splarncontamination{$<\,\SI{25}{ppm}$}
\def\splarpurity{$<$\,\SI{100}{ppt}}
\def\splightyield{$>\,\SI{20}{PE/MeV}$ (avg), $>\,\SI{0.5}{PE/MeV}$ (min)}
\def\splocalefields{$<\,\SI{30}{kV/cm}$}
\def\spmindriftfield{$>$\,\SI{250}{ V/cm}}
\def\spmisalignmentfielduniformity{$<\,1\,$\% throughout volume}
\def\spnonfenoise{$<<\,\SI{1000}\,e^- $}
\def\spradiopurity{less than that from $^{39}$Ar}
\def\spsntrigger{$>\,\SI{95}{\%}$ efficiency for a SNB producing at least 60 interactions with a neutrino energy >10 MeV in 12 kt of active detector mass during the first 10 seconds of the burst.}
\def\spspsignalsaturation{\num{500000} $e^-$}
\def\spsystemnoise{$<\,\SI{1000}\,e^-$}
\def\sptimeresolutionpds{$<\,\SI{1}{\micro\second}$}

% This file is generated, any edits may be lost.

% It defines macros which expand to corresponding
% specification values for subsystem DP-FD


\def\dp{}
\def\dpdenonfenoise{$<<\,\SI{1000}\,e^- $}
\def\dpdpadcnumberofbits{\num{12} bits}
\def\dpdpadcsamplingfreq{$\sim\,\SI{2.5}{\mega\hertz}$}
\def\dpdpcathoderesistivity{$>\,\SI{1}{\mega\ohm/square}$}
\def\dpdpcepowerconsumption{$<\,\SI{50}{ mW/channel} $}
\def\dpdpcrpeffgain{\num{6}}
\def\dpdpcrpgaps{$<\,\SI{30}{mm}$ between adjacent CRPs}
\def\dpdpcrpplanarity{$\pm\,\SI{0.5}{mm}$}
\def\dpdpcrpstripspacing{$<\,\SI{4.7}{mm}$}
\def\dpdpcryomonitordevices{}
\def\dpdpdataratetotape{$<\,\SI{30}{PB/year}$}
\def\dpdpdeadchannels{$<\,\SI{1}{\%}$}
\def\dpdpdetmoduptime{$>\,$90\%}
\def\dpdpdetuptime{$>\,$98\%}
\def\dpdpfepeaktime{\SI{1}{\micro\second}}
\def\dpdphvpsripple{$<\,\SI{100}e^-$}
\def\dpdphvsfielduniformity{$<\,\SI{1}{\%}$ throughout volume}
\def\dpdplarimpuritycontrib{$<<\,\SI{30}{ppt} $}
\def\dpdplarncontamination{$<\,\SI{3}{ppm}$}
\def\dpdplarpurity{$>$\,\SI{5}{ms}}
\def\dpdplightyield{$>\,\SI{1}{PE/MeV}$ (at anode), $>\,\SI{5}{PE/MeV}$ (avg  over active volume)}
\def\dpdplocalefields{$<\,\SI{30}{kV/cm}$}
\def\dpdpmindriftfield{$>$\,\SI{250}{ V/cm}}
\def\dpdpmisalignmentfielduniformity{$<\,1\,$\% throughout volume}
\def\dpdpradiopurity{less than that from $^{39}$Ar}
\def\dpdpsignalsaturation{\num{7500000} electrons}
\def\dpdpsntrigger{$>\,\SI{95}{\%}$ efficiency for a SNB producing at least 60 interactions with a neutrino energy >10 MeV in 12 kt of active detector mass during the first 10 seconds of the burst.}
\def\dpdpsystemnoise{$<\,\SI{1000}\,e^-$}
\def\dpdptimeresolutionpds{$<\,\SI{1}{\micro\second}$}

% This file is generated, any edits may be lost.

% It defines macros which expand to corresponding
% specification values for subsystem SP-APA


\def\spapaactivearea{Maximize total active area.}
\def\spapabadchannels{$<$1\%, with a goal of $<$0.5\%}
\def\spapabiasvoltage{The setup, including boards, must hold 150\% of max operating voltage.}
\def\spapaframeplanarity{$<$\SI{5}{mm}}
\def\spapaunitsize{\SI{6.0}{m} tall $\times$ \SI{2.3}{m} wide}
\def\spapawiretension{\SI{6}{N} $\pm$ \SI{1}{N}}

% This file is generated, any edits may be lost.

% It defines macros which expand to corresponding
% specification values for subsystem DP-CRP


\def\dpcrpvertprecision{$<\,\SI{1}{\milli\meter}$}

% This file is generated, any edits may be lost.

% It defines macros which expand to corresponding
% specification values for subsystem SP-HV


\def\sphvconnectionredundancy{Two-fold}
\def\sppowersupplystability{$>\,\SI{95}{\%}$ uptime}

% This file is generated, any edits may be lost.

% It defines macros which expand to corresponding
% specification values for subsystem DP-HV


\def\dphvdbredundancy{$>\,\num{2}$ HVDB chain}

% This file is generated, any edits may be lost.

% It defines macros which expand to corresponding
% specification values for subsystem SP-PDS


\def\spapainstall{$>\,\SI{1}{\milli\meter}$}
\def\spenvhumiditylimit{$<\,\SI{50}{\%}$ RH at \SI{70}{\degree F}}
\def\spenvlightexposure{No exposure to sunlight. All other unfiltered sources: $<\,\num{30}$ minutes integrated across all exposures}
\def\splighttightness{Cryostat light leaks responsible for $<\,\SI{10}{\%}$  of data transferred from PDS to DAQ}
\def\spmechcompatibility{$>\,\SI{1}{\milli\meter}$}
\def\spmechdeflection{$<\,\SI{5}{\milli\meter}$}
\def\sppdscable{$<\,\SI{6}{\milli\meter}$}
\def\sppdscablemate{\SI{0}{\milli\meter} separation mechanically allowed}
\def\sppdscleanroomassbly{Class \num{100000} clean assembly area}
\def\sppdsclearance{$>\,\SI{0.5}{\milli\meter}$}
\def\sppdsdarkrate{$<\,\SI{1}{kHz}$}
\def\sppdsdatarate{$<\,\SI{8}{Gbps}$}
\def\sppdsdynamicrange{$<\,\SI{20}{\%}$}
\def\sppdssignaltonoise{$>\,\num{4}$}
\def\spspatiallocalization{$<\,\SI{2.5}{\meter}$}

% This file is generated, any edits may be lost.

% It defines macros which expand to corresponding
% specification values for subsystem DP-PDS


\def\dphitrelativetiming{$<\,\SI{100}{ns RMS}$}
\def\dphitsnr{$>\,\num{5}$}
\def\dppdsanalogrange{$>\,\SI{100}{PE/(ch\times 6\nano\second)}$}
\def\dppmtdarkrate{$<\,\SI{100}{kHz}$}

% This file is generated, any edits may be lost.

% It defines macros which expand to corresponding
% specification values for subsystem SP-DAQ


\def\spdaqdeadtime{}
\def\spDAQreadout{1.5 TB/s per single phase detector module}
\def\spDAQreadoutwindow{10 $\mu$s < readout window < 100 s}
\def\spDAQthroughput{10 Gb/s average storage throughput; 100 Gb/s peak temporary storage throughput per single phase detector module}
\def\spdatarecord{}
\def\spdataverification{}
\def\sptriggercalibration{}
\def\sptriggerhighenergy{$>$\SI{100}{\MeV}}
\def\sptriggerlowenergy{$>$\SI{10}{\MeV}}

% This file is generated, any edits may be lost.

% It defines macros which expand to corresponding
% specification values for subsystem DP-DAQ


\def\dpdaqdeadtime{}
\def\dpDAQreadout{\SI{65}{GB/s} per dual phase detector module}
\def\dpDAQreadoutwindow{\SI{10}{\micro\second} < readout window < \SI{100}{s}}
\def\dpDAQthroughput{\SI{2}{GB/s} average storage throughput; \SI{100}{GB/s} peak temporary storage throughput per dual phase detector module}
\def\dpdatarecord{}
\def\dpdataverification{}
\def\dptriggercalibration{}
\def\dptriggerhighenergy{$>\,$\SI{100}{\MeV}}
\def\dptriggerlowenergy{$>\,$\SI{10}{\MeV}}

% This file is generated, any edits may be lost.

% It defines macros which expand to corresponding
% specification values for subsystem SP-CISC


\def\spcameracoldcoverage{$>\,$80\% of HV surfaces}
\def\speleclifetimeprec{$<\,$1.4\%}
\def\speleclifetimerange{\SIrange{0.04}{10}{ms} in cryostat, \SIrange{0.04}{30}{ms} inline}
\def\spinstefield{$<\,\SI{30}{kV/cm}$}
\def\spinstnoise{$<<\,\SI{1000}\,e^- $}
\def\spslowcontrolalarmrate{$<\,$150/day}
\def\spslowcontrolarchiverate{\SI{0.02}{Hz}}
\def\spslowcontrolnumvars{$>\,\num{150000}$}
\def\sptemprepro{$<\,\SI{5}{mK}$}
\def\sptempstability{$<\,\SI{2}{mK}$ at all places and times}

% This file is generated, any edits may be lost.

% It defines macros which expand to corresponding
% specification values for subsystem DP-CISC


\def\dpcameracoldcoverage{$>\,$80\% of HV surfaces}
\def\dpeleclifetimeprec{$<\,$2.3\%}
\def\dpeleclifetimerange{\SIrange{0.04}{10}{ms} in cryostat, \SIrange{0.04}{30}{ms} inline}
\def\dpinstefield{$<\,\SI{30}{kV/cm}$}
\def\dpinstnoise{$<<\,\SI{1000}\,e^- $}
\def\dpslowcontrolalarmrate{$<\,$150/day}
\def\dpslowcontrolarchiverate{\SI{0.02}{Hz}}
\def\dpslowcontrolnumvars{$>\,\num{150000}$}
\def\dptemprepro{$<\,\SI{5}{mK}$}
\def\dptempstability{$<\,\SI{2}{mK}$ at all places and times}

% This file is generated, any edits may be lost.

% It defines macros which expand to corresponding
% specification values for subsystem SP-ELEC


\def\spcoldcablesxsec{\SI{6.35}{cm} (2.5")}
\def\spFEMBdatalink{\num{4} at \SI{1.28}{Gbps}}
\def\spgainFEamplifier{$\sim\SI{20}{mV/fC}$}
\def\spnumchannelsFEMB{\num{128}}
\def\spnumFEbaselines{\num{2}}
\def\spnumberFEMBperWIB{\num{4}}
\def\spsyncronizationCE{\SI{50}{ns}}
\def\spWIBdatalink{\SI{10}{Gbps}}

% This file is generated, any edits may be lost.

% It defines macros which expand to corresponding
% specification values for subsystem DP-ELEC



% This file is generated, any edits may be lost.

% It defines macros which expand to corresponding
% specification values for subsystem SP-CALIB


\def\spdatavolumelaser{$>\SI{184}{TB/yr/10 kt}$}
\def\spdatavolumepns{$>\SI{144}{TB/yr/10 kt}$}
\def\spefieldcalibcoverage{$>\SI{75}{\%}$}
\def\spefieldcalibgranularity{$~30\times 30\times 30~$\SI{}{\centi\metre\cubed}}
\def\spefieldcalibprecision{$\SI{1}{\%}$}
\def\splaserlocationprecision{$~\SI{0.5}{\milli\radian}$}
\def\spneutronsourcecoverage{$>\SI{75}{\%}$}

% This file is generated, any edits may be lost.

% It defines macros which expand to corresponding
% specification values for subsystem DP-CALIB


\def\dpdatavolumelaser{$>\,\SI{37}{TB/yr/10 kt}$}
\def\dpdatavolumepns{$>\,\SI{170}{TB/yr/10 kt}$}
\def\dpefieldcalibcoverage{$>\,\SI{93}{\%}$}
\def\dpefieldcalibgranularity{\SI{10 x 10 x 10}{\centi\metre}}
\def\dpefieldcalibprecision{\SI{1}{\%}}
\def\dplaserpositionprecision{\SI{0.5}{\milli\radian}}
\def\dpneutronsourcecoverage{$>\,\SI{75}{\%}$}

% This file is generated, any edits may be lost.

% It defines macros which expand to corresponding
% specification values for subsystem SP-INST


\def\spapastoragesd{700 m$^2$}
\def\spcleanroomspecification{ISO 8}
\def\spcleanroomuvfilters{filter $<\SI{400}{nm}$ for $>$ 2 wk exp; $<\SI{520}{nm}$ all else}
\def\splogisticsmaterialhandling{SURF Material Handling Specification}
\def\splogisticsmaterialsbuffer{$>1$ month}
\def\splogisticsshippingcoord{2 wk notice to CMGC}

% This file is generated, any edits may be lost.

% It defines macros which expand to corresponding
% specification values for subsystem DP-INST


\def\dpcleanroomspecs{>ISO-8}
\def\dpcoldboxcryo{Independent operation of the four cold boxes}
\def\dphandlingspecs{Fulfill material handling specification}
\def\dpmaterialbuffer{>1 month}
\def\dpundergroundstorage{>4~\dshorts{crp}}



% For abbrev/acronym lists, see also init.tex for entry point of acronyms.tex.
% \usepackage[intoc]{nomencl}
% \makenomenclature
% \renewcommand{\nomname}{Acronyms, Abbreviations and Terms}
% \setlength{\nomlabelwidth}{0.2\textwidth}

% For glossaries, needs to be loaded *after* hyperref to get clickable links.
% See dune-words.tex for detailed explanation.

% http://mirrors.ctan.org/macros/latex/contrib/glossaries/glossaries-user.pdf

% \usepackage[acronyms,toc]{glossaries}
\usepackage[toc]{glossaries}
\makeglossaries


% for terms with acronyms
\newcommand{\dshort}[1]{\glsentrytext{#1}}  % doesn't provide link
\newcommand{\dshorts}[1]{\glsentryshortpl{#1}}  % doesn't provide link
\newcommand{\dlong}[1]{\glsentrylong{#1}}  % doesn't provide link
\newcommand{\dlongs}[1]{\glsentrylongpl{#1}}  % doesn't provide link

% force the "first time" behavior
% \newcommand{\dfirst}[1]{\glsfirst{#1}}
\newcommand{\dfirst}[1]{\glsfirst{#1}\glsunset{#1}}
\newcommand{\dfirsts}[1]{\glsfirstplural{#1}\glsunset{#1}}

\newcommand{\dword}[1]{\gls{#1}}
\newcommand{\dwords}[1]{\glspl{#1}}
\newcommand{\Dword}[1]{\Gls{#1}}
\newcommand{\Dwords}[1]{\Glspl{#1}}


% use this to define terms that do NOT have acronyms.
% \newduneword{label}{term}{description}
\newcommand{\newduneword}[3]{
    \newglossaryentry{#1}{
        text={#2},
        long={#2},
        name={\glsentrylong{#1}},
        first={\glsentryname{#1}},
        firstplural={\glsentrylong{#1}\glspluralsuffix},
        description={#3},
        sort={#2}
    }
}

% use this to define terms that DO have acronyms.
%                 1      2     3       4 
% \newduneabbrev{label}{abbrev}{term}{description}
%%%% note: there is something wonky about capitalization which
%%%% is why \glsentry* isn't used in some of the arguments below.
\newcommand{\newduneabbrev}[4]{
  \newglossaryentry{#1}{
    text={#2},
    long={#3},
    shortplural={{#2}\glspluralsuffix},
    longplural={{#3}\glspluralsuffix{}},
    name={\glsentrylong{#1}{} (\glsentrytext{#1}{})},
    first={#3 (#2)},
    firstplural={#3\glspluralsuffix{} (\glsentrytext{#1}\glspluralsuffix{})},
    description={#4},
    sort={#2}
  }
}

%% If plural needs special spelling besides adding an "s"
%                 1      2     3       4        5
% \newduneabbrev{label}{abbrev}{term}{terms}{description}
\newcommand{\newduneabbrevs}[5]{
  \newglossaryentry{#1}{
    text={#2},
    long={#3},
    plural={#4},
    shortplural={{#2}\glspluralsuffix},
    longplural={#4},
    name={\glsentrylong{#1}{} (\glsentrytext{#1}{})},
    first={#3 (#2)},
    firstplural={#4 (\glsentrytext{#1}\glspluralsuffix{})},
    description={#5},
    sort={#2}    
  }
}


% tres meta
\newduneword{dword}{DUNE Word}{A term in the DUNE lexicon}

%%%%%%     START ADDING WORDS, IN ALPHABETICAL ORDER IF POSSIBLE!    %%%%%%%%
\newduneword{nasa}{NASA}{U.S. National Aereonautics and Space Administration}

%near detector
\newduneabbrev{nd}{ND}{near detector}{Refers to the detector(s) %or more  generally the experimental site
 installed close to the neutrino source at \gls{fnal} }

%far detector
\newduneabbrev{fd}{FD}{far detector}{The \SI{70}{kt} total (\fdfiducialmass fiducial) mass \gls{lartpc} DUNE detector, composed of four \larmass total (\nominalmodsize fiducial) mass modules,  %or more generally the experimental 
  to be installed at the far site at \gls{surf} in
  Lead, SD, USA}

%single-phase
\newduneabbrev{sp}{SP}{single-phase}{Distinguishes one of the DUNE far detector technologies by the fact that it operates using argon in its liquid phase only}
  %Distinguishes one of the four  \SI{10}{\kton} \glspl{detmodule} of the DUNE far detector by the  fact that it operates using argon in just its liquid phase}
  

%dual-phase
\newduneabbrev{dp}{DP}{dual-phase}{Distinguishes one of the DUNE far detector technologies by the fact that it operates using argon 
 in both gas and liquid phases; sometimes called double-phase} % 1 sept2020; Anne added `double' after SSR mentioned that EU tends to use it

%photon detection system
\newduneabbrev{pds}{PDS}{photon detection system}{The detector 
  subsystem sensitive to light produced in the \gls{lar} }

%high voltage system
\newduneabbrev{hvs}{HVS}{high voltage system}{The detector 
  subsystem that provides the \gls{tpc} drift field}

%time projection chamber
\newduneabbrev{tpc}{TPC}{time projection chamber}{A type of particle detector that uses an \efield together with a sensitive volume of gas or liquid, e.g., \gls{lar}, to perform a \threed reconstruction of a particle trajectory or interaction. The activity is recorded by digitizing the waveforms of current
  induced on the anode as the distribution of ionization charge passes by
  or is collected on the electrode (TPC is also used for ``total project cost'')} 

%liquid argon time-projection chamber
\newduneabbrev{lartpc}{LArTPC}{liquid argon time-projection chamber}{A \gls{tpc} filled with liquid argon; %A class of detector technology that this technology forms 
the basis for the \gls{dune} \gls{fd} modules} %.   It typically entails observation of ionization activity by  electrical signals and of scintillation by optical signals}

%anode plane assembly
\newduneabbrevs{apa}{APA}{anode plane assembly}{anode plane assemblies}{A unit of the \gls{sphd}
  detector module containing the elements sensitive to ionization in the \gls{lar}. 
  It contains two faces each of three planes of wires, and interfaces to the cold
  electronics and photon detection system} 

\newduneabbrev{awg}{AWG}{American wire gauge} {U.S. standard set of non-ferrous wire conductor sizes}

\newduneabbrev{ufer}{Ufer}{concrete encased electrode} {U.S. National Electrical Code grounding method refered to as Concrete Encased Electrode}

%charge readout
\newduneabbrev{cro}{CRO}{charge readout}{The system for detecting
  ionization charge distributions in a \gls{dp} detector module}

%light readout
\newduneabbrev{lro}{LRO}{light readout}{The system for detecting
  scintillation photons in a \gls{dp}  detector module}

%safe high voltage
\newduneabbrev{shv}{SHV}{safe high voltage}{Type of bayonet mount
connector used on coaxial cables that has additional insulation 
compared to standard BNC and MHV connectors that makes it safer
for handling \gls{hv} by preventing accidental contact with the
live wire connector in an unmated connector or plug}

%front-end
\newduneabbrev{fe}{FE}{front-end}{The front-end refers to a point that is
  ``upstream'' of the data flow for a particular subsystem. 
 For example the \gls{sphd} front-end electronics is where the cold electronics
  meet the sense wires of the TPC and the front-end \gls{daq} is where the \gls{daq} meets the output of the electronics}

\newduneabbrev{daqrou}{DAQ RU}{DAQ readout unit}{The first element in the data flow of the \gls{daq}}

\newduneabbrev{cots}{COTS}{commercial off-the-shelf}{Items, typically hardware such as 
computers, that may be purchased whole, without any custom design or fabrication and 
thus at normal consumer prices and availability}

\newduneabbrev{i2c}{I2C}{Inter-Integrated Circuit}{I$^2$C or I2C is a synchronous, 
multi-master, multi-slave, packet switched, single-ended, serial computer bus widely used 
for attaching lower-speed peripheral ICs to processors and microcontrollers in short-distance, 
intra-board communication} %leave upper case

\newduneabbrev{spi}{SPI}{Serial Peripheral Interface}{The Serial Peripheral Interface is a 
synchronous serial communication interface specification used for short distance 
communication, primarily in embedded systems}%leave upper case

\newduneabbrev{miso}{MISO}{master in slave out}{The Master In Slave Out is a logic
signal on the \gls{spi} bus on which the data from the slave are transmitted once
a request from the master is received} %leave upper case

\newduneabbrev{mosi}{MOSI}{master out slave in}{The Master Out Slave In is a logic
signal on the \gls{spi} bus on which the data from the master is transmitted} %leave upper case

\newduneabbrev{uart}{UART}{Universal Asynchrous Receiver/Transmitter}{A universal 
asynchronous receiver-transmitter is a computer hardware device for asynchronous 
serial communication in which the data format and transmission speeds are configurable}%leave upper case

\newduneword{cr}{CR}{Capacitance-Resistance} %leave upper case

\newduneword{dc}{DC}{direct coupling} % I think these are ok lower case (anne)

\newduneword{ac}{AC}{Alternating Current; when used in the phrase ``AC coupling'' refers to a circuit element that filters out low-frequency components, such as constant offsets, leaving higher frequency signal components. The frequency filtering is determined both by a resistor and a capacitor}

\newduneabbrev{pll}{PLL}{Phase-Locked Loop}{A control system that generates an
output signal whose phase is related to the phase of an input signal}  %leave upper case

\newduneword{fifo}{FIFO}{First-In-First-Out} % leave in upper case

\newduneword{tsmc}{TSMC}{Taiwan Semiconductor Manufacturing Company}

\newduneword{saci}{SACI}{\gls{slac} \gls{asic} Control Interface}

\newduneword{om3}{OM3}{Type of multi-mode fiber optic cable, typically capable of \SI{10}{Gbps} data transmission at lengths up to \SI{300}{m}}

\newduneword{om4}{OM4}{Type of multi-mode fiber optic cable, typically capable of \SI{10}{Gbps} data transmission at lengths up to \SI{550}{m}}

\newduneword{qfp}{QFP}{Quad Flat Package} % leave in upper case

\newduneabbrev{ams}{AMS}{analog and mixed signal}{Verilog-AMS is a derivative of the Verilog hardware description language that includes analog and mixed-signal extensions (AMS) in order to define the behavior of analog and mixed-signal systems}

\newduneabbrev{hepa}{HEPA}{High Efficiency Particulate Air}{The High Efficiency Particulate Air filters are a type of air filter that remove 99.97\% of particles that have a size greater than or equal to \SI{0.3}{\micro\meter}}  % leave in upper case

\newduneabbrev{uvm}{UVM}{universal verification methodology}{The Universal Verification Methodology is a standardized methodology for verifying integrated circuit designs}   % leave in upper case

\newduneword{lhc}{LHC}{Large Hadron Collider}

\newduneabbrev{lsb}{LSB}{least significant bit}{The bit with the lowest numerical value in a binary number}

\newduneabbrev{ldo}{LDO}{low-dropout regulator}{A low-dropout or LDO regulator is a \gls{dc} linear voltage regulator that can regulate the output voltage even when the supply voltage is very close to the output voltage}

%analog digital converter
\newduneabbrev{adc}{ADC}{analog-to-digital converter}{A sampling of a voltage
  resulting in a discrete integer count corresponding in some way to
  the input}

\newduneabbrev{inl}{INL}{integral non-linearity}{A commonly used measure of performance in \glspl{adc}. It is the deviation between the ideal input threshold value and the measured threshold level of a certain output code}

\newduneabbrev{dnl}{DNL}{differential non-linearity}{A commonly used measure of performance in \glspl{adc}. The DNL error is defined as the difference between an actual step width and the ideal value of one \gls{lsb}}

\newduneword{pnp}{PNP}{Type of bipolar junction transistor consistning of a
layer of N-doped semiconductor sandwiched between two layers of P-doped material}

\newduneword{spice}{SPICE}{SPICE
(``Simulation Program with Integrated Circuit Emphasis'') is a general-purpose, 
open-source analog electronic circuit simulator. It is a program used in integrated 
circuit and board-level design to check the integrity of circuit designs and to 
predict circuit behavior}

%data acquisition
\newduneabbrev{daq}{DAQ}{data acquisition}{The data acquisition system
  accepts data from the detector \gls{fe} electronics, buffers
  the data, performs a \gls{trigdecision}, builds events from the selected
  data and delivers the result to the offline \gls{diskbuffer}}

% interval of validity
\newduneabbrev{iov}{IOV}{interval of validity}{Interval over which something is valid}

%CALC
\newduneword{calci}{CALCI}{Calibration and Cryogenic Instrumentation}

%detector module
\newduneword{detmodule}{far detector module}{The entire DUNE far detector is
  segmented into four modules, each with a nominal \SI{10}{\kton}
  fiducial mass}
  
\newduneword{module}{module}{Many aspects of the DUNE far and near detectors are modular, so ``module'' must be understood in context. It may refer to one of the four far detector modules, distinct portions of a subdetector as in a ``field cage module,'' a software or electronics module, e.g., a separate framework plug-in, and so on}

%detector unit  
\newduneword{detunit}{detector unit}{A portion of a \gls{detmodule} may be further partitioned into a number of similar parts.   For example, the \gls{sphd} \gls{tpc} is made up of \gls{apa}  units (and other elements)}


%secondary DAQ buffer
\newduneword{diskbuffer}{secondary DAQ buffer}{A secondary
  \gls{daq} buffer holds a small subset of the full rate as
  selected by a \gls{trigcommand}. 
  This buffer also marks the interface with the DUNE Offline}

%online  monitoring
\newduneabbrev{om}{OM}{online monitoring}{Processes that run inside
  the \gls{daq} on data ``in flight,'' specifically before landing on the
  offline disk buffer, and that provide feedback on the operation of
  the \gls{daq} itself and the general health of the data it is marshalling}

%data quality monitoring
\newduneabbrev{dqm}{DQM}{data quality monitoring}{Analysis of the raw
  data to monitor the integrity of the data and the performance of the
  detectors and their electronics. This type of monitoring may be
  performed in real time, within the \gls{daq} system, or in later
  stages of processing, using disk files as input}

%DAQ dump buffer
\newduneword{dumpbuffer}{DAQ dump buffer}{This \gls{daq} buffer
  accepts a high-rate data stream, in aggregate, from an associated
%  \gls{submodule} (elim by anne)
  portion of a \gls{detmodule} sufficient to collect all data likely relevant to
  a potential \gls{snb}}
\newduneabbrev{etf}{ETF}{Experiment Test Framework}{\gls{wlcg} testing middleware running grid jobs that actively test distributed sites services and capabilities, and report back to monitoring services}

%Global Trigger Logic
\newduneabbrev{etl}{ETL}{external trigger logic}{Trigger processing
  that consumes \gls{detmodule} level \gls{trignote} information
  and other global sources of trigger input and emits
  \gls{trigcommand} information back to the \glspl{mtl}}
\newduneabbrev{daqeti}{ETI}{external trigger interface}{Interface between \glspl{mtl} and external source and sinks of relevant trigger information}

%trigger notification
\newduneword{trignote}{trigger notification}{Information provided by
  \gls{mtl} to \gls{etl} about \gls{trigdecision} %its 
  processing}

%trigger primitive
\newduneword{trigprimitive}{trigger primitive}{Information derived by
  the \gls{daq} \gls{fe} hardware that describes a region of space (e.g.,
  one or several neighboring channels) and time (e.g., a contiguous set
  of \gls{adc} sample ticks) associated with some activity}

%external trigger candidate
\newduneword{externtrigger}{external trigger candidate}{Information
  provided to the \gls{mtl} about events external to a
  \gls{detmodule} so that it may be considered in forming
  \glspl{trigcommand}}

%Out-of-band trigger command dispatcher
\newduneabbrev{daqoob}{OOB dispatcher}{out-of-band trigger command
  dispatcher}{This component is responsible for dispatching a \gls{snb} dump
  command to all \glspl{daqfer} in the \gls{detmodule}}

%module trigger logic
\newduneabbrev{mtl}{MTL}{module trigger logic}{Trigger processing
  that consumes \gls{detunit} level \gls{trigcommand} information
  and emits \glspl{trigcommand}. 
  It provides the \gls{etl} with \glspl{trignote} and receives back any
  \glspl{externtrigger}}

%octant
\newduneword{octant}{octant}{Any of the eight parts into which 4$\pi$
  is divided by three mutually perpendicular axes. 
  In particular in referencing the value for the mixing angle
  $\theta_{23}$}

%sub-detector ??? %%%%%%%%%%%%%%%%%%%     CHeck if used (Anne)  %%%%%% ???????
%\newduneword{submodule}{subdetector}{A detector unit of granularity less  than one \gls{detmodule} such as the TPC of either a \gls{sp} or \gls{dp}   module}

%trigger candidate
\newduneword{trigcandidate}{trigger candidate}{Summary information derived
  from the full data stream and representing a contribution toward
  forming a \gls{trigdecision}}

%trigger command
\newduneword{trigcommand}{trigger command}{Information derived from
  one or more \glspl{trigcandidate}  that directs elements of the
  \gls{detmodule} to read out a portion of the data stream}

%trigger command message
\newduneabbrev{tcm}{TCM}{trigger command message}{A message flowing
  down the trigger hierarchy from global to local context.  Also see \gls{tpm}}

\newduneabbrev{mlt}{MLT}{module level trigger}{The \gls{daq} component responsible for producing a \gls{trigdecision} that will be used to command the readout of a detector module}

%trigger decision
\newduneword{trigdecision}{trigger decision}{The process by which
  \glspl{trigcandidate} are converted into \glspl{trigcommand}}

%trigger primitive message
\newduneabbrev{tpm}{TPM}{trigger primitive message}{A message flowing
  up the trigger hierarchy from local to global context.  Also see \gls{tcm}}

\newduneabbrev{ipc}{IPC}{inter-process communication}{A system for software elements to exchange information between threads, local processes or across a data network.  An IPC system is typically specified in terms of protocols  composed of message types and their associated data schema}

\newduneword{daqdispre}{discovery and presence}{As used in the context of the \gls{ipc}, a system that provides mechanisms for a node on a communication network to learn of the existence of peers and their identity (discovery) as well as determine if they are currently operational or have become unresponsive (presence)}

\newduneabbrev{pubsub}{PUB/SUB}{publish-subscribe communication pattern}{An \gls{ipc} communication pattern where one element, the publisher, sends data to all connected elements, the subscribers.  Each subscriber may connect to multiple publishers.  A variant is PUB/SUB with topics where a subscriber may register an identifier, the topic, to limit the information received to just an associated subset}

%event builder
\newduneabbrev{eb}{EB}{event builder}{A software agent that executes \glspl{trigcommand}  for one  \gls{detmodule} by reading out the requested data}
%{A software agent servicing one \gls{detmodule} by executing \glspl{trigcommand} by reading out  the requested data} (Anne: awkward wording)

\newduneabbrev{daqdfo}{DFO}{data flow orchestrator}{The process by which trigger commands are executed in parallel and asynchronous manner by the back-end output subsystem of the \gls{daq}}

\newduneabbrev{daqubi}{UBI}{upstream DAQ buffer interface}{The process which provides read-only access to data residing in the upstream \gls{daq} buffers to processes on the network}

%cluster on board
\newduneabbrev{cob}{COB}{cluster on board}{An ATCA motherboard housing four RCEs}

%reconfigurable computing element
\newduneabbrev{rce}{RCE}{reconfigurable computing element}{Data processor located outside of the cryostat on a \gls{cob} that contains \gls{fpga}, RAM and \gls{ssd} resources, responsible for buffering data, producing trigger primitives, responding to triggered requests for data and synching \gls{snb} dumps}

%bump on wire
\newduneabbrev{bow}{BOW}{Bump On Wire}{A working name for the front-end readout computing elements used in the nominal \gls{daq} design to interface the \gls{dp}  crates to the \gls{daq} front-end computers}

%advanced telecommunication computing architecture
\newduneabbrev{atca}{ATCA}{Advanced Telecommunications Computing
  Architecture}{An advanced computer architecture specification developed for the telecommunications, military, and aerospace industries that incorporates the latest trends in  high-speed interconnect technologies, next-generation processors, and improved reliability, availability and serviceability} 

%Micro Telecommunications Computing Architecture
\newduneabbrev{utca}{$\mu$TCA}{Micro Telecommunications Computing Architecture}{The computer architecture specification followed by the crates that house charge and light readout electronics in the \gls{dpmod}} 

%user datagram protocol
\newduneabbrev{udp}{UDP}{user datagram protocol}{A simple,
  connectionless Internet protocol that supports data integrity
  checksums, requires no handshaking, and does not guarantee packet delivery}

%advanced meazzanine card
\newduneabbrev{amc}{AMC}{advanced mezzanine card}{Holds digitizing
  electronics and lives in \gls{utca} crates}

%radio frequency
\newduneabbrev{rf}{RF}{radio frequency}{Electromagnetic emissions
  that are within the (radio) frequency band of sensitivity of the detector
  electronics}

%field programmable gate array
\newduneabbrev{fpga}{FPGA}{field programmable gate array}{An
integrated circuit technology that allows the hardware to be reconfigured to
execute different algorithms after its manufacture and deployment}

\newduneabbrev{fmc}{FMC}{FPGA mezzanine card}{Boards holding \glspl{fpga} and other integrated circuitry that attach to a motherboard}

%Front-End Link eXchange
\newduneabbrev{felix}{FELIX}{Front-End Link eXchange}{A
  high-throughput interface between \gls{fe} and trigger electronics
  and the standard PCIe computer bus}

%DAQ partition
\newduneword{daqpart}{DAQ partition}{A cohesive and
 coherent collection of \gls{daq} hardware and software working together to trigger and read out some portion of one detector module; it consists of an integral number of \glspl{daqfrag}. 
 Multiple \gls{daq} partitions may operate simultaneously, but each instance operates independently}

%front-end computer
\newduneabbrev{fec}{DAQ FEC}{DAQ front-end computer}{The portion of one
  \gls{daqpart} that hosts the \gls{daqdr}, \gls{daqbuf} and
  \gls{daqds}.  It hosts the \gls{daqfer} and corresponding portion of the \gls{daqbuf}}

%DAQ front-end fragment
\newduneword{daqfrag}{DAQ front-end fragment}{The portion of one
  \gls{daqpart} relating to a single \gls{fec} and corresponding to an
  integral number of \glspl{detunit}.  See also \gls{datafrag}}

%data fragment
\newduneword{datafrag}{data fragment}{A block of data read out from a single \gls{daqfrag} that
span a contiguous period of time as requested by a \gls{trigcommand}}

%DAQ front-end redout
\newduneabbrev{daqfer}{FER}{DAQ front-end readout}{The portion of a
  \gls{daqfrag} that accepts data from the detector electronics and
  provides it to the \gls{fec}}

%DAQ data receiver
\newduneabbrev{daqdr}{DDR}{DAQ data receiver}{The portion of the
  \gls{daqfrag} that accepts data from the \gls{daqfer}, emits
  trigger candidates produced from the input trigger primitives, and
  forwards the full data stream to the \gls{daqbuf}}

%primary DAQ buffer
\newduneword{daqbuf}{DAQ primary buffer}{The portion
  of the \gls{daqfrag} that accepts full data stream from the
  corresponding \gls{detunit} and retains it sufficiently long for it
  to be available to produce a \gls{datafrag}}

%data selector
\newduneword{daqds}{data selector}{The portion of the \gls{daqfrag}
  that accepts \glspl{trigcommand} and returns the corresponding
  \gls{datafrag}.  Not to be confused with \gls{daqdsn}}

\newduneword{daqdsn}{data selection}{The process of forming a trigger decision for selecting a subset of detector data for output by the \gls{daq} from the content of the detector data itself.  Not to be confused with \gls{daqds}}

\newduneabbrev{daqros}{DAQ RO}{DAQ readout subsystem}{The subsystem of the \gls{daq} for accepting and buffering data input from detector electronics}

\newduneabbrev{daqdss}{DAQ DS}{DAQ data selection subsystem}{The subsystem of the \gls{daq} responsible for forming a trigger decision based on a portion of the input data stream.  The majority subset of the \gls{daqtrs}}

\newduneabbrev{daqtrs}{DAQ TS}{DAQ trigger subsystem}{The subsystem of the \gls{daq} responsible for forming a trigger decision}

\newduneabbrev{daqbes}{DAQ BE}{DAQ back-end subsystem}{The portion of the \gls{daq} that is generally toward its output end.  It is responsible for accepting and executing trigger commands and marshaling the data they address to output storage buffers}

\newduneabbrev{daqtss}{DAQ TSS}{DAQ timing and synchronization subsystem}{The portion of the \gls{daq} that provides for timing and synchronization to various components}


%front-end mother board
\newduneabbrev{femb}{FEMB}{front-end mother board}{Refers a unit of
  the \gls{sp} \gls{ce} that contains the \gls{fe} amplifier
  and \gls{adc} \glspl{asic} covering 128 channels}

%application-specific integrated circuit
\newduneword{asic}{ASIC}{application-specific integrated circuit}

%low voltage
\newduneword{lv}{LV}{low voltage}

%iceberg
\newduneabbrev{iceberg}{ICEBERG}{ICEBERG R\&D cryostat and electronics}{Integrated Cryostat and Electronics Built for Experimental Research Goals: a new double-walled cryostat built and installed at \gls{fnal}  
%build at \gls{fnal} and installed in the Proton Assembly Building meant (Anne changed) 
for liquid argon detector R\&D and for testing of DUNE detector components}

%coldadc
\newduneword{coldadc}{ColdADC}{A newly developed 16-channels \gls{asic} providing analog to digital conversion}

%coldata
\newduneword{coldata}{COLDATA}{A 64-channel control and communications \gls{asic}}
%\newduneabbrev{coldata}{COLDATA}{a 64-channel control and communications ASIC}{A key component of the 128-channel \gls{femb} that provides a control and communication interface between cold \gls{lvds} channels and warm electronics external to the cryostat}

%cryo
\newduneword{cryo}{CRYO}{(1) Integrated ASIC including \gls{fe} circuitry providing signal amplification and pulse shaping, analog to digital conversion, and control and communication functionalities for 64 channels; (2) acryonym for cryogenic systems and cryostat work scopes in \gls{lbnf}}

%liquid argon application-specific integrated circuit
\newduneword{larasic}{LArASIC}{A 16-channel \gls{fe} \gls{asic} that provides signal amplification and pulse shaping}

%complementary metal-oxide-semiconductor
\newduneword{cmos}{CMOS}{Complementary metal-oxide-semiconductor}

%equivalent noise charge
\newduneabbrev{enc}{ENC}{equivalent noise charge}{The equivalent noise charge is the input charge that corresponds to a \gls{snr}$=1$}
%equivalent number of bits - not used

%dynamic range enhancement not used

%successive approximation register
\newduneword{sar}{SAR}{successive approximation register}

%protodune
\newduneword{protodune}{ProtoDUNE}{Either of the two DUNE prototype detectors constructed at \gls{cern}. % and operated in  a CERN test beam (expected fall 2018). 
  One prototype implements \gls{sp} technology and the other \gls{dp}}
  
\newduneword{protodune2}{ProtoDUNE-II}{The second run of a \gls{protodune} detector}  % #33

%the single-phase ProtoDUNE detector
\newduneword{pdsp}{ProtoDUNE-SP}{The \gls{sphd} \gls{protodune} detector at \gls{cern}}

%the dual-phase ProtoDune detector
\newduneword{pddp}{ProtoDUNE-DP}{The \gls{dp} \gls{protodune} detector at \gls{cern}}

%WA105 dual-phase demonstrator
\newduneword{wa105}{WA105 DP demonstrator}{The \SI[product-units=power]{3x1x1}{m} WA105 \gls{dp} prototype detector at \gls{cern}}

%data aquisition event block --- includes dirty word: "event"
\newduneword{rawevent}{DAQ event block}{The unit of data output by the
  \gls{daq}.  
  It contains trigger and detector data spanning a unique, contiguous
  time period and a subset of the detector channels}

%solid-state disk
\newduneabbrev{ssd}{SSD}{solid-state disk}{Any storage device that
  may provide sufficient write throughput to receive, both collectively and
  distributed, the sustained full rate of data from a \gls{detmodule}
  for many seconds}
\newduneabbrev{nvme}{NVMe}{Non-volatile memory express}{A specification for an interface to storage media attached via PCIe}

%high-level trigger --- 
\newduneabbrev{hlt}{HLT}{high-level trigger}{This is actually a filter applied to data that has been triggered and aggregated in order to further reduce or characterize it}

%particle identification
\newduneabbrev{pid}{PID}{particle ID}{Particle identification}

%readout window
\newduneword{readout window}{readout window}{A fixed, atomic and
  continuous period of time over which data from a \gls{detmodule}, in
  whole or in part, is recorded. 
  This period may differ based on the trigger that initiated the
  readout}

%zero-suppression
\newduneabbrev{zs}{ZS}{zero-suppression}{Used to delete some portion of a
  data stream that does not significantly deviate from zero or
  intrinsic noise levels. 
  It may be applied at different granularity from per-channel to per
  \gls{detunit}}

%run control 
%\newduneabbrev{rc}{RC}{run control}{The system for configuring, starting and terminating the \gls{daq}}

\newduneword{rc}{RC}{Depending on context, one of (1) resistive-capacitive (circuit), (2) run control, the system for configuring, starting and terminating the \gls{daq}, or (3) resource coordinator, a member of the \gls{dune} management team responsible for coordinating the financial resources of the project} % #33


\newduneabbrev{daqccm}{CCM}{DAQ control, configuration and monitoring subsystem}{A system for controlling, configuring and monitoring other systems in particular those that make up the \gls{daq} where the CCM encompasses \gls{rc}}

\newduneword{daqrun}{DAQ run}{A period of time over which relevant data taking conditions and \gls{daq} configuration are asserted to be unchanged. 
  Multiple \gls{daq} runs may occur simultaneously when multiple \glspl{daqpart} are active. 
  This term should not be confused with DUNE experiment or beam ``runs'' that typically span many \gls{daq} runs}
\newduneword{daqrunnum}{DAQ run number}{A monotonically increasing count that uniquely and globally identifies a \gls{daqrun}}

%supernova neutrino burst  
\newduneabbrev{snb}{SNB}{supernova neutrino burst}{A prompt 
  increase in the flux of low-energy neutrinos emitted in the first few seconds of a core-collapse supernova.  It can also refer to a trigger command type that may be due to this phenomenon,
  or detector conditions that mimic its interaction signature}

%supernova burst and low energy
\newduneabbrev{snble}{SNB/LE}{supernova neutrino burst and low
  energy}{Supernova neutrino burst and low-energy physics program}

%supernova early warning system
\newduneabbrev{snews}{SNEWS}{SuperNova Early Warning System}{A global
  supernova neutrino burst trigger formed by a coincidence of \gls{snb} 
  triggers collected from participating experiments}

%one pulse per second signal
\newduneabbrev{pps}{1PPS signal}{one-pulse-per-second signal}{An
  electrical signal with a fast rise time and that arrives in real
  time with a precise period of one second}

%spill location system
\newduneabbrev{sls}{SLS}{spill location system}{A system residing at
  the DUNE far detector site that provides information, possibly
  predictive, indicating periods of time when neutrinos are being
  produced by the \gls{fnal} Main Injector beam spills}

%warm interface board
\newduneabbrev{wib}{WIB}{warm interface board}{Digital electronics
  situated just outside a FD cryostat that receives digital data
  from the \glspl{femb} (part of \gls{ce}) over cold copper connections and sends it to the \gls{rce}
  \gls{fe} readout hardware}

\newduneabbrev{gps}{GPS}{Global Positioning System}{A satellite-based system that provides a highly accurate \gls{pps} that may be used to synchronize clocks and determine location}

\newduneabbrev{ntp}{NTP}{Network Time Protocol}{A networking protocol that allows synchronizing of clocks to within a few \si{\milli\second} of a time standard on a local network and within a few tens of \si{\milli\second} over the Internet} 

\newduneword{ptp}{PTP}{Depending on context, either p-terphenyl, a \gls{wls} material, or Precision Time Protocol, a networking protocol that allows synchronizing of clocks to within a few \si{\micro\second} of a time standard on a local network}  % #33

\newduneabbrev{irig}{IRIG}{inter-range instrumentation group}{A standards body that defined a time-code standard for transferring timing information}

%network interfce controller
\newduneabbrev{nic}{NIC}{network interface controller}{Hardware for controlling the interface to a communication network.  Typically, one that obeys the Ethernet protocol}

%warm interface electronics crate
\newduneabbrev{wiec}{WIEC}{warm interface electronics crate}{Crates mounted on the signal flanges that contain the \glspl{wib}}

%power and timing cards
\newduneabbrev{ptc}{PTC}{power and timing card}{Cards that provide further processing and distribution of the signals entering and exiting the \gls{sp} cryostat}

\newduneabbrev{ptb}{PTB}{power and timing backplane}{Backplane used to connect the \glspl{wib} and the \glspl{ptc} on the \gls{wiec}. Also connects the \gls{ce} flange on the cryostat penetration}

%silicon photomultipler
\newduneabbrev{sipm}{SiPM}{silicon photomultiplier}{A solid-state
  avalanche photodiode sensitive to single \phel signals}

%cryogenic instrumentation and slow control
\newduneabbrev{cisc}{CISC}{cryogenic instrumentation and slow controls}{Includes equipment to monitor all detector  components and  \gls{lar} quality and behavior, and provides a control system for many of the detector components}

%FTE
\newduneword{fte}{FTE}{full-time equivalent. A unit of labor
  for the project. One year of work from one person}



%art 
\newduneword{art}{art}{A software framework implementing an
  event-based execution paradigm} %http://art.fnal.gov/

%sequential access via metadata  
\newduneabbrev{sam}{SAM}{sequential
  access via metadata}{A data-handling system to store and retrieve
  files and associated metadata, including a complete record of the
  processing that has used the files}

%art data aquisition
\newduneword{artdaq}{artdaq}{A data acquisition toolkit for data transfer, aggregation and processing}

%beamline
\newduneword{beamline}{beamline}{A sequence of control and monitoring devices used for the formation of a directed collection of particles; a subproject within \gls{lbnf}} % #36


\newduneword{cdr}{CDR}{Depending on context, either ``conceptual design report,'' a formal project  document  that describes the experiment at a conceptual level, or ``conceptual design review,'' a formal review of the conceptual design of the experiment or of a component}  % #33


%conventional facilities
\newduneabbrev{cf}{CF}{conventional facilities}{Pertaining to
  construction and operation of buildings and conventional infrastructure, and for the \gls{lbnf-dune}, CF includes the excavation caverns}

%charge parity
\newduneabbrev{cp}{CP}{charge conjugation and parity}{Product of charge conjugation and parity
  transformations} % updated to include `conjugation' 12/14/20 per Steve Manly

%product of charge, parity and time-reversal
\newduneabbrev{cpt}{CPT}{charge, parity, and time reversal symmetry}{product of charge, parity
  and time-reversal transformations}

%charge-parity symmetry violation
\newduneabbrev{cpv}{CPV}{charge-parity symmetry violation}{Lack of
  symmetry in a system before and after charge and parity
  transformations are applied. 
  For CP symmetry to hold,  a particle turns into its
 corresponding antiparticle under a charge transformation, and a parity
transformation inverts its space coordinates, i.e. produces the mirror image}

%us department of energy
\newduneword{doe}{DOE}{U.S. Department of Energy}

\newduneabbrev{fra}{FRA}{Fermi Research Alliance}{A joint partnership of the University of Chicago and the Universities Research Association (URA) that manages and operates Fermilab on behalf of the \gls{doe}}

%\newduneword{us}{USA}{United States of America}

%deep underground neutrino experiment
\newduneabbrev{dune}{DUNE}{Deep Underground Neutrino Experiment}{A leading-edge, international experiment for neutrino science and proton decay studies}

%environment, safety and health
\newduneabbrev{esh}{ES\&H}{environment, safety and health}{A discipline and specialty that studies and implements practical aspects of environmental protection and safety at work} % The LBNF/DUNE ES\&H program complies with applicable standards and local, state, and federal legal requirements through the Fermilab ``work smart'' set of standards and the contract between Fermi Research Alliance and the DOE Office of Science (FRA-DOE)}

\newduneabbrev{ppe}{PPE}{personnel protective equipment}{Equipment worn to minimize exposure to hazards that cause serious workplace injuries and illnesses}

\newduneabbrev{odh}{ODH}{oxygen deficiency hazard}{a hazard that occurs when inert gases such as nitrogen, helium, or argon displace room air and thus reduce the percentage of oxygen below the level required for human life}

\newduneabbrev{feshm}{FESHM}{Fermilab Environment, Safety and Health Manual}{The document that contains Fermilab's policies and procedures designed to manage environment, safety, and health in all its programs}


%far site conventional facilities
\newduneabbrev{fscf}{FSCF}{far site conventional facilities}{The
  \gls{cf} at the DUNE far detector site, \gls{surf}}
  
%near site conventional facilities
\newduneabbrev{nscf}{NSCF}{near site conventional facilities}{The
  \gls{cf} at the DUNE near detector site, \gls{fnal}}

%grand unified theory
\newduneabbrevs{gut}{GUT}{grand unified theory}{grand unified theories}{A class of theories that unifies the electroweak and strong forces}

%liquid argon
\newduneabbrev{lar}{LAr}{liquid argon}{Argon in its liquid phase; it is a cryogenic liquid with a boiling point of \SI{87}{K} and density of \SI{1.4}{g/ml}}


%long-baseline
\newduneabbrev{lbl}{LBL}{long-baseline}{Refers to the distance between the 
  neutrino source  and the \gls{fd}.  It can also refer to the distance between the near and far detectors. 
  The ``long'' designation is an approximate and relative distinction. For DUNE, this distance  (between \gls{fnal} and \gls{surf}) is approximately \SI{1300}{km}}

%long-baseline neutrino facility
\newduneabbrev{lbnf}{LBNF}{Long-Baseline Neutrino Facility}{The project scope and  
  organizational entity responsible for developing the neutrino beam, the cryostats
  and cryogenics systems, and the conventional facilities for DUNE} % #36
  
\newduneabbrev{lbnf-dune}{LBNF/DUNE}{LBNF and DUNE enterprise}{The overall enterprise including the LBNF/DUNE-US \gls{doe} project, other international contributing projects, and the \gls{dune} collaboration and experiment}
%\newduneabbrev{lbnf-dune}{LBNF/DUNE}{LBNF and DUNE project}{The overall global project, including \gls{lbnf} and \gls{dune}} % #35 09nov20

\newduneabbrev{lbnc}{LBNC}{Long-Baseline Neutrino Committee}{The committee, composed of internationally prominent scientists with relevant expertise, charged by the \gls{fnal} director to review the scientific, technical, and managerial progress, plans and decisions associated with \gls{dune}}

\newduneabbrev{ncg}{NCG}{Neutrino Cost Group}{A group of internationally prominent scientists with relevant experience that is charged by the \gls{fnal} director to review the cost, schedule, and associated risks for the \gls{dune} experiment}

%mass hierarchy
\newduneabbrev{mh}{MH}{mass hierarchy}{Describes the separation
  between the mass squared differences related to the solar and
  atmospheric neutrino problems (also written as \gls{mo})}

\newduneabbrev{mo}{MO}{mass ordering}{See \gls{mh}}

%fnal main injector
\newduneabbrev{mi}{MI}{Fermilab Main Injector}{An accelerator at
  \gls{fnal} that provides a beam of high-energy protons that upon
  striking a target produce secondaries that decay to provide the
  neutrinos directed toward the DUNE far detector}

%protons on target
\newduneabbrev{pot}{POT}{protons on target}{Typically used as a unit
  of normalization for the number of protons striking the neutrino
  production target}

%quality assurance
\newduneabbrev{qa}{QA}{quality assurance}{The process of ensuring that %The set of actions taken to provide confidence that 
the quality of each element meets requirements during design and development, and to detect and correct poor results prior to production} % #33

%quality control
\newduneabbrev{qc}{QC}{quality control}{The process (e.g., inspection, testing, measurements) %An aggregate of activities (such as design analysis and inspection for defects) performed 
of ensuring that each manufactured element meets its quality requirements prior to assembly or installation} %adequate quality in manufactured products}  % #33

%standard model
\newduneabbrev{sm}{SM}{Standard Model}{Refers to a theory describing
  the interaction of elementary particles}

%technical design report
\newduneword{tdr}{TDR}{Depending on context, either ``technical design report,'' a formal project  document  that describes the experiment at a technical level, or ``technical design review,'' a formal review of the technical design of the experiment or of a component}  % #33

%  #33 \newduneabbrev{prelimdr}{PDR}{preliminary design report}{A formal project document  that describes the experiment at a preliminary design level}

%interim design report
\newduneabbrev{tp}{IDR}{interim design report}{An intermediate
milestone on the path to a full \gls{tdr}} % changed from ``technical proposal'' 6/6/2018

%%%%%%%%%%%%% PROJECT AND PHYSICS VOLUME list for acronyms below %%%%%%%%%%%%
\newduneabbrev{ckm}{CKM matrix}{Cabibbo-Kobayashi-Maskawa
  matrix}{Refers to the matrix describing the mixing between mass and
  weak eigenstates of quarks}

\newduneabbrev{cl}{CL}{confidence level}{Refers to a probability
  used to determine the value of a random variable given its
  distribution}

\newduneabbrev{pmns}{PMNS}{Pontecorvo-Maki-Nakagawa-Sakata}{A type of matrix that describes the mixing between mass and weak eigenstates of
  the neutrino}


%%%%%%%%%%%%%%%%%.....................

%%%%%%%%%%%%% PROJECT AND DETECTORS VOLUME list for acronyms below %%%%%%%%%%%%



\newduneabbrevs{cpa}{CPA}{cathode plane assembly}{cathode plane assemblies}{The component of the \gls{sp} detector module that provides the drift HV cathode}

\newduneabbrev{fc}{FC}{field cage}{The component of a \gls{lartpc} that contains and shapes the applied \efield}

\newduneword{cpafc}{CPA/FC}{A pair of \gls{cpa} panels and the top and bottom \gls{fc} portions that attach to the pair; an intermediate assembly for installation into the \gls{spmod} }

\newduneabbrev{topfc}{top FC}{top field cage}{The horizontal portions of the \gls{sphd} \gls{fc}   on the top of the \gls{tpc}}

\newduneabbrev{botfc}{bottom FC}{bottom field cage}{The horizontal portions of the \gls{sphd} \gls{fc} on the bottom of the \gls{tpc}}

\newduneabbrev{ewfc}{endwall FC}{endwall field cage}{The vertical portions of the \gls{fc} near the end walls}

\newduneabbrev{gp}{GP}{ground plane}{An electrode held electrically neutral relative to Earth ground voltage; it is mounted on the \gls{fc} in a \gls{spmod} to protect the cryostat wall}

\newduneword{gg}{ground grid}{An electrode held electrically neutral relative to Earth ground voltage; it is installed between the cathode and the \glspl{pd} in a \gls{dpmod} to protect the \glspl{pmt}, maintaining high transparency to light}


\newduneabbrev{alara}{ALARA}{as low as reasonably
  achievable}{Typically used with regard management of radiation
  exposure but may be used more generally. It means making every
  reasonable effort to maintain e.g., exposures, to as far below the
  limits as practical, consistent with the purpose for that the
  activity is undertaken}

\newduneabbrev{ecal}{ECAL}{electromagnetic calorimeter}{A detector
  component that measures energy deposition of traversing particles (in the near detector conceptual design)}

\newduneabbrev{hv}{HV}{high voltage}{Generally describes a voltage
  applied to drive the motion of free electrons through some media, e.g., LAr}

% can also use in the text: \gls{sp} \gls{detmodule} 
\newduneword{spmod}{SP module}{single-phase DUNE \gls{fd} module}
\newduneword{dpmod}{DP module}{dual-phase DUNE \gls{fd} module}
\newduneword{dsp}{DUNE-SP}{a single-phase DUNE far detector module} % 33
\newduneword{ddp}{DUNE-DP}{a dual-phase DUNE far detector module}% 33

\newduneabbrev{tcoord}{TC}{technical coordinator}{A member of the \gls{dune} management team responsible for organizing the technical aspects of the project effort; is head of \gls{tc}}

%\newduneabbrev{rcoord}{RC}{resource coordinator}{A member of the \gls{dune} management team responsible for coordinating the financial resources of the project effort} % #33

\newduneabbrev{tc}{TCN}{technical coordination}{The DUNE organization responsible for overall integration of the detector elements and successful execution of the detector construction project; areas of responsibility include general project oversight, systems engineering, \gls{qa} and safety}  % #33


\newduneabbrev{exb}{EB}{executive board}{The highest level DUNE
  decision-making body for the collaboration}

\newduneabbrev{tb}{TB}{technical board}{The DUNE organization responsible for
  evaluating technical decisions}

\newduneabbrev{rrb}{RRB}{Resources Review Board}{A part of \gls{dune}'s international project governance structure, composed of representatives of all funding agencies that sponsor the project, and of  \gls{fnal} management, established to provide coordination among funding partners and oversight of \gls{dune}}

\newduneabbrev{inc}{INC}{International Neutrino Council}{A highest-level international advisory body to the U.S. \gls{doe} and the  \gls{fnal} directorate on matters related to the  \gls{lbnf} and the  \gls{pip2} projects. This council is composed of representatives from the international funding agencies and  \gls{cern} that make major contributions the infrastructure}


%%%%%%%%%%%%% PHYSICS AND DETECTORS VOLUME list for acronyms below %%%%%%%%%%%%

\newduneabbrev{cc}{CC}{charged current}{Refers to an interaction
  between elementary particles where a charged weak force carrier
  ($W^+$ or $W^-$) is exchanged}


\newduneabbrev{dis}{DIS}{deep inelastic scattering}{Refers to interaction between
  elementary particles and a nucleus in an energy range where the
  interaction can be modeled as occurring between constituent quarks
  of one nucleon and resulting in no bulk recoil of the resulting
  nucleus} % 1sep2020 ArturA: swapped with {qe}

\newduneabbrev{fsi}{FSI}{final-state interactions}{Refers to
  interactions between elementary or composite particles subsequent to
  the initial, fundamental particle interaction, such as may occur as
  the products exit a nucleus}
  
\newduneabbrev{fsint}{FSI}{far site integration}{The scope of work at the \gls{fs} for the \gls{integoff}} % #36 - can't put it together with prev item

\newduneword{geant4}{Geant4}{A
  software toolkit for the simulation of the passage of particles
  through matter using \gls{mc} methods}

\newduneabbrev{genie}{GENIE}{Generates Events for Neutrino Interaction
  Experiments}{Software providing an object-oriented neutrino
  interaction simulation resulting in kinematics of the products of
  the interaction}

\newduneabbrev{mc}{MC}{Monte Carlo}{Refers to a method of numerical
  integration that entails the statistical sampling of the integrand
  function. 
  Forms the basis for some types of detector and physics simulations}

\newduneabbrev{qe}{QE}{quasi-elastic}{Refers to the 
  interaction of an elementary charged particle with a nucleus in an
  energy range where the interaction can be modeled as taking place with
  individual nucleons} % 1sep2020 ArturA: swapped with {dis}

%%%%%%%%%%%%%%%%%%%%%%%%% PROJECT VOLUME list for acronyms below %%%%%%%%%%%%%%%

\newduneabbrev{mou}{MoU}{memorandum of understanding}{A project management methodology that documents an agreement between \gls{fnal} and the LBNF/DUNE-US Project for how Fermilab will support the project. More generally, a document
  summarizing an agreement between two or more parties} % #36 elaine

\newduneabbrev{pip2}{PIP-II}{Proton Improvement Plan II}{A \gls{fnal} project for
  improving the protons on target delivered delivered by the \gls{lbnf} neutrino production beam. 
  This is version two of this plan and it is planned to be followed by a PIP-III}
  
\newduneabbrev{sdsta}{SDSTA}{South Dakota Science and Technology
  Authority}{The legal entity that manages \gls{surf}, in Lead, S.D}
  
\newduneabbrev{sdsd}{SDSD}{Fermilab South Dakota Services Division}{A Fermilab division responsible providing host laboratory functions at SURF in South Dakota}

\newduneabbrev{firus}{FIRUS}{Facility Information Reporting Utility System}
 {Facility incident reporting systems, one at \gls{fnal} and at \gls{surf}, that monitors and reports the status of various fire, security and utility sensors} % #36

\newduneabbrev{bsi}{BSI}{building and site infrastructure}
 {The work package for outfitting of the \gls{lbnf} underground infrastructure}



\newduneabbrev{wbs}{WBS}{work breakdown structure}{An organizational
  project management tool by which the tasks to be performed are
  partitioned in a hierarchical manner}

%%%%%%%%%%%%%%%%%%%%%%%%% PHYSICS VOLUME list for acronyms below %%%%%%%%%%%%%%%
\newduneabbrev{br}{BR}{branching ratio}{A fractional probability for a
  decay of a composite particle to occur into some specified set or
  sets of products}
\newduneword{bsm}{BSM}{beyond the Standard Model}

\newduneabbrev{dm}{DM}{dark matter}{The term given to the unknown
  matter or force that explains measurements of galaxy motion % motion of galaxies
  that are otherwise inconsistent with the amount of mass associated
  with the observed amount of photon production}
  
\newduneabbrev{bdm}{BDM}{boosted dark matter}{A new model that describes a relativistic dark matter particle boosted by the annihilation of heavier dark matter particles in the galactic center or the sun}

\newduneabbrev{cern}{CERN}{European Laboratory for Particle Physics}{The leading particle physics laboratory in Europe and home to the ProtoDUNEs} % per Steve Manly/Hiro Tanaka Dec 2020 (In French, the Organisation Europ\'{e}enne pour la Recherche Nucl\'{e}aire, derived from Conseil Europ\'{e}en pour la Recherche Nucl\'{e}aire)}


\newduneabbrev{dsnb}{DSNB}{diffuse supernova neutrino background}{The
  term describing the pervasive, constant flux of neutrinos due to all
  past supernova neutrino bursts}

\newduneabbrev{espp}{ESPP}{European Strategy for Particle Physics}{The
%European Strategy for Particle Physics is the 
cornerstone of Europe's
decision-making process for the long-term future of the
field. Mandated by the \gls{cern} Council, it is formed through a broad
consultation of the grass-roots particle physics community, it
actively solicits the opinions of physicists from around the world,
and it is developed in close coordination with similar processes in
the USA and Japan in order to ensure coordination between regions and
optimal use of resources globally}

\newduneabbrev{gar}{GAr}{gaseous argon}{argon in its gas phase}
\newduneabbrev{gartpc}{GArTPC}{gaseous argon time-projection chamber}{A \gls{tpc} filled with gaseous argon} %; a possible technology choice for the \gls{nd}} AH rmvd 12/2020



\newduneabbrev{globes}{GLoBES}{General Long-Baseline Experiment
  Simulator}{A software package for simulating energy spectra of
  neutrino flux, interactions, and energy spectra measured after application of some
  model of a detector response)}

\newduneabbrev{snowglobes}{SNOwGLoBES}{SuperNova
Observatories with GLoBES} {From the official description: SNOwGLoBES is public software for computing interaction rates and distributions of observed quantities for \gls{snb} neutrinos in common detector materials} % (Anne thinks this was too long.)
% The intent is to provide a very simple and fast code and data package which can be used for tests of observability of physics signatures in current and future detectors, and for evaluation of relative sensitivities of different detector configurations. The event estimates are made using available cross-sections and parameterized detector responses. Water, argon, scintillator and lead-based configurations are included. The package makes use of GLoBES front-end software. SNOwGLoBES is not intended to replace full detector simulations; however output should be useful for many types of studies}


% are these really used anywhere?
\newduneword{l/e}{L/E}{length-to-energy ratio}
\newduneword{lri}{LRI}{long-range interactions}
%\newduneword{solarmass}{$M_{\odot}$}{solar mass}

\newduneabbrev{nc}{NC}{neutral current}{Refers to an interaction
  between elementary particles where a neutrally charged weak force carrier
  ($Z^0$) is exchanged}

\newduneabbrev{nh}{NH}{normal hierarchy}{Refers to the neutrino mass
  eigenstate ordering whereby the sign of the mass squared difference
  associated with the atmospheric neutrino problem is positive}

\newduneabbrev{ih}{IH}{inverted hierarchy}{Refers to the neutrino mass
  eigenstate ordering whereby the sign of the mass squared difference
  associated with the atmospheric neutrino problem is negative}

\newduneabbrev{no}{NO}{normal ordering}{Refers to the neutrino mass
  eigenstate ordering whereby the sign of the mass squared difference
  associated with the atmospheric neutrino problem is positive}

\newduneabbrev{io}{IO}{inverted ordering}{Refers to the neutrino mass
  eigenstate ordering whereby the sign of the mass squared difference
  associated with the atmospheric neutrino problem is negative}


\newduneabbrev{msw}{MSW}{Mikheyev-Smirnov-Wolfenstein effect}{Explains
  the oscillatory behavior of neutrinos produced inside the sun as
  they traverse the solar matter}

\newduneabbrev{nsi}{NSI}{nonstandard interaction}{A general class of
  theory of elementary particles other than the Standard Model}



\newduneabbrev{pfive}{P5}{Particle Physics Project Prioritization
Panel}{The Particle Physics Project Prioritization Panel (P5) was a
subpanel of the High Energy Physics Advisory Panel (HEPAP). It completed
its Report, a ten-year strategic plan for high energy physics in the
U.S., in 2014. This report included a recommendation that ``host a world-leading neutrino
program that will have an optimized set of short- and long-baseline neutrino oscillation experiments, and its long-term focus
is a reformulated venture referred to here as the Long Baseline
Neutrino Facility (LBNF)''}

\newduneabbrev{sme}{SME}{standard-model extension}{an effective field theory that contains the \gls{sm}, general relativity, and all possible operators that break Lorentz symmetry (Wikipedia)}

\newduneabbrev{susy}{SUSY}{supersymmetry}{Theoretical symmetry between a fermion and a boson}

\newduneabbrev{wimp}{WIMP}{weakly-interacting massive particle}{A
  hypothesized particle that may be a component of dark matter}

%%%%%%%%%%%%%%%%%%%%%%%%% DETECTORS VOLUME list for acronyms below %%%%%%%%%%%%%%%

\newduneabbrev{ce}{CE}{cold electronics}{Analog and digital readout electronics that operate at cryogenic temperatures}

\newduneabbrev{crp}{CRP}{charge-readout plane}{In the \gls{dp} technology, a  collection of
  electrodes in a planar arrangement placed at a particular voltage
  relative to some applied \efield such that drifting electrons
  may be collected and their number and time may be measured}

\newduneabbrev{dram}{DRAM}{dynamic random access memory}{A computer memory technology}


\newduneabbrev{fnal}{Fermilab}
{Fermi National Accelerator Laboratory}{U.S. national laboratory in Batavia, IL. It is the laboratory that hosts \gls{dune} and serves as its near site}

\newduneabbrev{bnl}{BNL}{Brookhaven National Laboratory}{US national laboratory in Upton, NY}

\newduneabbrev{slac}{SLAC}{SLAC National Accelerator Laboratory}{US national laboratory in Menlo Park, CA}

\newduneabbrev{lbnl}{LBNL}{Lawrence Berkeley National Laboratory}{US national laboratory in Berkeley, CA}

\newduneabbrev{anl}{ANL}{Argonne National Laboratory}{US national laboratory in Lemont, IL}

\newduneabbrev{lanl}{LANL}{Los Alamos National Laboratory}{US national laboratory in Los Alamos, NM}

%\newduneabbrev{fs}{FS}{full stream}{Relates to a data stream that has not undergone selection, compression or other form of reduction}
\newduneword{fs}{FS}{(1) The far site, \gls{surf}, where the DUNE far detector is located; (2) ``Full stream'' relates to a data stream that has not undergone selection, compression or other form of reduction} % #36 (see change from item above)


\newduneabbrev{lem}{LEM}{large electron multiplier}{A micro-pattern detector suitable for use in ultra-pure argon vapor; LEMs consist of copper-clad PCB boards with sub-millimeter-size holes through which electrons undergo amplification}


\newduneabbrev{lng}{LNG}{liquefied natural gas}{Pertaining to natural gas in its liquid phase}



\newduneabbrev{mip}{MIP}{minimum ionizing particle}{Refers to a
  particle traversing some medium such that the particle's mean energy loss is  
  near the minimum}
%{Refers to a  momentum traversing some medium such that the particle is losing  near the minimum amount of energy per distance traversed} % some \mip and some \gls{mip}. If time, rectify. ??


\newduneabbrev{pd}{PD}{photon detector}{The detector
  elements involved in measurement of the number and arrival times of
  optical photons produced in a detector module} 

\newduneabbrev{pmt}{PMT}{photomultiplier tube}{A device that makes use
  of the photoelectric effect to produce an electrical signal from the
  arrival of optical photons}

\newduneabbrev{ppm}{ppm}{parts per million}{A concentration equal to one part in $10^{-6}$}
\newduneabbrev{ppb}{ppb}{parts per billion}{A concentration equal to one part in $10^{-9}$}
\newduneabbrev{ppt}{ppt}{parts per trillion}{A concentration equal to one part in $10^{-12}$}

% these should be abbrev
\newduneword{rio}{RIO}{reconfigurable input output}

%signal over noise ratio -- merge these two (tpcelec sp uses 2nd)

\newduneabbrev{s/n}{S/N}{signal-to-noise}{signal-to-noise ratio}
\newduneword{snr}{\mbox{S/N}}{signal-to-noise ratio}


\newduneword{ssp}{SSP}{\gls{sipm} signal processor}

\newduneabbrev{sbn}{SBN}{Short-Baseline Neutrino}{A \gls{fnal} program consisting of three collaborations, \gls{microboone}, \gls{sbnd}, and \gls{icarus}, to perform sensitive searches for $\nue$ appearance and $\numu$ disappearance in the Booster Neutrino Beam}

\newduneword{stt}{STT}{straw tube tracker}


\newduneword{wire board}{wire board}{At the head end of the APA in the \gls{sphd} \gls{tpc}, stacks of electronics boards referred to as ``wire boards'' are arrayed to anchor the wires.  They also provide the connection between the wires and the cold electronics} %?? long for a word. ??

\newduneabbrev{wls}{WLS}{wavelength-shifting}{A material or process by
  which incident photons are absorbed by a material and photons are
  emitted at a different, typically longer, wavelength}
  
\newduneabbrev{tpb}{TPB}{tetra-phenyl butadiene}{A \gls{wls} material}

%\newduneabbrev{ptp}{PTP}{p-terphenyl}{A \gls{wls} material}  % #33

\newduneabbrev{sft}{SFT}{signal feedthrough}{A cryostat penetration allowing for the passage of cables or other extended parts}

\newduneabbrev{sftchimney}{SFT chimney}{signal feedthrough chimney}{In the \gls{dp} technology, a volume above the cryostat penetration used for a signal feedthrough}


\newduneabbrev{catiroc}{CATIROC}{charge and time integrated readout chip}{A complete read-out chip manufactured in AustriaMicroSystem designed to read arrays of 16 photomultipliers}

\newduneabbrev{wr}{WR}{White Rabbit}{A component of the timing system that forwards clock signal and time-of-day reference data to the master timing unit}

\newduneabbrev{mch}{MCH}{MicroTCA Carrier Hub}{An network switching device}

\newduneabbrev{wrmch}{WR-MCH}{White Rabbit \gls{utca} Carrier Hub}{A card mounted in \gls{utca} crate that recieves time syncronization information and trigger data packets over \gls{wr} network and disributes them to the \gls{amc} over \gls{utca} backplane} 

\newduneabbrev{wrtsn}{WR-TSN}{White Rabbit TimeStamping Node}{A unit on the \gls{wr} network that timestamps the trigger signals and sends out trigger data packets to \gls{wrmch}}

% these should be abbrevs
%\newduneword{cmp}{CMP}{configuration management plan}  #36 replace with sys eng mgmt plan

\newduneword{qap}{QAP}{quality assurance plan} %{A project management device for planning \gls{qa}}
\newduneword{ieshp}{IESHP}{integrated environmental, safety and health plan}%{Refers to the LBNF/DUNE project planning instrument}
\newduneword{dmp}{DMP}{data management plan} %{A project management device to state how the experimental data will be managed}
\newduneword{qam}{QAM}{quality assurance manager} %{The manager of \gls{qa} for the LBNF/DUNE project}

\newduneabbrev{dss}{DSS}{detector support system}{The system of rails suspended from the cryostat ceiling in a \gls{spmod} used to support the \gls{apa}s, \gls{cpa}s, and the \glspl{ewfc}} % update from MV 1sept2020 

\newduneabbrev{ddss}{DDSS}{DUNE detector safety system}{The hardware system responsible for the safety of the detector, implemented either via a \gls{plc} or via custom hardware protections} % update from MV 1sept2020 

%\newduneabbrev{lc}{LC}{logistics center}{A facility where \gls{lbnf} and \gls{dune} components will be received and transhipped to \gls{surf}} % 33 (it's the SDWF not LC)

\newduneabbrev{tco}{TCO}{temporary construction opening}{An opening in the side of a cryostat through which detector elements are brought into the cryostat; utilized during construction and installation}

\newduneabbrev{surf}{SURF}{Sanford Underground Research Facility}{The laboratory in South Dakota where the \gls{dune} \gls{fd} will be installed and operated; also where the \gls{lbnf} \gls{fscf} and the \gls{fs} cryostat and cryogenic systems will be constructed} %  36

%\newduneabbrev{sit}{SIT}{surface installation team}{An organizational unit responsible for logistics and integration in South Dakota} #35 09nov20

\newduneabbrev{uit}{UIT}{underground installation team}{An organizational unit responsible for installation in the underground area at the \gls{surf} site}

\newduneabbrev{cmgc}{CMGC}{construction manager/general contractor}{The contracted company hired to manage overall construction, used by \gls{lbnf} at the \gls{surf} site for the \gls{fscf} construction} % #36 old: The organizational unit responsible for management of the construction of conventional facilities at the underground area at the \gls{surf} site}

%\newduneabbrev{cdrev}{CDR}{conceptual design review}{A project management device by which a conceptual design is reviewed}   % #33

\newduneword{pdr}{PDR}{Depending on context, either ``preliminary design report,'' a formal project document  that describes the experiment at a preliminary level, or ``preliminary design review,'' a formal review of the preliminary design of the experiment or of a component} % #33

\newduneword{fdr}{FDR}{Depending on context, either ``final design report,'' a formal project document  that describes the experiment at a final level, or ``final design review,'' a formal review of the final design of the experiment or of a component} %  per #33, not sure if we need a `final design report' - yes per #36

\newduneabbrev{prr}{PRR}{production readiness review}{A project management device by which the production readiness is reviewed}  % #33

\newduneabbrev{irr}{IRR}{installation readiness review}{A project management device by which the plan for installation is reviewed}  % #33
\newduneabbrev{orr}{ORR}{operational readiness review}{A project management device by which the operational readiness is reviewed}  % #33

\newduneabbrev{ppr}{PPR}{production progress review}{A project management device by which the progress of production is reviewed}  % #33

\newduneabbrev{edms}{EDMS}{engineering document management system}{A computerized document management system developed and supported at \gls{cern} in which some LBNF/DUNE documents, drawings and engineering models are managed} % #36  add `lbnf' before dune
%\newduneabbrev{ecr}{ECR}{engineering change request}{The first step in the change control process in which a proposed change is described} #36 replace with bcr
\newduneabbrev{docdb}{DocDB}{Document DataBase}{A computerized document management system developed and supported at \gls{fnal} in which virtually all LBNF and most DUNE documents are managed (docs.dunescience.org)}

\newduneword{wrgm}{WR grandmaster}{White Rabbit grandmaster}


%%%%% Software and computing %%%%

\newduneabbrev{larsoft}{LArSoft}{Liquid Argon Software}{A shared base of physics software across \gls{lartpc} experiments}
% these should be abbrevs
\newduneword{nova}{NOvA}{The \gls{nova} off-axis neutrino oscillation experiment at \gls{fnal}}
\newduneword{minerva}{MINERvA}{Neutrino cross sections experiment at \gls{fnal}}
\newduneword{microboone}{MicroBooNE}{A \gls{lartpc} neutrino oscillation experiment at \gls{fnal}}
\newduneword{sbnd}{SBND}{The Short-Baseline Near Detector experiment at  \gls{fnal}}
\newduneabbrev{nexo}{nEXO}{Enriched Xenon Observatory}{Experiment at Lawrence Livermore National Laboratory (U.S. national lab in Livermore, CA) searching for new physics with neutrinoless double-beta decay}
\newduneword{argoneut}{ArgoNeuT}{The ArgoNeuT test-beam experiment and \gls{lartpc} prototype at  \gls{fnal}}
\newduneword{icarus}{ICARUS}{A neutrino experiment that was located at the Laboratori Nazionali del Gran Sasso (LNGS) in Italy, then refurbished at \gls{cern} for re-use in the same neutrino beam from \gls{fnal} used by the \gls{miniboone} , \gls{microboone} and \gls{sbnd} experiments. The ICARUS detector is being reassembled at \gls{fnal}}
\newduneword{atlas}{ATLAS}{One of two general-purpose detectors at the \gls{lhc}. It investigates a wide range of physics, from the measurements of the Higgs boson properties to searches for extra dimensions and particles that could make up \gls{dm}}

\newduneword{lbne}{LBNE}{Long Baseline Neutrino Experiment; (1) a terminated U.S. experiment that was reformulated in 2014 under the auspices of the new \gls{dune} collaboration, an internationally coordinated and internationally funded program, with \gls{fnal} as host; and (2) the former name of the DOE LBNF/DUNE Project} % #36


\newduneabbrev{lbno}{LBNO}{Long Baseline Neutrino Observatory} {A terminated European project that, during its six-year duration, assessed the feasibility of a next-generation deep underground neutrino observatory in Europe)}
%\newduneabbrev{lbno}{LBNO}{Long Baseline Neutrino Observatory}{During its six-year duration, its members assessed the feasibility of a next-generation deep underground neutrino observatory in Europe}


\newduneword{wirecell}{Wire-Cell}{A tomographic automated \threed neutrino event reconstruction method for \glspl{lartpc}}
\newduneabbrev{wct}{WCT}{Wire-Cell Toolkit}{A software toolkit with data flow processing components for \gls{lartpc} noise and signal simulation, noise filtering, signal processing, and tomographic \threed ionization activity imaging}
\newduneword{ftslite}{F-FTS-lite}{Light-weight version of the \gls{fnal} File Transfer system used for rapid data transfers out of the online systems}
\newduneabbrev{fts}{FTS}{File Transfer System}{A file transfer system developed at \gls{fnal} to catalog and move data to permanent storage}

%%% new ones that I haven't categorized (Anne)
\newduneword{35t}{35 ton prototype}{A prototype cryostat and \gls{sp} detector built at \gls{fnal} before the \gls{protodune} detectors}

\newduneabbrev{mcr}{MCR}{main communications room}{Space at the \gls{fd} site for cyber infrastructure}

\newduneabbrev{cuc}{CUC}{central utility cavern}{The utility cavern at the 4850L of \gls{surf} located between the two detector caverns. It contains utilities such as central cryogenics and other systems, and the underground data center and control room}

\newduneabbrev{cfd}{CFD}{computational fluid dynamics}{High performance computer-assisted modeling of fluid dynamical systems}
\newduneword{vuv}{VUV}{vacuum ultra-violet}
\newduneword{tallbo}{TallBo}{A cylindrical cryostat at \gls{fnal} primarily used for developing scintillation light collection technologies for \gls{lartpc} detectors}

\newduneword{root}{ROOT}{A modular scientific software toolkit. It provides all the functionalities needed to deal with big data processing, statistical analysis, visualisation and storage. It is mainly written in C++ but integrated with other languages such as Python and R}

\newduneabbrev{eos}{EOS}{EOS}{The XRootD-based distributed file system developed by CERN}
\newduneabbrev{ehn1}{EHN1}{Experiment Hall North One}{Location at CERN of the ProtoDUNE experiments}
\newduneword{led}{LED}{Light-emitting diode}
\newduneabbrev{rtd}{RTD}{resistance temperature detector}{A temperature sensor consisting of a material with an accurate and reproducible resistance/temperature relationship}
\newduneword{swc}{SWC}{Software \& Computing}
\newduneabbrev{las}{LAS}{LEM-anode Sandwich}{In the \gls{dp} technology, a \gls{lem} and its corresponding anode are mounted together in a module called a LEM-anode sandwich}

\newduneword{roi}{ROI}{region of interest}
\newduneabbrev{hpc}{HPC}{high-performance computing}{high-performance computing facilities; generally computing facilities emphasizing parallel computing with aggregate power of more than a teraflop}


\newduneword{comfund}{common fund}{The shared resources of the collaboration}
\newduneabbrev{ims}{IMS}{integrated master schedule}{A project management device consisting of linked tasks and milestones}

\newduneword{hvdb}{HVDB}{HV divider board}

\newduneword{hvft}{HVFT}{HV feedthrough}  % #33

\newduneword{sas}{SAS}{Another term for the materials airlock; a pass-through chamber used to ensure safe transfer of materials into a clean room, avoiding contamination in both directions}

\newduneabbrev{fea}{FEA}{finite element analysis}{Simulation of a physical phenomenon using the numerical technique called Finite Element Method (FEM), a numerical method for solving problems of engineering and mathematical physics}

\newduneword{fss}{FSS}{field shaping strips}
\newduneword{lvds}{LVDS}{low-voltage differential signaling}



%electrostatic discharge  
\newduneword{esd}{ESD}{electrostatic discharge}%{ESD is the sudden flow of electricity between two electrically charged objects caused by contact, an electrical short, or dielectric breakdown. ESD can cause failure of electronic components such as integrated circuits}

\newduneabbrev{rp}{RP}{resistive panel}{Resistive panels form the constant potential surfaces for a \gls{spmod} \gls{cpa}; they are composed of a thin layer of carbon-impregnated Kapton and laminated to both sides of a \frfour sheet}

\newduneword{uhmwpe}{UHMWPE}{ultra-high molecular weight polyethylene}

\newduneword{cts}{CTS}{Cryogenic Test System}

\newduneabbrev{plc}{PLC}{programmable logic controller}{An industrial digital computer that has been ruggedized and adapted for the control of manufacturing or other processes that require high reliability, ease of programming, and process fault diagnosis} % update from Marco V 1sep2020

\newduneword{mppc}{MPPC}{\SI{6}{mm}$\times$\SI{6}{mm} Multi-Pixel Photon Counters produced by Hamamatsu\texttrademark{} Photonics K.K}

\newduneabbrev{sfp}{SFP}{small form-factor pluggable}{a particular standard for optical transceivers}

\newduneabbrev{minipod}{MiniPOD}{miniature parallel optical device}{a family of types of multi-channel optical transceivers}

\newduneword{ccc}{CCC}{configuration change command}
\newduneword{act}{ACT}{activation time stamp}
\newduneword{lcm}{LCM}{light calibration module}
\newduneword{lpm}{LPM}{light pulser module}
\newduneword{dac}{DAC}{digital-to-analog converter}
\newduneword{arapuca}{ARAPUCA}{A \gls{pds} design that consists of a light trap that captures wavelength-shifted photons inside boxes with highly reflective internal surfaces until they are eventually detected by \gls{sipm} detectors or are lost}
\newduneword{sarapu}{S-ARAPUCA}{Standard \gls{arapuca} design with different \gls{wls} coatings on both faces of the dichroic filter window(s) of the cell}
\newduneword{xarapu}{X-ARAPUCA}{Extended \gls{arapuca} design with \gls{wls} coating on only the external face of the dichroic filter window(s) but with a \gls{wls} doped plate inside the cell}
\newduneword{feb}{FEB}{front-end board}

\newduneabbrev{lsnd}{LSND}{Liquid Scintilator Neutrino Detector}{A scintillation detector and associated experiment located at Los Alamos National Laboratory}

\newduneabbrev{cvn}{CVN}{convolutional visual network}{An algorithm for identifying neutrino interactions based on their topology and without the need for detailed reconstruction algorithms}

\newduneword{pandora}{Pandora}{The Pandora multi-algorithm approach to pattern recognition} 

%Lisa added
\newduneabbrev{pma}{PMA}{Projection Matching Algorithm}{A reconstruction algorithm that combines \twod reconstructed objects to form a \threed representation}
\newduneabbrev{bdt}{BDT}{boosted decision tree}{A method of multivariate analysis}
\newduneabbrev{cnn}{CNN}{convolutional neural network}{A deep learning technique most commonly applied to analyzing visual imagery}
\newduneword{pdg}{PDG}{Particle Data Group}

% from CISC
\newduneword{pci}{PCI}{Peripheral Component Interconnect}

\newduneword{labview}{LabVIEW}{Laboratory Virtual Instrument Engineering Workbench is a system-design platform and development environment for a visual programming language from National Instruments}

\newduneword{pcb}{PCB}{printed circuit board}

\newduneword{crio}{cRIO}{Compact Reconfigurable Input Output}

\newduneword{dcs}{DCS}{Distributed Communications System}

\newduneword{opc-ua}{OPC-UA}{OPC  Unified Architecture is a machine to machine communication protocol for industrial automation developed by the OPC Foundation. OPC stands for Object Linking and Embedding for Process Control}

\newduneword{cabangle}{Cabibbo angle}{A quark mixing parameter that governs the coupling of up quarks to strange quarks}
\newduneword{valor}{VALOR}{A neutrino oscillation fitting framework that is used by \gls{t2k}; the name stands for VALencia-Oxford-Rutherford, the original three institutions that developed it}
\newduneword{cafana}{CAFAna}{Common Analysis File Analysis}
\newduneabbrev{pca}{PCA}{principal component analysis}{A statistical procedure that uses an orthogonal transformation to convert a set of observations of possibly correlated variables into a set of values of linearly uncorrelated variables called principal components (Wikipedia)}
\newduneword{numi}{NuMI}{a set of facilities at \gls{fnal}, collectively called ``Neutrinos at the Main Injector.''  The NuMI neutrino beamline target system converts an intense proton beam into a focused neutrino beam}
\newduneword{gibuu}{GiBUU}{Giessen Boltzmann-Uehling-Uhlenback Project; a unified theory and transport framework in the MeV and GeV energy regimes for elementary reactions on nuclei }
\newduneabbrev{rpa}{RPA}{random phase approximation} {an approximation method commonly used for describing the dynamic linear electronic response of electron systems (Wikipedia)}%(from nu-osc 05)}
\newduneword{t2k}{T2K}{T2K (Tokai to Kamioka) is a long-baseline neutrino experiment in Japan studying neutrino oscillations}
\newduneword{mptdet}{MPT detector}{multipurpose tracking detector}

\newduneword{lariat}{LArIAT}{The repurposed ArgoNeuT \gls{lartpc}, modified for use in a charged particle beam, dedicated to the calibration and precise characterization of the output response of these detectors}

\newduneword{captain}{CAPTAIN}{Experimental program sited at \gls{lanl} that is designed to make measurements of scientific importance to \gls{lbl} neutrino physics and physics topics that will be explored by large underground detectors}

\newduneword{dayabay}{Daya Bay}{a neutrino-oscillation experiment in Daya Bay, China, designed to measure the mixing angle $\Theta_{13}$  using antineutrinos produced by the reactors of the Daya Bay and Ling Ao nuclear power plants}

\newduneword{nuwro}{NuWro}{neutrino interaction generator}

\newduneabbrev{neut}{NEUT}{neutrino interaction generator}{A neutrino interaction simulation program library for the studies of atmospheric and accelerator neutrinos}

\newduneword{minos}{MINOS}{A long-baseline neutrino experiment, with a near detector at \gls{fnal} and a far detector in the Soudan mine in Minnesota, designed to observe the phenomena of neutrino oscillations (ended data runs in 2012)}


\newduneabbrev{efig}{EFIG}{Experimental Facilities Interface Group}{The body responsible for the required high-level coordination between the \gls{lbnf} and \gls{dune} projects}
\newduneword{ashriver}{Ash River}{The Ash River, Minnesota, USA \gls{nova} experiment far site, used as an assembly test site for \gls{dune}} 

%\newduneabbrev{ipd}{PI-DIR}{project integration director}{Responsible for integration and installation of \gls{lbnf} and \gls{dune} deliverables in South Dakota. Manages the \gls{integoff}}
\newduneword{ipd}{project integration director}{Responsible for integration and installation of DUNE detector deliverables. Manages the integration project} %#35 09nov20 Responsible for integration and installation of \gls{lbnf} and \gls{dune} deliverables in South Dakota. Manages the \gls{integoff}}

\newduneabbrev{jpo}{JPO}{Joint Project Office}{The framework through which team members from the LBNF project office, \gls{integoff}, and DUNE \gls{tc} work together to provide coherence in project support functions across the global enterprise. 
JPO functions include systems engineering, procurement, \gls{esh}, \gls{qa}, finance, project controls, risk management, compliance, internal review, partner agreement management, document management, and administrative support} % #35 09nov20 global project configuration and integration, installation planning and coordination, scheduling, safety assurance, technical review planning and oversight, development of partner agreements, and financial reporting}

\newduneword{ifbeam}{IFbeam}{Database that stores beamline information indexed by timestamp}

\newduneabbrev{marley}{MARLEY}{Model of Argon Reaction Low Energy Yields}{Developed at UC Davis, MARLEY is the first realistic model of neutrino electron interactions on argon for enegies less than \SI{50}{MeV}. This includes the energy range important for \gls{snb} neutrinos and also solar 8--boron neutrinos}

\newduneabbrev{es}{ES}{elastic scattering}{Events in which a neutrino
elastically scatters off of another particle}


\newduneabbrev{cno}{CNO}{carbon nitrogen oxygen}{The CNO cycle (for carbon-nitrogen-oxygen) is one of the two known sets of fusion reactions by which stars convert hydrogen to helium, the other being the proton-proton chain reaction (pp-chain reaction). In the CNO cycle, four protons fuse, using carbon, nitrogen, and oxygen isotopes as catalysts, to produce one alpha particle, two positrons and two electron neutrinos}

\newduneabbrev{sdwf}{SDWF}{South Dakota Warehouse Facility}{Warehousing operations in South Dakota responsible for receiving LBNF and DUNE goods and coordinating shipments to the access shaft (Ross Shaft) at \gls{surf}}

\newduneabbrev{wms}{WMS}{warehouse management system}{Commercial software package used to track shipments and interface to freight forwarders. This includes a database for shipping}

\newduneabbrev{dcdb}{DCDB}{DUNE construction database}{Database used by DUNE to track the history and testing of all parts of each \gls{detmodule}}

\newduneabbrev{aup}{AUP}{acceptance for use and possession}{Required for beneficial occupancy of the underground areas at \gls{surf} for \gls{lbnf} and \gls{dune}}

\newduneabbrev{bms}{BMS}{building management system}{A system provided by the \gls{cf} to manage the utility (cooling, ventilation, power, etc.) and fire/life safety systems. Separate systems are provided at \gls{surf} and at \gls{fnal}} % #36 Elaine  old: Part of the safety system at \gls{surf} that includes the fire and life safety system}

\newduneabbrev{fls}{FLS}{fire and life safety system}{Fire and life safety; systems designed with \gls{cf} to meet building/safety code compliance for safe facilities at \gls{surf} and at \gls{fnal}} % #36 



\newduneabbrev{sno}{SNO}{Sudbury Neutrino Observatory}{The Sudbury Neutrino Observatory was a detector built 6800 feet under ground, in INCO's Creighton mine near Sudbury, Ontario, Canada. SNO was a heavy-water Cherenkov detector designed to detect neutrinos produced by fusion reactions in the sun}

\newduneword{sk}{Super-Kamiokande}{Experiment sited in the Kamioka-mine, Hida-city, Gifu, Japan that uses a large water Cherenkov detector to study neutrino properties through the observation of solar neutrinos, atmospheric neutrinos and man-made neutrinos}


\newduneabbrev{id}{ID}{inner diameter}{Inner diameter of a tube}

\newduneabbrev{od}{OD}{outer diameter}{Outer diameter of a tube}


\newduneabbrev{rms}{RMS}{root mean square}{The square root of the arithmetic mean of the squares of a set of values, used as a measure of the typical magnitude of a set of numbers, regardless of their sign}

\newduneabbrev{orc}{ORC}{operational readiness clearance}{Final safety approval prior to the start of operation}

\newduneabbrev{gsc}{GSC group}{global safety coordination group}{DUNE group that evaluates applicable codes and standards, including international code equivalency, for the design, assembly, and installation of the \gls{fd}}

\newduneabbrev{ha}{HA}{hazard analysis}{A first step in a process to assess risk; the result of hazard analysis is the identification of the hazards present for a task or process}%different types of hazards}(anne)
\newduneword{har}{HAR}{hazard analysis report}

\newduneabbrev{tap}{TAP}{trip action plan}{A document required for any trip by a worker to the underground area at \gls{surf}, per that site's access control program; it describes the work to be accomplished during the trip} % ask Jim Stewart to check

\newduneword{em}{EM}{emergency management}
\newduneword{ert}{ERT}{emergency response team}

% from DP-PDS --begin
\newduneabbrev{ndk}{NDK}{nucleon decay}{The hypothetical, baryon number violating decay of a proton or a bound neutron into lighter particles}

\newduneabbrev{emi}{EMI}{electromagnetic interference}{Disturbance generated by an external source that affects an electrical circuit by electromagnetic induction, electrostatic coupling, or conduction}

\newduneabbrev{pe}{PE}{photoelectron}{An electron ejected from the surface of a material by the photoelectric effect}

\newduneabbrev{spe}{SPE}{single photoelectron}{A single photoelectron}

\newduneabbrev{fwhm}{FWHM}{full width at half maximum}{Width of a distribution measured between those points at which the distribution is equal to half of its maximum amplitude}

\newduneabbrev{gdml}{GDML}{geometry description markup language}{An application-indepedent, geometry-description format based on XML}

\newduneabbrev{xml}{XML}{extensible markup language}{A markup language that defines a set of rules for encoding documents in a format that is both human-readable and machine-readable}

\newduneabbrev{crt}{CRT}{cosmic-ray tagger}{Detector external to the TPC designed to tag TPC-traversing cosmic ray particles}

\newduneabbrev{sn}{SN}{supernova}{Event that occurs upon the death of certain types of stars}

\newduneabbrev{wg}{WG}{working group}{A group of persons working together to achieve specified goals}

\newduneabbrev{ctsf}{CTSF}{coating, testing and storage facility}{A facility where the the \gls{dp} photon detectors will be coated, tested, and stored}

% from DP-PDS --end


% from Schellman

\newduneword{rucio}{Rucio}{Data management system originally developed
by \gls{atlas} but now open-source and shared across HEP}
\newduneabbrev{doma}{DOMA}{data organization, management, and access}{data organization, management, and access efforts through the HEP Software Foundation}

\newduneabbrev{hsf}{HSC}{HEP Software Foundation Collaboration}{A foundation that facilitates cooperation and common efforts in high energy physics software and computing internationally}

\newduneabbrev{wlcg}{WLCG}{Worldwide LHC Computing Grid}{Worldwide LHC
Computing Grid}
\newduneabbrev{osg}{OSG}{Open Science Grid}{Open Science Grid}
\newduneabbrev{sci}{SCI}{Scientific Computing Infrastructure}{Proposed
extension of the infrastructure component of \gls{wlcg} to other
experiments}
\newduneabbrev{csc}{CSC}{computing and software consortium}{DUNE
computing and software consortium}

\newduneword{dirac}{DIRAC}{Computing workflow management designed for LHCb and now used by many HEP experiments}

% from DP-HV --start
\newduneword{frp}{FRP}{fiber-reinforced plastic}
\newduneabbrev{hdpe}{HDPE}{high-density polyethylene}{A thermoplastic polymer made from petroleum commonly used to make plastic bottles}
\newduneword{hvps}{HVPS}{\gls{hv} power supply}
\newduneword{aisi}{AISI}{American Iron and Steel Institute}
\newduneword{ific}{IFIC}{Instituto de Fisica Corpuscular (in Valencia, Spain)}
\newduneabbrev{rsds}{RSDS}{radioactive source deployment system}{Proposed calibration system based on the deployment of
radioactive sources inside the \gls{dune} cryostat}
\newduneword{2p2h}{2p2h}{two particle, two hole}
\newduneabbrev{duneprism}{DUNE-PRISM}{\gls{dune} Precision Reaction-Independent Spectrum Measurement}{a mobile near detector that can perform measurements over a range of angles off-axis from the neutrino beam direction in order to sample many different neutrino energy distributions}
\newduneword{arcube}{ArgonCube}{The name of the core part of the \gls{dune} \gls{nd}, a \gls{lartpc}}

\newduneabbrev{citf}{CITF}{cryogenic instrumentation test facility}{A facility at \gls{fnal} with small ($<$\SI{1}{ton}) to intermediate ($\sim$\SI{1}{ton}) volumes of instrumented, purified TPC-grade \gls{lar}, used for testing devices intended for use in \gls{dune}}

\newduneabbrev{3dst}{3DST}{3D scintillator tracker}{The core part of the \threed projection scintillator tracker spectrometer in the near detector conceptual design}
\newduneabbrev{3dsts}{3DST-S}{3D scintillator tracker spectrometer}{The \threed projection scintillator tracker spectrometer  in the near detector conceptual design}
\newduneabbrev{mpd}{MPD}{multi-purpose detector}{A component of the near detector conceptual design; it is a magnetized system consisting of a \gls{hpgtpc} and a surrounding \gls{ecal}}
\newduneabbrev{hpg}{HPG}{high-pressure gas}{gas at high pressure to be used in a \gls{hpgtpc}} 
\newduneabbrev{hpgtpc}{HPgTPC}{high-pressure gaseous argon TPC}{A \gls{tpc} filled with gaseous argon; a possible component of the \gls{dune} \gls{nd}}

\newduneword{src}{SRC}{short-range correlated nucleon-nucleon interactions}
\newduneword{larpix}{LArPix}{ \gls{asic} pixelated charge readout for a \gls{tpc} }
\newduneword{arclt}{ArCLight}{a light detector for the \gls{arcube} effort}
\newduneword{fhc}{FHC}{forward horn current ($\numu$ mode)}
\newduneword{rhc}{RHC}{reverse horn current (\numubartonumubar mode)}
\newduneword{mwpc}{MWPC}{multi-wire proportional chamber}
\newduneword{na61}{NA61}{CERN hadron production experiment}
\newduneword{pdnd}{ProtoDUNE-ND}{a prototype \gls{dune} \gls{nd}}
% replaced by abbrev from ND CDR: \newduneword{ccqe}{CCQE}{charged current quasielastic interaction} 
\newduneabbrev{roc}{ROC}{readout chamber}{readout chamber for gaseous argon \gls{tpc}}
\newduneabbrev{iroc}{IROC}{inner readout chamber}{inner (radial) readout chamber for gaseous argon \gls{tpc}}
\newduneabbrev{oroc}{OROC}{outer readout chamber}{outer (radial) readout chamber for gaseous argon \gls{tpc}}

\newduneword{lux}{LUX}{Large Underground Xenon (LUX) dark matter detector at \gls{surf} }

\newduneword{mjdemo}{Majorana Demonstrator}{Experiment sited at \gls{surf} that  seeks to determine whether neutrinos are their own antiparticles}

\newduneword{lz}{LZ}{Experiment sited at \gls{surf} that  seeks to detect faint interactions between galactic dark matter and regular matter}

\newduneword{mu2e}{Mu2e}{An experiment sited at \gls{fnal} that searches for charged-lepton flavor violation and seeks to discover physics beyond the \gls{sm}}

\newduneword{pdsp2}{ProtoDUNE-SP-II}{A second test run in the singe-phase
ProtoDUNE test stand at CERN, acting as a validation of the final
single-phase detector design}  % #33


\newduneword{osha}{OSHA}{Occupational Safety and Health Administration (USA Department of Labor) formed by the Occupational Safety and Health Act of 1970}
\newduneabbrev{pns}{PNS}{pulsed neutron source}{Calibration system based
on neutron capture gamma showers spread out in the whole detector}

\newduneabbrev{fv}{FV}{fiducial volume}{The detector volume within the \gls{tpc} that is selected for physics analysis through cuts on reconstructed event position}

\newduneword{p6}{P6}{framework used to plan and status the resource-loaded schedule of activities associated with the USA contributions to \gls{lbnf} and \gls{dune} }
\newduneabbrev{evms}{EVMS}{earned value management system}{Earned Value Management is a systematic approach to the integration and measurement of cost, schedule, and technical (scope) accomplishments on a project or task. It provides both the government and contractors the ability to examine detailed schedule information, critical program and technical milestones, and cost data (text from the US DOE); the EVMS is a system that implements this approach}


\newduneword{core}{CORE}{CORE contributions are in either monetary units or labor hours. They can be technical components for the facility or experiment and the effort of the staff needed to produce, install, and test them;  major facilities for the experiment; or other products and services relevant for the completion of the facility or experiment} % 15 May - gina needs to `bless' this


\newduneabbrev{ahj}{AHJ}{Authority Having Jurisdiction}{An organization, office, or individual responsible for enforcing the requirements of a code or standard, or for approving equipment, materials, an installation, or a procedure (\gls{osha})}
\newduneword{cte}{CTE}{coefficient of thermal expansion}

\newduneabbrev{opc}{OPC}{open platform communications}{Open platform communications is a series of standards and specifications for industrial telecommunication} 
\newduneword{scada}{SCADA}{supervisory control and data acquisition}
\newduneword{ln}{LN$_2$}{liquid nitrogen}
\newduneabbrev{lapd}{LAPD}{Liquid Argon Purity Demonstrator}{Cryostat at Fermilab for long-term studies requiring a large volume of argon}

\newduneabbrev{pab}{PAB}{Proton Assembly Building}{Home of several \gls{lar} facilities at Fermilab}
\newduneword{hep}{HEP}{high energy physics}
%\newduneword{sc}{SC}{scientific computing}  % #33
\newduneword{cms}{CMS}{Compact Muon Solenoid experiment at CERN}
\newduneword{alice}{ALICE}{A Large Ion Collider Experiment, at CERN}
\newduneword{gpib}{GPIB}{general purpose interface bus}


\newduneabbrev{pfparticle}{PFParticle}{particle flow particle}{Each of the individual reconstructed particles in the hierarchy (or particle flow) describing the reconstructed event interaction}

\newduneabbrev{mcparticle}{MCParticle}{Monte Carlo Particle}{Individual true simulated particle}
\newduneword{au}{AU}{astronomical unit}
\newduneword{nufit}{NuFIT 4.0}{The NuFIT 4.0 global fit to neutrino oscillation data}

%\newduneabbrev{sgft}{SGFT}{term}{add def (DP install)} % #33
\newduneword{uhv}{UHV}{ultra high vacuum}
\newduneword{lps}{LPS}{laser positioning system}

\newduneword{unicamp}{UNICAMP}{University of Campinas, Sao Paulo, Brazil}
 
\newduneabbrev{fbk}{FBK}{Fondazione Bruno Kessler}{FBK is a research non-profit entity in Trento, Italy that partners in the development of technology with applications in various fields including High Energy Physics}

% the below not used 
%15may anne

%\newduneabbrev{fgt}{FGT}{fine-grained tracker}{A near detector module Add def??}

%\newduneword{consortium}{consortium}{A unit of organization in the  DUNE project focused on one major component of the far detector}
\newduneword{fft}{FFT}{fast Fourier transform}
\newduneabbrev{enob}{ENOB}{effective number of bits}{The effective number of bits is a measure of the dynamic range of an \gls{adc} and its associated circuitry. The resolution of an \gls{adc} is specified by the number of bits used to represent the analog value, in principle giving 2N signal levels for an N-bit signal. However, all real \gls{adc} circuits introduce noise and distortion. ENOB specifies the resolution of an ideal \gls{adc} circuit that would have the same resolution as the circuit under consideration}
\newduneabbrev{sndr}{SNDR}{signal to noise and distortion ratio}{Also known as SINAD. Ratio of the \gls{rms} signal amplitude to the mean value of the root-sum-square of all other spectral components, including harmonics, but excluding \gls{dc} levels. It is a good indication of the overall dynamic performance of an \gls{adc} because it includes all components which make up noise and distortion}
\newduneabbrev{sfdr}{SFDR}{spurious free dynamic range}{Spurious free dynamic range is the ratio of the \gls{rms} value of the signal to the \gls{rms} value of the worst spurious signal regardless of where it falls in the frequency spectrum. The worst spur may or may not be a harmonic of the original signal}
\newduneabbrev{thd}{THD}{total harmonic distortion}{Total harmonic distortion is the ratio of the \gls{rms} value of the fundamental signal to the mean value of the root-sum-square of its harmonics} 
\newduneword{tvs}{TVS}{transient voltage suppression}

\newduneword{riskprob}{risk probabilities}{The risk probability, after taking into account the planned mitigation activities, is ranked as 
 L (low $<\,$\SI{10}{\%}), 
M (medium \SIrange{10}{25}{\%}), or 
H (high $>\,$\SI{25}{\%}). 
The cost and schedule impacts are ranked as 
L (cost increase $<\,$\SI{5}{\%}, schedule delay $<\,$2 months), 
M (\SIrange{5}{25}{\%} and 2--6 months, respectively) and 
H ($>\,$\SI{20}{\%} and $>\,$2 months, respectively)}

\newduneabbrev{lbls}{LBLS}{laser beam location system}
{Auxiliary calibration system providing an independent location measurement of the ionization laser beams direction}


\newduneabbrev{lsst}{LSST}{Large Synoptic Survey Telescope}{8.4 m telescope with 3.2G-pixel camera that will start taking data in 2023}
\newduneabbrev{ska}{SKA}{Square Kilometer Array}{International radio telescope array planned to start data-taking in 2027}
\newduneabbrev{hyperk}{HyperK}{Hyper Kamiokande}{260 kt water Cerenkov neutrino detector to begin construction at Kamiokande in 2020}
\newduneword{lhcb}{LHCb}{LHC experiment dedicated to forward physics}
\newduneword{belleii}{Belle II}{B-factory experiment now running at KEK}

\newduneabbrev{ldm}{LDM}{light-mass dark matter}{Refers to dark matter particles with mass values much lower than the electroweak scale, specifically below the 1~GeV level}
 
\newduneabbrev{bnv}{BNV}{baryon-number violating}{Describing an interaction where \gls{baryonnumber} is not conserved}

\newduneword{bugey}{Bugey}{Neutrino experiment that operated at the Bugey nuclear power plant in France}

\newduneword{minosplus}{MINOS$+$}{The successor to the \gls{minos} experiment, utilizing the same detectors and beam line, but operating at higher beam energy tune than \gls{minos}, parasitic with \gls{nova}}

\newduneword{baryonnumber}{baryon number}{A quantity expressing the total number of baryons in a system minus the number of antibaryons}

%np02/4 and h2/4 new/updated per SK and EJ (by AH 10/22/20)
\newduneword{np04}{NP04}{Experiment in the CERN North Area \gls{h4} hadron beamline;  \gls{pdsp}}  % #33

\newduneword{np02}{NP02}{Experiment in the CERN North Area \gls{h2} hadron beamline;  \gls{pddp}}  % #33

\newduneword{h4}{H4}{CERN North Area hadron beamline used for the \gls{pdsp} test beam run}  % #33

\newduneword{h2}{H2}{CERN North Area hadron beamline used for the \gls{pddp} test beam run}  % #33

\newduneword{ua1}{UA1}{UA1 (Underground Area 1) was a particle detector at \gls{cern}'s  Super Proton Synchrotron (SPS). It ran from 1981 until 1990, when the SPS was used as a proton-antiproton collider, searching for traces of W and Z particles in collisions. (CERN) The UA1 dipole magnet was reused in the NOMAD experiment and currently provides the magnetic field for the \gls{t2k} ND280 detector}

\newduneword{ssc}{SSC}{The Superconducting Super Collider was to be a huge underground ring complex beneath the area near Waxahachie, Texas, USA, that would have been the world’s most energetic particle accelerator. It was begun in 1990, but canceled by the U.S. Congress in 1993 (scientificamerican.com Oct 2013)}

\newduneword{daphne}{DAPHNE}{Detector electronics for Acquiring PHotons from NEutrinos is a custom-developed warm front-end waveform digitizing electronics module derived from the readout system developed at Fermilab for the Mu2e experiment}
 
\newduneword{nersc}{NERSC}{National Energy Research Computing Facility at \gls{lbnl}}

% \newduneabbrev{integoff}{IO}{integration office}{The office that incorporates the onsite team responsible for coordinating integration and installation activities at SURF}
\newduneword{integoff}{integration project}{The \gls{doe} project element that organizes the onsite teams responsible for coordinating far detector installation and detector-facility integration activities at \gls{surf} as well as near detector installation activities at \gls{fnal}.  The integration project office includes overall LBNF/DUNE systems engineering, compliance and review offices} % #35

\newduneabbrev{sma}{SMA}{SubMiniature version A}{Connector interface for coaxial cables
with a screw-type coupling mechanism}

\newduneword{kloe}{KLOE}{KLOE is a $e^+ e^-$ collider detector spectrometer operated at DAFNE, the $\phi$-meson factory at Frascati, Rome. In DUNE it will consist of a \SI{26}{cm} Pb+scintillating fiber ECAL surrounding a cylindrical open detector region that is  \SI{4.00}{m} in diameter and \SI{4.30}{m} long. The ECAL and detector region are embedded in a \SI{0.6}{T} magnetic field created by a \SI{4.86}{m} diameter superconducting coil and a \SI{475}{tonne} iron yoke}

\newduneabbrev{ro}{RO}{review office}{An office within the \gls{integoff} that organizes reviews} % #33
%\newduneword{ro}{review office}{An office within the \gls{integoff} that organizes reviews } % #33

\newduneabbrev{doecd}{CD}{critical decision}{The U.S. DOE's Order 413.3B outlines a series of staged project approvals, each of which is referred to as a critical decision (CD)}

\newduneabbrev{lbnfspac}{LBNF/DUNE-US SPAC}{LBNF/DUNE-US Strategic Project Advisory Committee}{A committee charged by the host laboratory director to provide expert, independent advice on significant issues and strategies related to LBNF/DUNE-US project organization, management, and risks} % #36

\newduneabbrev{sand}{SAND}{System for on-Axis Neutrino Detection}{The beam monitor component of the near detector that remains on-axis at all times and serves as a dedicated neutrino spectrum monitor}

\newduneword{4850l}{4850L}{The depth in feet (1480 m) of the access level for the DUNE underground area at SURF; called the ``4850 level''} % #33

\newduneword{apb}{APB}{authorship and publications board}
\newduneword{crb}{CRB}{collaboration resources board}
\newduneword{cube}{CuBe}{beryllium copper, used to make \gls{sphd} \gls{apa} readout planes}
\newduneword{drift}{drift}{(1) refers to electron drift under the influence of an electric field in a \gls{tpc}; (2) an excavated horizontal corridors (tunnels) in the underground areas at \gls{surf}}
\newduneword{exposure}{exposure}{The integrated detector fiducial mass times beam intensity; it is proportional to the number of interactions and is used to normalize cross sections in a data sample}
\newduneword{fr4}{FR-4}{Flame-retardant fiberglass-reinforced epoxy resin laminate used in making PCBs and other detector components}
\newduneword{g10}{G-10}{Non-flame-retardant fiberglass-reinforced epoxy resin laminate used in making PCBs}
\newduneword{kapton}{Kapton}{A polyimide plastic film that is stable over a broad range of temperatures and is resistant to radiation damage}
\newduneword{shaft}{shaft}{A vertical excavation at \gls{surf} connecting with the surface}
\newduneword{winze}{winze}{A vertical excavation at \gls{surf} connecting two drifts, not connecting to the surface}
\newduneword{ib}{IB}{institutional board}
\newduneword{irb}{IBR}{institutional board representative}
\newduneword{sc}{SC}{depending on context, either speakers committee or scientific computing} % #33
\newduneword{sac}{SAC}{spokespersons advisory committee}
\newduneword{ccondc}{CCC}{code of conduct committee}
\newduneword{eoc}{EOC}{education and outreach committee}
\newduneword{indico}{Indico}{Web-based meeting organization tool}
% replaced by abbrev from ND CDR:  \newduneword{mec}{meson exchange current}{See \gls{2p2h}}
\newduneabbrev{htc}{HTC}{High Throughput Computing}{Computing facilities typically consisting of large numbers of commodity servers as opposed to a single large machine. Best suited for running large numbers of independent jobs in parallel, these facilities are what is usually meant by ``grid computing''}
\newduneword{dcache}{dCache}{A distributed, highly scalable (multi-PB) storage system, usable as both a standalone system and as a high-speed frontend to a tape storage system (such as /pnfs at Fermilab)}
\newduneabbrev{ifdhc}{IFDHC}{Intensity Frontier Data Handling Client}{A multi-protocol tool for data transfer and file delivery in jobs. It is able to automatically select transfer protocols based on source and destination characteristics}
\newduneabbrev{ifdh}{ifdh}{Intensity Frontier Data Handling}{The actual command invoked when using \gls{ifdhc}, on the command line, e.g. ifdh cp source\_file dest\_file}
\newduneabbrev{pnfs}{PNFS}{Pseudo Network File System}{A file system often used in large storage systems. Typically interaction is very similar to a regular NFS volume, but there can be some subtle and important differences}

% next 3 from NS cryogenics (Anne 8/28/20)
\newduneword{nde}{NDE}{non-destructive evaluation} % 33
\newduneword{psv}{PSV}{pressure safety valve}
\newduneword{pickling}{pickling}{steel pickling and oiling is a metal surface treatment finishing process used to remove surface impurities such as rust and carbon scale from hot rolled carbon steel}

% from ND CDR (the next 42 lines) Anne; Aug 31, 2020
\newduneabbrev{tof}{ToF}{time of flight}{The time a particle takes to fly between two visible interactions observed in the detector. If combined with the distance traveled by the particle, for example a neutron, it can be used for energy reconstruction}

\newduneword{pep4}{PEP-4}{TPC for the Positron Electron Project 4 Collider at Stanford}


\newduneword{ndlar}{ND-LAr}{\gls{lartpc} component of the near detector based on \gls{arcube} technology}

\newduneword{ndgar}{ND-GAr}{component of the near detector with a core gaseous argon \gls{tpc} surrounded by an \gls{ecal} and a magnet}

\newduneword{sfgd}{SuperFGD}{Super Fine-Grained Detector (SuperFGD) is a 3D granular plastic scintillator detector that adopts the same technology as \dword{3dst}. It will be installed in the \dword{t2k} \dword{nd280} system \cite{Abe:2019whr}. The \dword{3dst} design will inherit in large part from the SuperFGD detector}

\newduneword{nd280}{ND280}{Near Detector 280, is the \dword{t2k} magnetized near detector \cite{Abe:2011ks}}

\newduneword{fee}{FEE}{front-end electronics}

\newduneword{mpgd}{MPGD}{MicroPattern Gas Detectors}

\newduneword{rmm}{RMM}{Resistive MicroMegas}

\newduneabbrev{stv}{STV}{Single Transverse Variables}{Kinematical variables obtained by projecting the neutrino interaction onto the transverse plane}

\newduneabbrev{ccqe}{CCQE}{charged current quasielastic interaction} {An interaction where a neutrino scatters from a nucleon, producing a charged lepton and converting a neutron to a proton or vice versa}

\newduneword{sis}{SIS}{shallow inelastic scattering}

\newduneabbrev{res}{RES}{resonant scattering}{The mode of scattering where the target nucleon is excited to a resonant state and decays, typically producing one or more pions}

\newduneabbrev{hadw}{W}{invariant mass of the hadronic system}{Refers to the invariant mass of the hadronic system formed during the neutrino scatter}

\newduneabbrev{agky}{AGKY}{Andreopoulos-Gallagher-Kehayias-Yang}{A model for hadronization of non-resonant inelastic neutrino reactions used in \gls{genie}. At low invariant hadronic masses, typically less than 2.3\,GeV/c$^2$, it is a KNO-inspired empirical model anchored on several bubble chamber measurements of neutrino-induced shower characteristics. For invariant hadronic masses between 2.3 and 3.0\,GeV/c$^2$, the model transitions linearly to a \gls{genie}-tuned version of PYTHIA, which is also used for the simulation of events at higher invariant masses} % update from Costas Andreopoulos sept2020
%A model for hadronization of nonresonant inelastic neutrino reactions at invariant hadronic masses of less than 3 \SI{3}{\GeV\per\square c} used in \dshort{GENIE}} 

\newduneabbrev{clas}{CLAS}{CEBAF Large Acceptance Spectrometer}{A nuclear and particle physics detector located in the experimental Hall B at Jefferson Laboratory in Newport News, Virginia, United States. It is used to study the properties of the nuclear matter by the collaboration of over 200 physicists. Of particular relevance is its study of electron interactions with nuclei, including argon} 

\newduneabbrev{e4nu}{e4nu}{Electrons for Neutrinos}{A collaboration dedicated to using JLab's electron-scattering data to deliver improved neutrino-nucleus cross sections} 

\newduneabbrev{ldmx}{LDMX}{Light Dark Matter eXperiment}{The LDMX detector concept consists of a small precision tracker, and electromagnetic and hadronic calorimeters, all with near $2\pi$ azimuthal acceptance from the forward beam axis out to $\sim40^\circ$ angle. This detector would be capable of measuring correlations among electrons, pions, protons, and neutrons in electron-nucleus scattering at exactly the energies relevant for DUNE physics} 

\newduneabbrev{mec}{MEC}{meson-exchange currents} {An nuclear effect wherein pairs or larger groups of nucleons within a nucleus are bound together through the exchange of pions or other mesons. Neutrinos and other particles can scatter from these correlated pairs}

\newduneword{imt}{IMT}{Intranuclear momentum transfer}

\newduneword{miniboone}{MiniBooNE}{The Mini Booster Neutrino Experiment, at Fermilab, was designed to fully explore the LSND result}

\newduneabbrev{tms}{TMS}{Temporary Muon Spectrometer}{A muon spectrometer for the Near Detector that will be installed for the initial running period of DUNE, before the \gls{mpd} detector component is ready} % #34 09nov20 

\newduneabbrev{cvmfs}{CVMFS}{CERN VM File System}{A distributed file system designed for scalable, high-performance distribution of software to interactive and batch computers} %from tjunk sept2020 #21

\newduneword{kerberos}{Kerberos}{A strong authentication system used by the computing resources at Fermilab and CERN} %from tjunk sept2020 #23

\newduneabbrev{mrb}{MRB}{Multi Repository Build System}{A Fermilab-developed build system based on \dword{cmake} that allows development and builds of code from multiple repositories} %heidi sept2020

\newduneword{cmake}{Cmake}{CMake is an open-source, cross-platform family of tools designed to build, test and package software}%heidi sept2020

\newduneabbrev{ups}{UPS}{UNIX product support}{A software tool that sets up a consistent environment of versions of pre-installed products and their dependencies on UNIX-like platforms} % tjunk sept2020 #20

\newduneabbrev{upd}{UPD}{UNIX product distribution}{A tool for uploading and downloading pre-built software products between local systems and centralized software distribution servers.  UPD is not frequently used on DUNE because newer tools are more convenient} % tjunk sept2020 #22

\newduneabbrev{sso}{SSO}{single sign-on}{Often used to indicate that a group of services, such as DocDB or the DUNE Wiki share common sign-in credentials and active sessions.  Fermilab services that say "Sign in with SSO username and password" mean to use your Services username and password} % tjunk sept2020 #24

\newduneabbrev{vo}{VO}{virtual organization}{A database containing a list of member names, certificate distinguishing information, and a list of permissions members have to access computing grid and data resources} % tjunk sept2020 #27

\newduneabbrev{nas}{NAS}{network attached storage}{Disk storage that is available on computers but shared between them.  Relies on NFS mounts rather than authenticated file transfer protocols.  Usually found on interactive servers to provide space for home directories, app and data storage} % tjunk sept2020 #25

\newduneabbrev{nfs}{NFS}{network file system}{Industry-standard mechanism for mounting disks over a network.  Provides regular UNIX file and directory access} % tjunk sept2020 #26

\newduneword{recombination}{recombination}{Electrons freed from Argon atoms will sometimes recombine with the positive argon ions, either the same ones from which they came or nearby ones.  Sometimes called ``quenching"} % tjunk sept2020 #29

\newduneword{elife}{electron lifetime}{The attachment of electrons drifting through liquid argon to impurity molecules such as oxygen or water is parameterized by an exponential as a function of time with a time constant called the electron lifetime} % tjunk sept2020 #30

\newduneword{gplane}{grid plane}{The uninstrumented plane of wires in a SP \gls{apa} that borders the drift volume.   It shapes the signals and provides \gls{esd} protection} % tjunk sept2020 #31

\newduneword{gmesh}{grounding mesh}{A metal mesh attached to the SP \gls{apa} frame between the collection-plane wires and the space inside the frame where the \gls{pd} modules are installed.  It provides electric field uniformity so the collection-plane wires all have similar fields around them} % tjunk sept2020 #32

%%  begin #36 from Elaine 9 nov 20
\newduneabbrev{bcr}{BCR}{baseline change request}{A DOE project change, part of the change management system process}

\newduneword{ncav}{North Cavern}{the location of two of the planned four DUNE far detector modules at \gls{surf}}

\newduneword{scav}{South Cavern}{the location of two of the planned four DUNE far detector modules at \gls{surf}}

\newduneword{semp}{SEMP}{systems engineering management plan}  % #36 replaced cmp 

\newduneabbrev{tpcost}{TPC}{total project cost}{The DOE terminology for the total budget and contingency for the entire LBNF/DUNE-US project from CD-0 to CD-4}  % #36 replaced cmp 
\newduneabbrev{nsint}{NSI}{near site integration}{The scope of work at the near site for the \gls{integoff}} % #36 - can't put it together with other nsi

\newduneabbrev{opcost}{OPC}{other project costs}{The DOE project costs that support conceptual design, pre-operations commissioning, technical coordination, and power}

\newduneabbrev{moa}{MOA}{memorandum of agreement}{A project management methodology that documents an agreement between \gls{fnal} and the LBNF/DUNE-US Project for how Fermilab will support the project} %%%  MISSED THIS IN 36!!! ADDED AFTER
%%  end #36 

\newduneabbrev{croc}{CROC}{central readout chamber}{central (radial) readout chamber for the ND \gls{gartpc}} % from SManley 12/2/20 via email

\newduneabbrev{tki}{TKI}{transverse kinematic imbalance}{The imbalance among final-state particle momenta in the transverse plane to the neutrino direction; different aspects of the imbalance are sensitive to the detail of the nuclear effects in neutrino-nucleus interactions} % from SManley 25feb2021 via email


\newduneabbrev{iandi}{I\&I}{Integration and Installation}{One of the three project areas in the LBNF/DUNE-US \gls{doe} project, along with LBNF and DUNE-US } 

\newduneword{hmi}{HMI}{human-machine interface}

\newduneword{lhe}{LHe}{liquid helium}

\newduneword{echain}{energy chain}{mechanical machine elements used to carry and guide power to moving parts of machines or structures, as required for \gls{duneprism} to carry power, data, and utilities to and from each movable near detector component at any arbitrary position along its travel path}


% From vertical drift CDR 
% 22 Feb 2021
\newduneabbrev{sphd}{SPHD}{single-phase horizontal drift}{LArTPC design in which electrons drift horizontally to wire plane anodes (\glspl{apa}) that along with the front-end electronics are immersed in LAr}

\newduneabbrev{spvd}{SPVD}{single-phase vertical drift}{LArTPC design in which electrons drift vertically to PCB-based anodes at the top and bottom of the LAr volume, with a cathode in the middle}

\newduneabbrev{cru}{CRU}{charge-readout unit}{In the \gls{spvd} design an assembly of the \glspl{pcbp} plus adapter boards; four to a \gls{crp}} 

\newduneword{mpv}{MPV}{most probable value}

\newduneword{pcbp}{PCB panel}{In the \gls{spvd} design, one of four \SI{1.5x1.7}{m} assembled into a \gls{cru}}

\newduneword{anodepln}{anode plane}{a planar array of charge readout devices covering an entire face of a detector module}

\newduneword{msps}{MSPS}{megasamples per second}

\newduneword{gpsdo}{GPSDO}{\gls{gps} disciplined oscillator}
% end From vertical drift CDR  22 Feb 2021



\newduneword{tdaq}{TDAQ}{trigger and DAQ system} % 25feb2021 SPVD DAQ

\newduneword{nios}{NIOS}{Network Identity Operating System} 

%%% from the computing CDR 26 Feb 2021

\newduneabbrev{corsika}{CORSIKA}{COsmic Ray SImulations for KAscade}{a program for detailed simulation of extensive air showers initiated by high-energy cosmic ray particles}

\newduneabbrev{scd}{SCD}{Scientific Computing Division}{Fermilab's Scientific Computing Division}

\newduneabbrev{garsoft}{GArSoft}{Gaseous Argon Software}{A software toolkit similar to \gls{larsoft}, but targeted at the gaseous argon time projection chamber and calorimeter of \gls{ndgar}}

\newduneword{ndgarlite}{ND-GAr-Lite}{a temporary muon spectrometer consisting of the magnet and steel flux return of \gls{ndgar}, but with a simplified tracking chamber made with scintillating bars}

\newduneword{github}{GitHub}{a commerical web service providing code version management, storage, and browsing via \gls{git}}

\newduneword{git}{git}{a distributed version-control system, commonly used to manage software}

\newduneword{xrootd}{xrootd}{a high-performance data system widely used in HEP to store and to distribute data to jobs.  It allows streaming of data}

\newduneabbrev{gpuaas}{GPUAAS}{GPU As A Service}{a technique that allows many non-GPU-enabled compute nodes to share a GPU resource by sending it work over the network and waiting for results to be returned}
%%% end from the computing CDR 26 Feb 2021


% label (always all lower case) is used in chapter files as \dword{label}

% Add a `dune word' or a `dune abbreviation' using these models:

%\newduneword{label }{full term}{description with no period at end}

%\newduneabbrev{label}{abbreviation}{full term}{description with no period at end}

% ADDED TO OVERALL GLOSSARY 26 Feb by Anne; REFRESH glossary.tex 
%\newduneabbrev{corsika}{CORSIKA}{Cosmic Ray SImulations for KAscade}{a program for detailed simulation of extensive air showers initiated by high-energy cosmic-ray particles.}

%\newduneabbrev{scd}{SCD}{Scientific Computing Division}{Fermilab's Scientific Computing Division}

%\newduneabbrev{garsoft}{GArSoft}{Gaseous Argon Software}{A software toolkit similar to LArSoft, but targeted at the gaseous argon time projection chamber and calorimeter of ND-GAr}

%\newduneword{ndgarlite}{ND-GAr-Lite}{a temporary muon spetrometer consisting of the magnet and steel flux return of ND-GAr, but with a simplified tracking chamber made with scintillating bars}

%\newduneword{github}{GitHub}{a commerical web service providing code version management, storage, and browsing via git}

%\newduneword{git}{git}{a distributed version-control system, commonly used to manage software}

%\newduneword{xrootd}{xrootd}{XRootD is a high performance data system widely used in HEP to store and to distribute data to jobs.  It allows streaming of data.}

%\newduneabbrev{gpuaas}{GPUAAS}{GPU As A Service}{A technique which allows many non-GPU-enabled compute nodes to share a GPU resource by sending work to it over the network and waiting for results to be returned.}
% END ADDED TO OVERALL GLOSSARY 26 Feb by Anne

% ADDED TO OVERALL GLOSSARY; REFRESH glossary.tex \newduneabbrev{cvmfs}{{\tt cvmfs}}{CernVM File System}{The CernVM File System provides a scalable, reliable and low-maintenance software distribution service. It was developed to assist High Energy Physics (HEP) collaborations to deploy software on the worldwide-distributed computing infrastructure used to run data processing applications} %TRJ put this in an issue; different (shorter) def: A distributed file system designed for scalable, high-performance distribution of software to interactive and batch computers

% ADDED TO OVERALL GLOSSARY; REFRESH glossary.tex\newduneabbrev{mrb}{{\tt mrb}}{Multi Repository Build System}{A Fermilab developed build system based on \dword{cmake} that allows development and builds of code from multiple repositories}
%please don't use \tt

% ADDED TO OVERALL GLOSSARY; REFRESH glossary.tex\newduneabbrev{cmake}{{\tt cmake}}{Cmake}{CMake is an open-source, cross-platform family of tools designed to build, test and package software} %chged to newduneword

%\newduneabbrev{ups}{{\tt ups}}{Unix Product Support}{UPS is a Fermilab developed package manager that supports installation of multiple versions of a product and multiple builds per version.} %TRJ put this in an issue; exists, but should we change definition?

\newduneabbrev{vd}{VD}{Vertical Drift}{Detector using vertical drift TPC technology.}

\newduneabbrev{hd}{HD}{Horizontal Drift}{Detector using horizontal drift TPC technology.}

%\newduneabbrev{hv}{HV}{High Voltage}{High voltage system for TPC or other detector system.}

%\newduneabbrev{cisc}{CISC}{Cryogenic Instrumentation and Slow Controls}{DAQ subgroup responsible for monitoring detector systems. }

\newduneabbrev{poms}{POMS}{Production Operations Management System}{A Fermilab web service interface that enables automated jobs submission on distributed resources and subsequent monitoring and recovery of failed submissions, debugging and record keeping.}

\newduneabbrev{compass}{Compass}{
Common Muon and Proton Apparatus for Structure and Spectroscopy (COMPASS)}{
COMPASS is a multipurpose experiment at CERN’s Super Proton Synchrotron (SPS).}

\newduneabbrev{esnet}{ESnet}{Energy Sciences Network}{The Department of Energy's dedicated science network.}

\newduneword{project.py}{project.py}{xml based job configuration system developed by the MicroBooNE collaboration.}

\newduneabbrev{ppfx}{PPFX}{Package to Predict the FluX}{Fermilab supported package which implements hadron production corrections to geant4 simulations and propagates uncertainties for the NuMI  and LBNF beam lines.}

\newduneabbrev{ml}{ML}{Machine Learning}{Machine Learning}

\newduneabbrev{proto}{ProtoDUNE}{ProtoDUNE}{The protoDUNE detectors at CERN, includes the single phase, dual phase and vertical drift prototypes.} %Anne - not needed; already have {protodune}

\newduneabbrev{metacat}{MetaCat}{MetaCat}{Metadata Catalog, a modern replacement for the file description portion of the sam metadata catalog.}

\newduneword{g4lbnf}{g4lbnf}{LBNF neutrino beamline simulation program\cite{g4lbnf}.}

\newduneword{Geant4Reweight}{Geant4Reweight}{Framework for evaluating and propagating hadronic interaction uncertainties in Geant4\cite{Calcutt:2021zck}.}

\newduneword{geantnet}{G\'EANT}{G\'EANT interconnects Europe's national research and education networking (NREN) organizations with the high bandwidth, high speed and highly resilient pan-European backbone.
}

\newduneabbrev{nren}{NREN}{National Research and Education Network}{National level research computing network infrastructure.}

\newduneabbrev{vrf}{VRF}{Virtual Routing and Forwarding}{Networking overlays that provide  a logical routing infrastructure  that allows flexible traffic engineering.}

\newduneword{samvalue}{value}{A generic quantity describing a file in the \dword{sam} data catalog.}

\newduneword{samparameter}{parameter}{A user or experiment described quantity describing a file in the \dword{sam} data catalog.}

\newduneword{samproject}{project}{A server process running centrally that maintains a predefined list of files and delivers information about  their locations when asked by distributed processes. The project tracks success and failure of file processing.}

\newduneword{samconsumer}{consumer}{A client process that requests information about file locations from a \dword{samproject}, process the file and reports success or failure.}

\newduneword{samdataset}{dataset}{ A dynamic collection of files defined by queries to the \dword{sam} data catalog.}

\newduneword{samsnapshot}{snapshot}{A fixed collection of files corresponding to a \dword{sam} \dword{samdataset} at a particular point in time.}

\newduneabbrev{mql}{MQL}{\dword{metacat} Queryl Language} {A query language which supports queries of the \dword{metacat} data catalog, including parentage and logical functions such as union, join and subtraction.}

\newduneword{pdhd}{ProtoDUNE-HD}{ProtoDUNE with horizontal drift technology.  This refers to the \dword{apa}-based prototype to run in ProtoDUNE Run II.} 

\newduneword{pdvd}{ProtoDUNE-VD}{ProtoDUNE with vertical drift technology.  This refers to the \dword{crp}-based prototype to run in ProtoDUNE Run II.} 

\newduneabbrev{ucondb}{uconDB}{Unstructured Conditions Database}{Unstructured conditions database developed for FNAL fixed target experiements\cite{bib:ucondb}.}

\newduneabbrev{json}{JSON}{JavaScript Object Notation}{Open standard data interchange format that uses pair-value pairs and maps well onto python data formats such as tuples and lists.}

\newduneabbrev{dqmd}{DQMD}{Data Quality and Monitoring Database}{Database storing the results of data quality monitoring.}

\newduneabbrev{db}{DB}{Database}{Database, can have many formats based on optimal use.}

\newduneabbrev{vm}{VM}{Virtual Machine}{Emulator of a physical computer that allows multiple users to configure different operating systems  while sharing physical hardware. }

\newduneword{enstore}{enstore}{Enstore is a mass storage system developed by Fermilab that provides distributed access and management of data stored on tapes \cite{Litvintsev:2012dw}.}

%\newduneabbrev{nas}{NAS}{Network Attached Storage}{High performance disk available on interactive nodes.}


\newduneword{stashcache}{StashCache}{A distributed caching federation that enables opportunistic users to utilize nearby opportunistic storage\cite{bib:stashcache}}

\newduneabbrev{hdf5}{HDF5}{Hierarchical Data Format}{Data format\cite{bib:hdf5} widely used in \dword{ml}.}

\newduneword{pdcoldbox}{ProtoDUNE-VD coldbox}{  Coldbox used for testing the \dword{pdvd} electronics and \dword{daq} systems.}

\newduneword{protoii}{ProtoDUNE Run II}{2022-23 run of the \dword{proto} prototypes at CERN.}

\newduneword{glideinwms}{GlideinWMS}{GlideninWMS \cite{sfiligoi2009pilot}.}

\newduneword{mars}{MARS}{The MARS code system is a set of Monte Carlo programs for detailed simulation of coupled hadronic and electromagnetic cascades, with heavy ion, muon and neutrino production and interactions.}

\newduneabbrev{fife}{FIFE}{Fabric for Intensity Frontier Experiments}{Fermilab computing infrastructure for Intensity Frontier Experiments \cite{box2014fife}.}

\newduneword{tr}{trigger record}{A data record produced by the \dword{dune} \dword{daq} system.  A Trigger Record can contain multiple interaction "events" or none.}%anne made it lower case

\newduneword{postgres}{Postgres}{PostgreSQL, also known as Postgres, is a free and open-source relational database management system used extensively for databases in HEP.}

\newduneword{hllhc}{HL-LHC}{High-luminosity \gls{lhc}} %anne added

\newduneword{garfield}{GARFIELD}{add def} %anne added

\newduneabbrevs{iov2}{IOV}{interval of validity}{intervals of validity}{Interval over which something is valid} %anne added - plural was wrong - may need to fix orig {iov} too

\newduneword{fhicl}{FHICL}{Fermilab Hierarchical Configuration Language; a standard configuration language for the
storage, communication, and manipulation of scientific parameter sets}  %anne added

\newduneword{hwdb}{HWDB}{hardware database}   %anne added

\newduneword{singlecube}{SingleCube}{add def}  %anne added

\newduneword{s3}{S3}{add def}  %anne added

\newduneword{openstack}{OpenStack}{add def}  %anne added

\newduneword{cric}{CRIC}{add def}  %anne added


\newduneword{htcondorce}{HTCondor-CE}{add def}  %anne added

\newduneabbrev{ccb}{CCB}{Computing Contributions Board}{The Computing Contributions is made up of institutional representatives for larger countried and laboratories. It meets annually to negotiate collaboration contributions to computing infrastructure.}

% Heidi S.  - this leads to duplicates so commented it out 
%\input{common/glossary-anne}

% Do this here to make use of the experiment name and title macro
\hypersetup{
    pdftitle={\expshort TDR \thedocsubtitle},
    pdfauthor={\expshort Collaboration},
    final=true,
    colorlinks=false,
    linktocpage=true,
    linkbordercolor=blue,
    citebordercolor=green,
    urlbordercolor=magenta,
    filecolor=black,
    pdfpagemode=UseOutlines,
    pdfborderstyle={/S/U},  
}



