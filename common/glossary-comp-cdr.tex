% label (always all lower case) is used in chapter files as \dword{label}

% Add a `dune word' or a `dune abbreviation' using these models:

%\newduneword{label }{full term}{description with no period at end}

%\newduneabbrev{label}{abbreviation}{full term}{description with no period at end}



% ADDED TO OVERALL GLOSSARY; REFRESH glossary.tex \newduneabbrev{cvmfs}{{\tt cvmfs}}{CernVM File System}{The CernVM File System provides a scalable, reliable and low-maintenance software distribution service. It was developed to assist High Energy Physics (HEP) collaborations to deploy software on the worldwide-distributed computing infrastructure used to run data processing applications} %TRJ put this in an issue; different (shorter) def: A distributed file system designed for scalable, high-performance distribution of software to interactive and batch computers

% ADDED TO OVERALL GLOSSARY; REFRESH glossary.tex\newduneabbrev{mrb}{{\tt mrb}}{Multi Repository Build System}{A Fermilab developed build system based on \dword{cmake} that allows development and builds of code from multiple repositories}
%please don't use \tt

% ADDED TO OVERALL GLOSSARY; REFRESH glossary.tex\newduneabbrev{cmake}{{\tt cmake}}{Cmake}{CMake is an open-source, cross-platform family of tools designed to build, test and package software} %chged to newduneword

%\newduneabbrev{ups}{{\tt ups}}{Unix Product Support}{UPS is a Fermilab developed package manager that supports installation of multiple versions of a product and multiple builds per version.} %TRJ put this in an issue; exists, but should we change definition?