\documentclass[../main-v1.tex]{subfiles}
\begin{document}
\chapter{Summary}
\label{ch:conclusions}

In this document we have outlined the major use cases and challenges identified for \dword{dune} computing for time scales ranging to next year \dword{pd2} to long term operations of the full \dword{dune}/\dword{lbnf}  \dword{fd} and \dword{nd}. 

Substantial preliminary studies and design work have been outlined in the previous sections. 

\section{Review of Challenges}
 

In the introduction to this document (Section \ref{intro:challenges}),  the following challenges were proposed, some unique to DUNE, some more general but all significant. 

%\section{Summary of Challenges %\hideme{Schellman-draft}}

%D%UNE offline computing faces four major challenges, some of which are unique to DUNE and others shared widely by \dword{hep} experiments.  

\begin{description}
\item{\bf Large memory footprints}  DUNE events, with multiple data objects consisting of  thousands of channels with thousands of time samples   present formidable challenges for reconstruction on typical \dword{hep} processing systems.  In this document we have described the use cases (Chapter \ref{ch:use}) and proposed event model (Chapter \ref {ch:format}), framework (Chapter \ref{ch:fworks}) and compute requirements (Chapter \ref{ch:cm}) to address these use cases. 




\item{\bf Storing and processing data on heterogeneous international  resources -} 
DUNE depends on the combined resources of the collaboration for large-scale storage and processing of data.   Tools for using shared resources ranging from small-scale clusters to dedicated \dword{hpc} systems are being designed but will need to be further developed and maintained.  
%Fortunately, \dword{hep}, through the \dword{wlcg}, \dword{osg} and \dword{hsf}  has a well developed ecosystem of tools that allow reasonably transparent use of collaboration computing resources.  
Chapters~\ref{ch:est}, \ref{ch:cm} and~\ref{ch:datamgmt} described the data volumes, computing model and data management plans. 

\item{
\bf Machine learning - }  Use of machine learning techniques can greatly improve simulation, reconstruction and analysis of data. However, integration of \dword{ml} techniques into a software ecosystem of the size and complexity of a large \dword{hep} experiment requires substantial effort beyond the original demonstration. 
%How is the \dword{ml} trained?  What special data format or processing requirements are present? How is the algorithm versioned and preserved to ensure reproducibility?   
Chapters~\ref{ch:use} and~\ref{ch:codemgmt} discussed some known applications and their management, but substantial effort will be needed to keep up with this rapidly evolving field. 

\item{\bf Keeping it all going -}  %anne There are a 
There is a very large suite of activities that are 
not necessarily novel but
still need to be done over the full lifetime of the experiment. The chapters  on databases(\ref{ch:db}), data management(\ref{ch:datamgmt}, \ref{ch:place}), workflow(\ref{ch:wkflow}), 
services (\ref{ch:serv}), authentication (\ref{ch:auth}), code management (\ref{ch:codemgmt}) and training and documentation (\ref{ch:train})
describe many of the systems which must be designed and then maintained for the lifetime of the experiment. 

In addition, while our compute needs are not fundamental challenges on the scale of the large LHC experiments, substantial human effort, \dword{cpu} resources, dedicated storage, and fast networks will need to be acquired and made easily available to the collaboration.  Chapter \ref{ch:cm} provided estimates of our needs while Sections  \ref{sec:ccb}, \ref{ch:cm} and \ref{ch:resource} describe our existing collaborative resources and future needs. 


\end{description}

\section{Conclusion}

In conclusion, we have identified the major challenges facing \dword{dune} computing and have begun the process of designing and deploying solutions. We have preliminary estimates of resource needs and have broad connections to collaborating institutions and countries who have already provided substantial resources on a voluntary basis. However, as \dword{dune} grows, more resources will need to be identified to support timely reconstruction and analysis the full physics program.  We hope that this document will be helpful in justifying and acquiring those resources. 


\end{document}