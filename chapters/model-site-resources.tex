\documentclass[../main-v1.tex]{subfiles}
\begin{document}
\chapter{Site Resources \hideme{McNab - in progress}}
\label{ch:sites}

%%%%%%%%%%%%%%%%%%%%%%%%%%%%%%%%

\section{Overview of sites}
\label{sec:sites:overview}

We expect the computational, storage, and archival resources will be provided by a mixture of Grid, Cloud, and High Performance Computing sites. 

\section{Grid sites \hideme{McNab - draft}}
\label{sec:sites:grid}  %% fix label according to section

The bulk of the computing resources for DUNE is assumed to be provided by Grid sites, certainly in the early years. These resources are presented as the family of grid services, such as HTCondor-CE, at sites which are members of grid federations such as OSG, EGI, and WLCG.

These infrastructures have seen an evolutionary development, with new service implementations and protocols introduced at on an experimental basis at a few sites, and building up to become widely accepted if they are proven. We expect to be able to track these changes as necessary, sharing experience with the other large particle physics experiments making use of them and through forums as the WLCG Grid Deployment Board and its working groups.


\section{Analysis Facilities \hideme{Tejin Cai for Victoria -needed
}} \label{sec:sites:anacluster}

\section{High Performance Computing sites \hideme{Herner and Norman - needed}}
\label{sec:sites:hpc}
Something equally true and reassuring about using HPC sites, and their specific constraints. HEPCloud's role in HPC

\section{Cloud providers\hideme{Timm/Yang needed}}
\label{sec:sites:cloud}

Something equally true and reassuring about using commercial cloud providers and arms-length academic cloud providers, and their specific constraints. HEPCloud's role
\end{document}