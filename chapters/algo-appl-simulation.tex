\section{Simulation Algorithms \hideme{TRJ}}
\label{sec:algo:sim}

\subsection{Beam simulation}
\label{sec:beamsim}

TBS

\subsubsection{Event Generators}
\label{sec:eventgen}

Extracting physics results from the DUNE experiment requires comparing the observed data with simulations which include detailed simulations of the physics processes under study as well as the response of the detectors.  The physics simulation is performed by the neutrino generators GENIE~\cite{Andreopoulos:2009rq}, NuWRO~\cite{NuWro2012}, GIBUU~\cite{Gallmeister:2016dnq}, NEUT~\cite{Hayato:2009zz}, and others.  Cosmic-ray simulations are performed with \dword{corsika}~\cite{Wentz:2003bp,Dembinski:2020wrp} for detectors on the surface, and MUSUN/MUSIC for detectors deep underground~\cite{Kudryavtsev:2008qh,LBNEDOCDB9673}.  Radiological decays are modeled with BXDECAY0~\cite{Ponkratenko:2000um} and the DUNE-specific RadioGen.

\subsubsection{Detector Simulation}
\label{sec:detsim}

The factorization of the simulation into a generation stage and a detector simulation stage is a common situation in collider experiments, such as ATLAS and CMS.  The fact that the interactions simulated by generators for collider physics happen inside an evacuated beampipe means that the details of the detector geometry and materials are not relevant for most event generation.  Lists of four-vectors of particles emerging from a primary vertex will suffice.  In a neutrino experiment, however, the detector material is the target material, and hence the generators must be aware of the detector geometry and materials, which affects the structure and performance of the generator code.  Currently, \dword{genie}, \dword{corsika}, MUSUN/MUSIC, BXDECAY0 and RadioGen are integrated with \dword{larsoft}.  \dword{genie} is integrated with \dword{garsoft}.


Two classes of simulation of DUNE's detectors exist at the time of writing.  Parameterized, or ``fast'' detector simulations involve smearing truth-level physics quantities based on expected detector performance metrics, such as acceptance and energy resolution.  These fast simulations are useful when optimizing detector designs, and for engaging physicists outside of the DUNE collaboration.  Full simulations, on the other hand, are based on detailed geometry models and \dword{geant4}~\cite{Agostinelli:2002hh,Allison:2016lfl}, and are needed for extraction of publication-level results.

In a \dword{geant4}-based simulation of a DUNE detector, \dword{geant4} is used only to simulate the interacting particles from neutrino scatters and other processes of interest, and energy depositions in the active detector material are stored as a distinct data product. The the many low-energy drifting electrons and scintillation photons are simulated in separate steps.  Drifting electrons are simulated parametrically using a model based on the measured drift velocity, longitudinal and transverse diffusion coefficients, and a parameterized model of space charge.  This last effect is particularly pronounced at \dword{pdsp}, giving rise to distortions in the apparent positions of particles of up to 30~cm.  Field distortions due to external imperfections such as the grounded electron diverters in \dword{pdsp}.

Once the electrons drift to the anode plane in wire-based \dwords{lartpc} in the simulation, a detailed two-dimensional model of the wire responses is applied~\cite{Abi:2020mwi}.  The two dimensions are wire number and time, and the effects of induced currents on neighboring wires are included in the simulation.  The electronics response function is folded in to a final model of the observed waveforms.  Simulated waveforms have been compared with real ones and are found to be very similar in \dword{pdsp}.
In the pixel-based \dword{ndlar} and \dword{ndgar}, the electronics simulation is at a simpler level, as the electronics have not been fully designed.

A large amount of code re-use and sharing via the design of \dword{larsoft} allows for the development of simulation algorithms for \dword{pddp}, the dual-phase far detector, and the Vertical Drift detector proposals.  Only the geometry, the field description, and the anode-plane models need to be updated; the rest of the simulation chain is re-used.

Photon propagation, scattering, absorption and detection are modeled in \dword{larsoft}'s simulation via a photon visibility lookup table, which gives the probability that a photon, emitted in a random direction at a specific location in the active volume of the detector, is detected by a specific photon detector.  The spatial granularity of the lookup table is a few cm.  This lookup table is populated with values that are computed from a \dword{geant4} simulation of scintillation photons in \dword{lar}, with the assumed values of the light attenuation length, the Rayleigh scattering length, the reflectivities of the surfaces inside the detector, and the transparency of the wire planes.  Scintillation photons are not yet simulated in the \dword{ndgar}, but photon detection systems are under consideration. 

\dword{ndgar} also has a calorimeter and a muon system.  The calorimeter is simulated via parameterized responses to the \dword{geant4}-simulated energy deposits, as are the responses to tracks in \dword{ndgarlite} and \dword{sand}.