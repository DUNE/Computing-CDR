\documentclass[../main-v1.tex]{subfiles}
\begin{document}
\section{Reconstruction Algorithms\hideme{ TRJ - draft}}
\label{sec:algo:reco}

This section gives brief outlines of reconstruction algorithms in use in DUNE's detectors.  The \dword{fd} will consist of four modules, at least three of which will be \dwords{lartpc}, and the fourth module, the module of opportunity, has yet to be designed.  The near detector will have a \dword{lartpc}, in addition to a muon spectrometer and/or \dword{ndgar}, and \dword{sand}, each of which consists of multiple subdetectors.  \dword{ndlar} will have a pixel-based readout, which enables a much easier reconstruction of densely-packed objects.  The \dword{pdsp} detector is also a \dword{lartpc}, and a vertical-drift prototype is planned.

\subsection{Liquid Argon TPC Reconstruction}
\label{sec:algo:reco:lartpc}

\subsubsection{Wire- and Strip-Based Liquid Argon TPC Reconstruction}
\label{sec:algo:reco:lartpc:wirestrip}

Reconstructing events in a \dword{lartpc} is challenging.  Each trigger record contains a large number of waveforms, one for each readout channel.  In \dword{pdsp}, the waveforms typically were 6000 time ticks long, with one \dword{adc} sample per time tick.  A detailed description of the reconstruction procedures, starting with these waveforms, is given in~\cite{Abi:2020mwi}, and summarized briefly here.

Typically, data from a single-phase \dword{lartpc} \dword{daq} are stored as compressed sequences of data frames, containing 128 or 256 channels' worth of data on each time sample.  These must be decompressed and rearranged into sequences of \dword{adc} values corresponding to the original waveforms.  Additional decoder modules are run to unpack \dword{pds}, \dword{crt} trigger and timing \dword{daq} fragments.
From there, mitigations are applied to correct, as best as possible, known failures of the front-end electronics.  As yet, these are only needed for the TPC readout electronics.  A particular failure mode in \dword{pdsp} is the sticky-code problem, which is mitigated by interpolating neighboring ADC samples in time.  Channel-to-channel gain corrections are applied, correlated noise is filtered out, the pedestal is found, and a correction is made for the \dword{ac} coupling between the preamp and the ADC.  The ADC mitigation, correlated noise removal, gain correction, pedestal finding and \dword{ac} coupling corrections are grouped into the data preparation module.  The \dword{wirecell} toolkit provides a two-dimensional deconvolution in wire index and charge arrival time, and it associates charge in the three views to produce a three-dimensional image of the charge locations where it was produced.  

The calibrated, filtered, deconvolved waveforms are then used as input to a hit-finding algorithm, which identifies peaks in the waveforms and fits Gaussians to each proposed peak.  These hits are then used as inputs to general reconstruction algorithms, such as the SpacePointSolver Pandora, TrajCluster, and PMA, which identify clusters, tracks and showers in three dimensions by associating objects in the three two-dimensional views.  The calorimetry modules sum up and calibrate the charge deposits for use in energy reconstruction and \dword{pid}.  The parameters of the clusters, tracks and showers are stored in ROOT trees that end users can analyze rapidly and repeatedly.

Additional modules find hits in the photon detector waveforms and group them into clusters called flashes.  The \dword{crt} data are analyzed in dedicated modules that produce associations between hits in the upstream and downstream \dword{crt} modules for use in analyses.

Separating track-like energy deposits from shower-like deposits is a key part of many DUNE analyses.  This is accomplished with a \dword{cvn} called {\tt EmTrackMichelID} in Table~\ref{tab:protodune_cpu_reco_by_module}.  It is one of the most CPU-intensive operations in the \dword{pdsp} reconstruction chain, when the algorithm is run on a grid node lacking a GPU.  Recently, however, a \dword{gpuaas} technique has been developed~\cite{Wang:2020fjr}, enabling a speedup of the order of a factor of ten, though it depends on the ratio of CPU-only nodes to GPU resources.


\subsubsection{Pixel-Based Liquid Argon TPC Recontruction}
\label{sec:algo:reco:lartpc:pixels}



\begin{longtable}
{l r}
\caption[Processing time for reconstruction modules for a \dword{pdsp} event]{Wall-clock module execution times for the reconstruction of a typical ProtoDUNE-SP event, in seconds.  The event is a data event from Run 5809, a 1 GeV beam run.} \\ \toprowrule
  \rowcolor{dunesky}
Module Label & time/event (sec)\\ \toprowrule
RootInput(read)                          &     0.147283          \\
timingrawdecoder:TimingRawDecoder        &     0.0095498         \\
ssprawdecoder:SSPRawDecoder              &     0.126099          \\
crtrawdecoder:CRTRawDecoder              &     0.0181903         \\
ctbrawdecoder:PDSPCTBRawDecoder          &     0.0204753         \\
beamevent:BeamEvent                      &      1.6154           \\
caldata:DataPrepByApaModule              &      83.0324          \\
wclsdatasp:WireCellToolkit               &      79.8616          \\
gaushit:GausHitFinder                    &      1.56342          \\
nhitsfilter:NumberOfHitsFilter           &    0.00177549         \\
reco3d:SpacePointSolver                  &      9.44157          \\
hitpdune:DisambigFromSpacePoints         &      1.52541          \\
pandora:StandardPandora                  &      39.8787          \\
pandoraWriter:StandardPandora            &     0.370009          \\
pandoraTrack:LArPandoraTrackCreation     &      4.57221          \\
pandoraShower:LArPandoraShowerCreation   &      3.73432          \\
pandoracalo:Calorimetry                  &      2.11152          \\
pandoracalonosce:Calorimetry             &      1.90852          \\
pandorapid:Chi2ParticleID                &     0.0115046         \\
pandoracali:CalibrationdEdXPDSP          &     0.106919          \\
pandoracalipid:Chi2ParticleID            &    0.00851985         \\
pandoraShowercalo:ShowerCalorimetry      &      3.26953          \\
pandoraShowercalonosce:ShowerCalorimetry &      3.17698          \\
emtrkmichelid:EmTrackMichelId            &      233.794          \\
ophitInternal:OpHitFinder                &     0.0190084         \\
ophitExternal:OpHitFinder                &    0.00828164         \\
opflashInternal:OpFlashFinder            &     0.0172739         \\
opflashExternal:OpFlashFinder            &    0.000597182        \\
opslicerInternal:OpSlicer                &     0.0184422         \\
opslicerExternal:OpSlicer                &    0.00538796         \\
crttag:SingleCRTMatchingProducer         &     0.0231673         \\
crtreco:TwoCRTMatchingProducer           &     0.0128138         \\
anodepiercerst0:T0RecoAnodePiercers      &      1.07673          \\
pandora2Track:LArPandoraTrackCreation    &      11.6525          \\
pandora2calo:Calorimetry                 &      4.95932          \\
pandora2calonosce:Calorimetry            &      4.55118          \\
pandora2pid:Chi2ParticleID               &     0.0211641         \\
pandora2cali:CalibrationdEdXPDSP         &     0.0867829         \\
pandora2calipid:Chi2ParticleID           &     0.0206842         \\
pandora2Shower:LArPandoraShowerCreation  &      4.17815          \\
pandora2Showercalo:ShowerCalorimetry     &      4.1729           \\
pandora2Showercalonosce:ShowerCalorimetry&      3.83562          \\
TriggerResults:TriggerResultInserter     &     0.000349372        \\
RootOutput                               &    2.8912e-05         \\
RootOutput(write)                        &     2.79458               \\
{\bf Total:}                             &     {\bf 507.881}      \\ \colhline
\label{tab:protodune_cpu_reco_by_module}
\end{longtable}



\subsubsection{Pixel-Based Gaseous Argon TPC Recontruction}
\label{sec:algo:reco:gartpc:pixels}

The \dword{ndgar} consists of two primary detectors -- a copy of the ALICE pixel-based \dword{tpc} in a 10-bar gas consisting predominantly of argon, surrounded by a calorimeter and a superconducting magneti coil.  A muon system is envisaged outside of the calorimeter but is not included in the simulation or the reconstruction at the time of writing.  Data are unpacked and hits are found on the per-readout-pad waveforms, similarly to how the initial stages of reconstruction for a \dword{lartpc} are followed.  Hits are then clustered into TPC clusters, which reduces the memory and CPU usage of subsequent steps, and also increases the spatial resolution of individual clusters.  Vector hits are found by grouping TPC clusters into short line segments, and the vector hits themselves are grouped into track candidates by a pattern recognition module.  A track fit based on a Kalman filter then finds the best estimates of the track parameters.  Vertices are then found using tracks with nearby endpoints.  Because there is a cathode in the middle of the drift volume in the nominal design, a cathode stitch module is then run to associate track segments on either side of the cathode, bring them together by solving for the best interaction time that lines the segments up, and also moves the associated tracks and vertices.  Vees ($K^0_s$ and
$\Lambda^0$ candidates) are found by pairing nearby tracks together, optionally requiring a displacement from another vertex, and forming the invariant mass.  The calorimeter consists of a mixture of strips and pads.  Calorimeter hits are found in the \dword{sipm} waveforms provided by the calomrimeter \dword{daq}, and they are clustered together to form reconstructed three-dimensional energy deposit objects with positions, directions, and energies.




\begin{dunetable}
[Average \dshort{ndgar} event wall-clock module execution times]
{l r}
{tab:garsoft_cpu_reco_by_module}
{Average wall-clock processing time for reconstruction modules for GArSoft events consisting of just one interaction in the gas.  An actual spill will contain of order 60 interactions, mostly  in the calorimeter, though many tracks will pass through the gas.}
Module Label & time/event (sec)\\ \toprowrule
RootInput(read)                             &    0.000353765       \\
init:EventInit                         &    1.31275e-05       \\
hit:CompressedHitFinder                &    0.00488308        \\
tpcclusterpass1:TPCHitCluster          &     0.0091922        \\
vechit:tpcvechitfinder2                &     0.0103787        \\
patrec:tpcpatrec2                      &     0.0130245        \\
trackpass1:tpctrackfit2                &     0.014211         \\
vertexpass1:vertexfinder1              &    0.000851081       \\
tpccluster:tpccathodestitch            &     0.0269436        \\
track:tpctrackfit2                     &     0.0135842        \\
vertex:vertexfinder1                   &    6.19847e-05       \\
veefinder1:veefinder1                  &    9.96417e-05       \\
sipmhit:SiPMHitFinder                  &    0.00153375        \\
sscalohit:CaloStripSplitter            &     0.0547754        \\
calocluster:CaloClustering             &    0.00492599        \\
trkecalassn:TPCECALAssociation         &    0.000237503       \\
TriggerResults:TriggerResultInserter        &    2.36397e-05       \\
RootOutput                                  &    3.6783e-06        \\
RootOutput(write)                           &    0.214077         \\
{\bf Total}                                  &     {\bf 0.369802}       \\ 
\end{dunetable}



\section {Anything missing? \hideme{??? -needed}}
\end{document}