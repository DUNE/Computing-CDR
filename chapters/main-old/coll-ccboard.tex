\documentclass[../main-v1.tex]{subfiles}
\begin{document}
\chapter{Computing Contributions Board (CCB) \hideme{ Peter Clarke - draft}}
\label{ch:ccb}

%%%%%%%%%%%%%%%%%%%%%%%%%%%%%%%%
%\section{xyz}
%\label{sec:ccb:xyz}  %% fix label according to section

The \dword{ccb} has been set up to formalize the contribution of DUNE partners to the computing and storage capacity required for computing production. In order to remain scale-able, the \dword{ccb} considers a nation as the natural unit of aggregation to request and then seek pledges for resources. The \dword{ccb} does, however, recognize that whilst within some countries centralized coordination is natural, this is not true for others. Nevertheless, requests will be published at a national granularity, with the assumption that Institutes within each nation can coordinate.

In order to set a fair-share request for computing capacity, the \dword{ccb} will eventually use a proxy metric for the relative proportion of each nation in DUNE. This is likely to be PhD paper authors once in the exploitation phase. This is, however, not relevant during construction as the current listing of DUNE members in the data base is a very poor proxy for this. Therefore, during construction all nations with more than a minimum number of DUNE participants or providing Tier-1 or large Tier-2 capacity to LHC experiments, are asked to provide a "reasonable" share (see below). We refer to these as compute-active-nations.
This is very flexible and is up to each nation to decide if it wishes to be classified as a compute-active-nation.

At present the \dword{ccb} is composed of a Chair, one member per compute-active-nation, one member for each of FNAL, CERN and BNL, and the Computing and Software Consortium  Management ex officio.

The Computing Consortium management produces an overall requirements document that should be scrutinized by the FNAL CRSG. The \dword{ccb} receives this document, and then seek pledges to meet those requirements. As host lab, FNAL plans to provide ~25\% of the capacity, including the primary tape service.
CERN also currently provides a substantive capacity, in particular for proto-DUNE.
The aim is then for the remaining capacity to come from other contributors according to the prevailing computing model. A substantial proportion of capacity is expected from outside of the USA  (~50\%). Contributions of at least 5-20\% are requested depending upon the circumstance and capability of each compute-active-nation.
Pledges of capacity received are recorded in the CRIC information system. In due course this process could be formalized into a (non-binding) MOU.

The \dword{ccb} may also receive requests and information from the Computing Consortium Management in respect of any other non-capacity matter. The \dword{ccb} may seek to help with such requests where they pertain to national contributions. This may include promotion of requests for software engineering support to be propagated within each nation.

\end{document}