\documentclass[../main-v1.tex]{subfiles}
\begin{document}
\chapter{Services Overview \hideme{Timm - needs more}}
\label{ch:serv}

%%%%%%%%%%%%%%%%%%%%%%%%%%%%%%%%
%\section{xyz}
%\label{sec:serv:xyz}  %% fix label according to section
\section{Introduction}
DUNE computing is dependent on a number of services that are not operated by DUNE itself.
Some of these are provided by the host lab, others are provided by the various remote sites where
computing is done, and still others are operated by commercial companies and hosted in the cloud.

% trj: added SharePoint -- March 2

\section{Host Lab Provided Services}
DUNE relies on the host lab to provide networking connections, authorization and authentication services, tape and disk based storage for data, code, and databases.  We use the conference scheduling services (currently Indico), SharePoint, the DocDB service, and a number of databases managed by the scientific databases group including but not limited to the hardware database,
the IFBeam database, the SAM database, and several others.  The data management services at the host lab, namely Rucio and Metacat are also operated and managed by the Fermilab scientific computing division.  
The host lab also provides a significant amount of interactive and batch computing, as well as the workload management
system GlideinWMS to provision resources at remote sites around the world and the HEPCloud system to access high performance computing sites as well as commercial clouds if necessary.
We expect that the Elastic Analysis Facility currently in beta will also be available to DUNE.
Although the DUNE token issuer for WLCG tokens is hosted at CILogon.org this business relationship is managed by Fermilab
on DUNE's behalf.
The dunescience.org family of web sites including the Wiki is also hosted at Fermilab.
\section{Remote Site Provided Services}
DUNE receives compute and storage resources from a number of sites around the world as detailed in the section on the Computing Contributions Board. In addition DUNE also receives a few services that are hosted at CERN.  These include the EDMS document
management system and  the MONIT monitoring system. Some DUNE meetings are organized using CERN's Indico instance.

%trj:  specified LaTeX for Overleaf March 2, and added Microsoft Teams.  Added a mention of CERN's indico instance

\section{Cloud Hosted Services}
DUNE uses the GitHub service for its code management, the Slack and Microsoft Teams services for interactive communication, the Overleaf service for editing \LaTeX{} documents, the Zoom teleconferencing application.
\todo{section on assumptions about external services that we depend on}
\end{document}