\documentclass[../main-v1.tex]{subfiles}
\begin{document}
\chapter{Services Overview \hideme{Timm - needs more}}
\label{ch:serv}

%%%%%%%%%%%%%%%%%%%%%%%%%%%%%%%%
%\section{xyz}
%\label{sec:serv:xyz}  %% fix label according to section
\section{Introduction}
DUNE computing is dependent on a number of services that are not operated by DUNE itself.
Some of these are provided by the host lab, others are provided by the various remote sites where
computing is done, and still others are operated by commercial companies and hosted in the cloud.

% trj: added SharePoint -- March 2, 2022
%ST complete rewrite based on %trj comments, split up into sections, added a lot more services.
\section{Host Lab Provided Services}
DUNE relies on the host lab, \dword{fnal}, to provide a wide variety of services.  These include web, database, computing, and storage services.  The networking services are described in \ref{ch:netw}.  We summarize the other services briefly.

\subsection{Web Services}
The conference-scheduling service (currently Indico), Sharepoint, the DUNE Document Database, and the DUNE Wiki, as well as the main dunescience.org web page, are all hosted at \dword{fnal}.  There are also authentication and authorization facilities such as \dword{voms}.  \dword{fnal} manages the business relationship with CILogon.org to provide X.509 certificates and \dword{wlcg} \dword{jwt} tokens for batch authentication as well.  \dword{fnal} provides an electronic logbook service.  It also hosts the web sites for a number of monitoring services. 

\subsection{Database Services}
\dword{fnal} maintains the collaboration database, which tracks the membership of the DUNE collaboration.  It runs the \dword{ferry} 
\fixme{add def in glossary-comp-cdr}
database which tracks compute permissions for DUNE collaborators.  It hosts the underlying databases for the data management
services \dword{rucio} and \dword{metacat}, as well as the legacy \dword{sam} workflow management. It also hosts, or will host, the various conditions databases and beam conditions databases which are part of the DUNE Database activity.

% trj: added sentences about Redmine March 4, 2022

\subsection{Compute Services}
\dword{fnal} provides the Jobsub service for batch job submission, the \dword{poms} workflow service to submit campaigns, and the GlideinWMS
and HEPCloud services to access remote sites including high performance computing and commercial clouds. It provides system administration and hardware maintenance of the various interactive and batch clusters (FermiGrid).  It also provides
continuous integration facilities (Jenkins), and build service machines.  \dword{fnal}  hosts a code repository system called Redmine which provides support for version control repositories of types git, svn and cvs, as well as for wikis and issue tracking.  Redmine is mainly used for legacy software and documentation for DUNE software efforts, as many have migrated to GitHub.  \dword{fnal}  provides instances of code distribution services, namely \dword{cvmfs} and its associated Rapid Code Distribution Facility for user code.  Streaming of large auxiliary data files is accomplished using the Open Science Data Federation (until recently known as 
Stashcache), which also uses the \dword{cvmfs} user interface. 
The recently-established containerized Elastic Analysis Facility at \dword{fnal}  is available to DUNE.  The host lab also maintains
the monitoring and log retrieval services for the batch and storage systems, collectively known as \dword{fifemon}. 
\fixme{anne added dword, needs def.} The host lab is assisting DUNE and other experiments to transition to the Spack\footnote{Spack\copyright, \url{https://spack.io/}} code packaging system.  Significant effort is also provided on managing the \dword{larsoft} project.

\subsection{Storage Services}
The host lab provides archival tape storage, currently delivered through the \dword{enstore} tape library system.  The \dword{dcache} 
disk caching system is the front end to this system, although also containing some standalone disk.  The lab also provides a number of data management and data movement services, including Fermi-File Transfer Service, \dword{sam}, \dword{rucio}, \dword{metacat}, and soon to be the Data Dispatcher. \fixme{and soon to be?? For TRJ: below in todo, pls use 'the \dword{cern}' if first usage in chapter, and use dwords!}

\todo{trj:  Mention Jobsub, CVMFS, StashCache, electronic logbooks (host lab and CERN).  FIFEMON.  FERRY.  Computer security.  Support via SNOW. DUNE also uses SNOW internally to keep track of DUNE-specific requests, like MC samples. Wilson Cluster.  Experiment feeback to Liaisons and requests for feedback in change of services.  For example, when networks are reconfigured, experiments are asked for their input.  Service status web pages and announcements of scheduled maintenance and unscheduled disruptions, along with estimates of when systems will be available.  Regular maintenance such as Kernel upgrades to interactive and batch servers.  Migration of data from old media to new.  DAQ group services (software, documentation, assistance in running the DAQ).  Mention annual resource portfolio review.  dCache and NAS usage monitoring.  Build system upgrades, maintenance, documentation.  Spack migration.  LArSoft management is provided by SCD.}

\section{Remote Site Provided Services}
DUNE receives compute and storage resources from a number of sites around the world, as detailed in %anne the section on the Computing Contributions Board 
Section~\ref{sec:ccb}. In addition, DUNE also receives a few services that are hosted at \dword{cern}.  These include the EDMS document
management system and the MONIT monitoring system. Some DUNE meetings are organized using \dword{cern}'s Indico instance.  \dword{cern} also hosts the DUNE CRIC, which is the master list of all DUNE compute and storage resources as well as the means to track allocations, and the ETF testing service, which routinely tests all remote compute sites on DUNE's behalf. %anne is also hosted and operated at \dword{cern}.

%trj:  specified LaTeX for Overleaf March 2, and added Microsoft Teams.  Added a mention of CERN's indico instance

\section{Cloud-Hosted Services}
DUNE uses the GitHub service for its code management, the Slack and Microsoft Teams services for interactive communication, the Overleaf service for editing \LaTeX{} documents, and the Zoom teleconferencing application.  The cloud-hosted ServiceNow application is used for communication between DUNE liaisons and %anne the SCD facility 
\dword{fnal}'s Computing Divisions with respect to outages and changes to service and also for %anne internal DUNE use to 
tracking of issues and workflow requests such as production requests, data movement requests.
\todo{section on assumptions about external services that we depend on}
\end{document}