\documentclass[../main-v1.tex]{subfiles}
\begin{document}
\chapter{Services Overview \hideme{revisions for SK/Anne 5/8}}
\label{ch:serv}

%%%%%%%%%%%%%%%%%%%%%%%%%%%%%%%%
%\section{xyz}
%\label{sec:serv:xyz}  %% fix label according to section
\section{Introduction}
DUNE computing is dependent on a number of services that are not operated by DUNE itself.
Some of these are provided by the host laboratories (CERN and FNAL), such as experiment LAN's and primary raw data storage,  others are provided by the various remote sites where
computing and data storage are done, and still others are operated by commercial companies and hosted in the cloud.

% trj: added SharePoint -- March 2, 2022
%ST complete rewrite based on %trj comments, split up into sections, added a lot more services.
\section{Host Lab Provided Services}
DUNE relies on the host laboratories, \dword{fnal} and \dword{cern}, to provide a wide variety of central services.  These include web (for example, EDMS and docdb) and databases as well as cyber security. Compute and storage services are more widely distributed across the collaboration, see \ref{model:global} for more detail.  The networking services are described in \ref{ch:netw}.  We summarize the other services briefly.

\subsection{Web Services}
The conference-scheduling service (currently Indico), Sharepoint, the DUNE Document Database, and the DUNE Wiki, as well as the main {\tt dunescience.org} web page, are all hosted at \dword{fnal}.  
\dword{fnal} also hosts authentication and authorization facilities such as \dword{voms}, and manages the business relationship with CILogon.org to provide X.509 certificates and \dword{wlcg} \dword{jwt} tokens for batch authentication as well.  
\dword{fnal} and \dword{cern} both  provide  electronic logbook services and 
host the web sites for a number of monitoring services. 

\subsection{Database Services}
\dword{fnal} maintains the collaboration database, which tracks the membership of the DUNE collaboration, and runs the \dword{ferry} 
%\fixme{add def in glossary-comp-cdr}
database, which tracks compute permissions for DUNE collaborators.  \dword{fnal} hosts the underlying databases for the data management
services \dword{rucio} and \dword{metacat}, as well as the legacy \dword{sam} workflow management. %It also hosts, or will host, the various conditions databases and beam conditions databases that are part of the DUNE Database activity.

% trj: added sentences about Redmine March 4, 2022

\subsection{Compute Support Services}
In addition to the storage and computer services provided by collaborating institutions through their national resources, central coordination and distributions support is needed. 
\dword{fnal} provides the \dword{jobsub} service for batch job submission, the \dword{poms} workflow service to submit campaigns, and the \dword{glideinwms}
and \dword{hepcloud} services to access remote sites including high performance computing and commercial clouds. It provides system administration and hardware maintenance of the various interactive and batch clusters (FermiGrid).  It also provides
continuous integration facilities (Jenkins), and build service machines.  

%\dword{fnal}  hosts a legacy code repository system,   Redmine,  that provides support for version control repositories of types git, svn and cvs, as well as for wikis and issue tracking.  Redmine is mainly used for legacy software and documentation for DUNE software efforts, as many have migrated to GitHub. 

\dword{fnal} currently provides the main instance for several distribution services, namely \dword{cvmfs} and its associated Rapid Code Distribution Facility for user code.  Streaming of large auxiliary data files is accomplished using the Open Science Data Federation (until recently known as 
\dword{stashcache}), which also uses the \dword{cvmfs} user interface. 

%The recently-established containerized Elastic Analysis Facility at \dword{fnal}  is available to DUNE.  

\dword{fnal} also maintains
the monitoring and log retrieval services for the batch and storage systems, collectively known as \dword{fifemon}. 
%DONE%\fixme{anne added dword, needs def.} 

\dword{fnal} is assisting DUNE and other experiments to transition to the \dword{spack}\footnote{Spack\copyright, \url{https://spack.io/}} code packaging system.  Significant effort is also provided on managing the \dword{art} framework and \dword{larsoft} project.

\subsection{Storage Services}
\dword{fnal} and \dword{cern} provide the primary archival tape storage for raw data , currently delivered through the \dword{enstore} and \dword{cta}  tape library systems. A copy of reconstructed and simulated data is also currently maintained at \dword{fnal}. At \dword{fnal} the \dword{dcache} 
disk caching system is the front end to this system, and also provides some standalone disk.  \dword{fnal}, \dword{bnl} and \dword{cern} also provide a number of data management and data movement services, including CERN-FTS, Fermi-File Transfer Service, \dword{sam}, \dword{rucio}, \dword{metacat}, and the Data Dispatcher. 



\section{Collaboration Contributed Services}
DUNE receives compute and storage resources from a number of sites around the world, not all of which are formally DUNE collaborators,  as detailed in the section on the Computing Contributions Board 
Section~\ref{sec:ccb}.

\dword{cern} provided services include: the CERN \dword{indico},  the \dword{edms} document
management system, the MONIT monitoring system,  \dword{cric} - which is the master list of all DUNE compute and storage resources as well as the means to track allocations,  and the \dword{etf} testing service which routinely tests all remote compute sites on DUNE's behalf. %anne is also hosted and operated at \dword{cern}.

While \dword{cern} and \dword{fnal} remain the major providers of support services, some monitoring and control activities are moving to collaborating institutions, notably the \dword{rucio} monitoring, now hosted by Edinburg and the new workload system at Manchester.  We anticipate a move to an even more distributed system as the experiment evolves. 
%trj:  specified LaTeX for Overleaf March 2, and added Microsoft Teams.  Added a mention of CERN's indico instance

\section{Cloud-Hosted Services}
DUNE uses the GitHub service for its code management, the Slack and Microsoft Teams services for interactive communication, the Overleaf service for editing \LaTeX{} documents, and the Zoom teleconferencing application.  The cloud-hosted \dword{servicenow} application is used for communication between DUNE liaisons and the two
\dword{fnal} Computing Divisions about outages and changes to service and also for internal DUNE use in 
tracking of issues and workflow requests such as production requests and data movement requests.
\end{document}