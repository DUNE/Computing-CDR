\documentclass[../main-v1.tex]{subfiles}
\begin{document}
\chapter{Authentication and Authorization for Distributed Computing\hideme{Timm - draft}}
\label{ch:auth}

%%%%%%%%%%%%%%%%%%%%%%%%%%%%%%%%
%ST changed section title to the chapter title and promoted all subsections to sections
%\section{Authentication and Authorization for Distributed Computing \hideme{Timm - draft}}
%\label{sec:auth:xyz}  %% fix label according to section
\section{Obtaining Access to DUNE Computing}
\label{sec:auth:access}
%ST reworked this whole paragraph
Collaborators on DUNE must first be added to the DUNE Collaboration Database by the procedures of joining the collaboration.
Once this is done, they may apply for access to computing at Fermilab, and the DUNE secretariat verifies that they
are members of DUNE.  There is a central repository called FERRY which is used to populate the user list of
Fermilab interactive login machines and who can obtain Grid X.509 credentials.
 When the Fermilab ID of the collaborator is expired, or the person leaves the DUNE collaboration, the access to computing is automatically removed.
 
 During the time when the ProtoDUNE experiments NP02 and NP04 are running at the CERN Neutrino Platform, DUNE collaborators who are members of those activities may also apply to CERN for interactive computer access at CERN.

\section{Current State of Authentication and Authorization}
%ST
Interactive login to Fermilab is strongly authenticated via Kerberos 5 credentials.  Only currently registered DUNE
users have accounts at Fermilab.

Authentication for batch submission and access to storage is currenlty done via X.509 certificates and proxies
on the grids of which we are members, namely the Open Science Grid (OSG) in the USA and the Worldwide LHC 
Computing Grid (WLCG) elsewhere in the world.  There are automated means at Fermilab which dynamically generate 
X.509 certificates on behalf of the user at the time of batch submission, from the CILogon-Silver 
certificate authority. This authentication is used to submit batch jobs both to the local
Fermilab clusters known as FermiGrid and to the rest of the distributed computing sites.  The proxies are
also used to access files on storage on the global storage elements around the world.  
X.509 certificates are issued by members of the International Global Trust Federation (IGTF).  They are then verified 
by the DUNE VOMS server.  Certificates issued by other IGTF authorities can be associated with a user by request.

Access to DUNE web sites hosted at Fermilab is controlled by the Fermilab Single Sign-on facility.  Some sites (https://wiki.dunescience.org) require a Fermilab "Services" credential. Some sites including the DUNE DocDB (https://docs.dunescience.org) will allow the user to present a CERN credential  or an X.509 certificate in addition to the Fermilab "Services" login.


\section{Planned Changes to Authentication Currently Under Way}

The OSG and WLCG are both in the process of changing over to JSON Web Tokens for authentication of
batch jobs and storage. This is an authentication protocol which has significant adoption in industry. A common schema has been agreed on between the various organizations that will be issuing these tokens.  T
Fermilab has set up a token-issuing service to issue these tokens on behalf of DUNE and early tests have been 
done to successfully access storage elements with these tokens. There have also been successful tests of batch job submission. The token issuer is populated with information 
from the FERRY server mentioned above. The tokens include a unique identifier to 
know which user is doing it, and a list of capabilities that the user is allowed to have, including which 
areas in the storage element the user is allowed to write.  The basic token infrastructure is available now but we expect a phased transition beginning in 2022, in parallel with the X.509 
based authentication infrastructure for some number of years.

Interactively-based web sites will continue to use Fermilab Single Sign-on (but allowing more Identity Providers)
for the foreseeable future.  It is likely that JWT tokens will be used to access various non-interactive web
services besides the compute and storage. 
 

\section{Requirements for Authentication/Authorization Going Forward}

DUNE Computing Management has expressed to Fermilab the following requirements:
We need a way for DUNE collaborators who may not yet have Fermilab credentials to see and edit internal DUNE 
web pages.  We also need a way for computing staff who work at institutions where we have access to the 
which are not themselves DUNE collaborators to have access to the computing documentation if needed.
There are a number of technical solutions for federated web identity including the Edugain federation
of identity providers.  

This request has begun to be implemented, with the allowed access of CERN credentials to DocDB being the first instance.

DUNE also supports the statement of computing and data access voted by the International Union of Pure and Applied Physics

Physics increasingly involves international collaborations that share in the planning, construction and operation of common apparatus and in the analysis of the resulting data. Such collaboration is most successful when all collaborating scientists have unimpeded access to both physical facilities and to the data that results from their common efforts.  We observe a disturbing international trend of restriction of access to facilities,  shared data and methods of communication.  While nations have legitimate needs to protect their critical facilities  and computing systems, they also have a responsibility to find ways to maintain and facilitate access for all international scientific collaborators.

\end{document}




