\documentclass[../main-v1.tex]{subfiles}
\begin{document}
\chapter{Authentication and Authorization for Distributed Computing\hideme{Timm - draft}}
\label{ch:auth}

%%%%%%%%%%%%%%%%%%%%%%%%%%%%%%%%
%ST changed section title to the chapter title and promoted all subsections to sections
%\section{Authentication and Authorization for Distributed Computing \hideme{Timm - draft}}
%\label{sec:auth:xyz}  %% fix label according to section
\section{Obtaining Access to DUNE Computing}
\label{sec:auth:access}
%ST reworked this whole paragraph
Collaborators on DUNE must first be added to the DUNE Collaboration Database by the procedure in place for joining the collaboration.
Once this is done, they may apply for access to computing at \dword{fnal}, and the DUNE secretariat verifies that they
are members of DUNE.  %There is a 
The central repository %called 
\dword{ferry} %that 
is used to populate the user list for the 
\dword{fnal} interactive login machines and to determine if a user can obtain DUNE Grid X.509 credentials.
 When a collaborator's \dword{fnal} ID expires, or they leave the DUNE collaboration, their access to computing is automatically removed.
 
 %During the time when 
 While the \dword{protodune} experiments \dword{np02} and \dword{np04} are running at the \dword{cern}, %CERN Neutrino Platform, DUNE 
 collaborators %who are members of 
 working on those activities may  apply %to CERN 
 for interactive computer access at \dword{cern}, as well.

\section{Current State of Authentication and Authorization}
%ST
Interactive login to \dword{fnal} is strongly authenticated via Kerberos 5 credentials.  %Only currently registered DUNE users 
DUNE collaborators must be currently registered to have accounts at \dword{fnal}.

Authentication for batch submission and access to storage is currenlty done via X.509 certificates and proxies
on the %grids of which we are members, namely the Open Science Grid (OSG) 
\dword{osg} in the U.S. and \dword{wlcg} % the Worldwide LHC 
Computing Grid (\dword{wlcg}) elsewhere in the world.  %There are a
Automated processes %means 
at \dword{fnal} %which 
dynamically generate 
X.509 certificates on behalf of the user at the time of batch submission, from the CILogon-Silver 
certificate authority\footnote{\url{https://www.cilogon.org/news/cilogonsilverenablesfederatedaccesstoopensciencegrid}}. This authentication is used to submit batch jobs both to the local
\dword{fnal} clusters known as FermiGrid and to the rest of the distributed computing sites.  The proxies are
also used to access files %on storage 
stored on the global storage elements around the world.  
X.509 certificates are issued by members of the International Global Trust Federation (IGTF).  They are then verified 
by the DUNE \dword{voms} server.  Certificates issued by other IGTF authorities can be associated with a user by request.

Access to DUNE web sites hosted at \dword{fnal} is controlled by the \dword{fnal} Single Sign-on (SSO) facility.  Some sites (e.g., the DUNE wiki) %https://wiki.dunescience.org) 
require a \dword{fnal} SSO.
%a Fermilab "Services" credential
 Some sites, e.g., % including the DUNE DocDB 
document database (DocDB), require either a \dword{fnal} or \dword{cern} SSO or  %(https://docs.dunescience.org) will allow the user to present a CERN credential  or 
an X.509 certificate. % in addition to the Fermilab "Services" login.
Both of these applications also currently allow access via a group username and password.


\section{Planned Changes to Authentication Currently Under Way}

The \dword{osg} and \dword{wlcg} are both in the process of changing over to \dwords{jwt} %JSON Web Tokens 
for authentication of batch jobs and storage. This %is an 
authentication protocol %which 
has significant adoption in industry. A common schema has been agreed on between the various issuing organizations. % that will be issuing these tokens.  
\dword{fnal} has set up a %token-issuing 
service to issue these tokens on behalf of DUNE and early tests have %been done to 
successfully demonstrated access to storage elements and batch job submission with these tokens. %There have also been successful tests of batch job submission. 
The token issuer is populated with information from the \dword{ferry} server. % mentioned above. 
The tokens include a unique user identifier %to know which user is doing it, 
and a list of capabilities that the user is allowed to have, including which 
areas in the storage element the user is allowed to write.  The basic token infrastructure is available now but we expect a phased transition beginning in 2022, during which %in parallel with 
the X.509-based authentication infrastructure will continue to be available for some number of years.

%Interactively-based 
Interactive web sites will continue to use %Fermilab Single Sign-on 
\dword{fnal} SSO 
for the foreseeable future, while gradually allowing more Identity Providers. 
It is likely that \dword{jwt} tokens will be used to access various non-interactive web
services besides the compute and storage. 
 

\section{Requirements for Authentication and Authorization Going Forward}

DUNE supports the statement of computing and data access voted on by the International Union of Pure and Applied Physics:

\textit{Physics increasingly involves international collaborations that share in the planning, construction and operation of common apparatus and in the analysis of the resulting data. Such collaboration is most successful when all collaborating scientists have unimpeded access to both physical facilities and to the data that results from their common efforts.  We observe a disturbing international trend of restriction of access to facilities,  shared data and methods of communication.  While nations have legitimate needs to protect their critical facilities  and computing systems, they also have a responsibility to find ways to maintain and facilitate access for all international scientific collaborators.}

To this end DUNE Computing Management has %anne expressed 
submitted to \dword{fnal} the following requirements:
%We need a way for DUNE collaborators who may not yet have Fermilab credentials to see and edit internal DUNE web pages.  We also need a way for computing staff who work at institutions where we have access to the which are not themselves DUNE collaborators to have access to the computing documentation if needed.There are a number of technical solutions for federated web identity including the Edugain federationof identity providers.  

\begin{itemize}
    \item DUNE collaborators without a \dword{fnal} SSO must be able to see and edit internal DUNE web pages.  
\item Computing staff who work at collaborating institutions but who are not themselves DUNE collaborators must be allowed access to the computing documentation.
 
\end{itemize}
There are a number of technical solutions for federated web identity, including the Edugain federation of identity providers. 

%This request has begun to be implemented, with the allowed access of CERN credentials to DocDB being the first instance.
\dword{fnal} is responding, and as of this writing, users with \dword{cern} credentials can now use those credentials to access the DUNE DocDB. \dword{fnal} is in the process of federating its SharePoint instances with \dword{cern}, and the DUNE wiki and Indico sites are next in line.


\end{document}
