\documentclass[../main-v1.tex]{subfiles}
\begin{document}
\chapter{Authentication/Authorization \hideme{Timm - draft}}
\label{ch:auth}

%%%%%%%%%%%%%%%%%%%%%%%%%%%%%%%%
\section{Authentication and Authorization for Distributed Computing \hideme{Timm - draft}}
%\label{sec:auth:xyz}  %% fix label according to section
\subsection{Obtaining access to DUNE Computing}

DUNE collaborators obtain access to computing currently by applying for a computing account at Fermilab.  
The DUNE secretariat then verifies that they are members of DUNE.  There is a central repository called FERRY
which is used to determine who can log into Fermilab interactive resources and also who can obtain Grid X.509
credentials.  When the Fermilab ID of the collaborator is expired, the access is automatically removed.

\subsection{Current State of Authentication and Authorization}

Interactive login to Fermilab is authenticated via Kerberos 5 credentials.  Only currently registered DUNE
users have accounts.

Authentication for batch submission and access to storage is done via X.509 certificates and proxies
on the grids of which we are members, namely the Open Science Grid (OSG) in the USA and the Worldwide LHC 
Computing Grid (WLCG) elsewhere in the world.  There are automated means at Fermilab which dynamically generate 
X.509 certificates on behalf of the user at the time of batch submission, from the CILogon-Silver 
certificate authority. This authentication is used to submit batch jobs both to the local
Fermilab clusters known as FermiGrid and to the rest of the distributed computing sites.  The proxies are
also used to access files on storage on the global storage elements around the world.  
X.509 certificates are issued by members of the International Global Trust Federation (IGTF).  They are then verified 
by the DUNE VOMS server.

Access to internal DUNE web sites can be done either by the certificates mentioned above, or by the Fermilab 
Single Sign On.  In either case new collaborators have to request access independently to these resources
when first joining DUNE, and at the moment a Fermilab computing account is required.  Some DUNE meeting web sites
are protected by password keys.

\subsection{Planned Changes to Authentication Currently Under Way}

The OSG and WLCG are both in the process of changing over to JSON Web Tokens for authentication of
batch jobs and storage. This is an authentication protocol which has significant adoption in industry.
Fermilab has set up a token-issuing service to issue these tokens on behalf of DUNE and early tests have been 
done to successfully access storage elements with these tokens.  The token issuer is populated with information 
from the FERRY server mentioned above. The tokens include a unique identifier to 
know which user is doing it, and a list of capabilities that the user is allowed to have, including which 
areas in the storage element the user is allowed to write.  The token infrastructure is supposed to 
be fully deployed by the end of calendar year 2021, but will probably run in parallel with the X.509 
based authentication infrastructure for some number of years.

It is not as yet clear if these JWT tokens will also be used for web access going forward.  It is likely 
that web access will continue to be via the Fermilab Single Sign-on for the foreseeable future.  

\subsection{Requirements for Authentication/Authorization Going Forward}

DUNE Computing Management has expressed to Fermilab the following requirements:
We need a way for DUNE collaborators who may not yet have Fermilab credentials to see and edit internal DUNE 
web pages.  We also need a way for computing staff who work at institutions where we have access to the 
which are not themselves DUNE collaborators to have access to the computing documentation if needed.
There are a number of technical solutions for federated web identity including the Edugain federation
of identity providers.  

\end{document}




