\documentclass[../main-v1.tex]{subfiles}
\begin{document}
\chapter{Training and Documentation \hideme{David,  Kirby - draft}}
\label{ch:train}
%%%%%%%%%%%%%%%%%%%%%%%%%%%%%%%%
%\section{xyz}
%\label{sec:train:xyz}  %% fix label according to section
DUNE is a rapidly growing collaboration from almost 200 institutions. The scale of the experiment demands a harmonized documentation effort, coupled with consistent training for both newcomers and existing users. Although treated separately in this section, the tasks of documentation and training are highly correlated. Both are crucial for the long term success of DUNE. 

\section{Documentation}
The documentation related to the DUNE’s computing aspects will be accessible on a variety of platforms, each with specific goals and access policies.

\subsection{Wikis}
The existing DUNE wiki acts as the landing page and starting point for ``DUNE Computing." It acts as a portal gathering information and links about the DUNE Computing consortium groups, its activities and related resources. The template has been revamped in early 2021 with a ‘block’ design for a more visual and compact layout. As with the rest of the DUNE wiki, its access is restricted; users need a FNAL Single Sign-On (SSO) service domain since the wiki contains information about computing access, node names, and other sensitive information.

A separate publicly accessible wiki for DUNE Computing is in preparation. The motivations for a dedicated public page are to make the information available for associated collaborations (e.g. OSG, WLCG, HSF), have a point of contact for newcomers not yet registered, and show the DUNE Computing activities to the outside world in general. This public portal has six main blocks: management, operation, getting started, user area, developer area, documentation. The management block informs about the consortium, the calendar of meetings, and list the associated collaborations. The operation block redirects to monitoring pages essential during data taking and/or production (FIFEmon tools). The block ‘getting started’ will have pointers on the tutorials, training sessions and how-tos. The block 'user area' collects links about analysis tools, framework, and helpdesk. The DUNE Computing working groups are listed on the developer area, and conveners of each working group will maintain their page. Last but not least, the documentation block gathers links about data policies and preservation, information protection, and the documents in progress.

Critical information such as login commands or computer node names will not be visible on the public wiki. Instead, the public wiki will provide links pointing to an access-restricted page on DUNE’s main wiki where details can be safely provided. Only users with approved DUNE credentials will be able to access them. The public wiki will not duplicate the information but rather offer a summary to the outside world, benefiting visitors from founding agencies and future DUNE’s newcomers, not to mention the potential to foster collaborations with scientists from other experiments.

\subsection{Redmine}
The current primary repository for DUNE Far Detector and ProtoDUNE software is the dunetpc repository that is accessed through the FNAL Redmine service. Along with the code repository, there is issue tracking and a wiki embedded within the service. This service will likely be retired within the 2021 calendar year and therefore work is on going to transition away from the Redmine service. The contents of the wiki will be transferred to the access-restricted DUNE wiki discussed previously, while the issue tracking will be transitioned to Github.

\subsection{Code Documentation}
One of the challenges for any large experiment is documentation of the algorithms and software. Following the lead of the Belle2 experiment, DUNE plans to explore the implementation of the Sphynx\footnote{https://www.sphinx-doc.org/en/master/} package for the automatic generation of code documentation. The generation of in-code comments and dedicated text files utilized by Sphynx will be the responsibility of code librarians and contributors of each sub-repository within the dunetpc code stack.

% trj: update to say we have migrated.  though the wikis and tickets have not yet.

\subsection{GitHub}
DUNE software repositories have migrated to the version-controlled platform GitHub.  GitHub offers a ticketing system facilitating debugging, revisions and updates from the community of users, though DUNE does not yet use GitHub's wikis or ticketing system extensively.  The {\it art} and LArSoft teams have transitions to using GitHub's ticketing system from the Redmine system.  The work to make this transition for DUNE is underway and is following the guidance of the HSF community. Issues of access, privacy, and de-authorization of accounts are currently under investigation and it has not yet been determined what the future policy shall be.

\todo{trj:  Are the issues of access and privacy related to code or documentation?  We want our code to be publicly readable, although we have not settled on a final write policy. Currently we let anyone write, but LArSoft locks that down to a set of managers, for example.  But this is a chapter on documentation, not code.}

\section{Training \hideme{David, DeMuth, Kirby - draft}}

\subsection{Goals of DUNE Training}
DUNE’s training aims at serving both newcomers and existing users, offering a smooth start for the former and continuous support for the latter. DUNE recognizes that the computing environment and tools utilized within an HEP experiment are unique and require specialized training. The goals are to teach the basics of the environment and software used for analysis, as well as best practices in programming and data management. The training is offered through various formats and platforms, as well as tools and collaborations, all discussed below. 

% trj: tweak the "several" clause to have fewer hyphens March 2

\subsection{Training sessions}
The primary training for new DUNE members is done twice a year during  dedicated sessions that extend over several days. These sessions are coordinated with the collaboration week, occurring alternatively before or after. Such timing secures a good attendance for both newcomers and lecturing experts.
The format of the training is an alternation of lectures and  hands-on sessions. Introductory instructions and preliminary homework are sent to participants before the tutorials to ensure the trainees arrive with access to computing resources, one of the primary hurdles to learning about and using new computing environments.

%trj: improved sentence about retaining old tutorial material

General topics such as DUNE computing basics and grid job submission are covered each year. However, due to the vast suite of software used by the DUNE collaboration for the various analysis steps and sub-detectors, each event will highlight a particular software package or analysis toolkit.  All materials and documentation from past tutorials are retained and prominently linked on the DUNE wiki to serve as self-education material, providing an asynchronous training for newcomers having missed a previous session.

\subsection{Training tools}
The past sessions have used the wiki as main support for the training material. The lectures proceed through a dedicated wiki page containing informative sections, sets of commands to run and numerous links pointing to further reading.

The DUNE Computing training group plans to create a dedicated FAQ platform. It would be similar to  Askbot\footnote{https://askbot.org}, an open source software offering user support with a question database whose answers are rated by relevance and popularity (inspired by the source-based websites Reddit and Stackoverflow). This choice is motivated by the positive feedback of the training organized in January 2021. Askbot is no longer available, so for now live documents (hosted on GoogleDoc) were shared to welcome questions from the participants. The anonymity is a key feature. Participants could ask anonymously, a key feature as many young members are usually not daring to ask. Experts then provide written answers. The numerous questions with their answers are later added to the tutorial wiki pages. Following this success, a dedicated FAQ instance would be an efficient, crowd-based, and self-maintained platform to eventually collect the most frequently asked questions, while offering the most relevant answers or solutions.

In the future, DUNE has the goal that training events will utilize Jupyter notebooks, and therefore provide an interactive environment for the trainees to run the examples, do exercises, and tweak portions of code to deepen their understanding of the analysis software. These notebooks will be accessible through a JupyterHub server provided by FNAL and CERN, as well as other hosting institutions equipped with computing resources and analysis facility tools. The notebooks will be archived and referenced on the wiki as self-training modules for newcomers as well as inspiration material for young lecturers.

Lectures will be recorded and the link of the video will be posted on the wiki or a dedicated platform. The training sessions will thus be available asynchronously at any time.

\subsection{Partnering with other collaborations}
The HEP Software Foundation (HSF) has started a continuous effort of harmonizing their tutorials under a common template called “training module.” The HSF aims at creating an introductory curriculum giving HEP newcomers the basic set of software needed while instilling good programming practices right at the start. Several of their modules for beginners are offered by the Software Carpentry Foundation\footnote{https://software-carpentry.org }. The growing list of modules can be found on the HSF website “Towards a HEP Software Training curriculum"~\footnote{https://hepsoftwarefoundation.org/training/curriculum.html}.

The DUNE Computing training group is working in collaboration with the HSF. The DUNE training is pointing newcomers to HSF's curriculum page so that they can learn on the prerequisites before attending an official DUNE training. This will give the HSF an audience to their self-serve training centre, which in return provide valuable feedback and thus the opportunity to improve and fine-tune their HSF tutorials. In the middle term, the neutrino specifics will be introduced for HSF and bring a diversity in their curriculum.

\subsection{From trainee to mentor, user to lecturer}
A key aspect of the success of the HSF workshops resides in its mentoring system. Participants are split in small groups with one mentor to assist the students. The mentors are young researchers and alumni of the previous training session. This facilitates communication as newcomers are less reluctant to ask questions. After a couple of training sessions, the mentors can be trained by experts to become lecturers, acquiring new skills with their favorite software. The DUNE Computing consortium plans to reproduce the HSF mentoring and training scheme with frequent calls for mentors within the collaboration.

\subsection{Future formats}
The DUNE Computing training group is also committed in surveying the needs and various demands for particular aspects and skills the members would like to learn. Recent surveys have shown that LArSoft and Pandora are popular topics, that there is a general interest to know the ‘best practices’ for building analysis code, not to mention learning about machine learning techniques (Keras, Tensorflow, GPU libraries). The breadth of the tutorial wish-list is too large to efficiently cover during one bi-annual DUNE training, but continued input from tutorial attendees allow the planning of future topic specific lectures and tutorials. Moreover, incoming students are each working on a specific aspect of the DUNE project. A project-based approach such as a hackathon would be an ideal format for the training, where students would team up on an aspect directly linked with their analysis. They “go back home” with a program directly needed for their research. The presence of an expert to give them early guidance would be greatly beneficial as it would instill good practices while building the student confidence as a future analyzer. An experimental hackathon training will be organized in 2021 to test this format for the coming years.



%Training Considerations (brainstorming w/ myself to verify Overleaf privs - DeMuth)
%\begin{itemize}
%    \item Introduction to High Performance Computing
%    \item \LaTeX ~documentation, tables, and figures
%    \item GitHub Usage
%    \item DUNE Collaboration DB
%    \item Shifters
%\end{itemize}

\end{document}