\chapter{Example Chapter}
\label{ch:chap-id}

%%%%%%%%%%%%%%%%%%%%%%%%%%%%%%%
\section{Introduction}
\label{sec:chap-id:intro}

Sections and subsections should have labels for cross-referencing. Labels must be unique. See the suggested format.

See Figure~\ref{fig:required-label}. Notice that only the first word is capitalized in both the short and full captions.  

%%%%%%%%%%%%%%%%%%
\subsection{Dwords and the Glossary}
\label{sec:chap-id:intro}

The dwords are defined in common/glossary.tex. You use \verb|\dword{}| the same way, whether the term is defined as a ``newduneword'' or a ``newduneabbrev.''  A term defined as an abbreviation will show the full term the first time it's used in a chapter and just the abbreviation thereafter.  For instance:

As significant as this information is, it is the \dword{lsb}. In fact, compared to many things it is the \dword{lsb}.

You can also use \verb|\dshort{}| which doesn't hyperlink, but is useful in captions and headings to use the standardized rendering of a term.

%%%%%%%%%%%%%%%%%%
\subsection{Figures}
\label{sec:chap-id:intro}


Figure~\ref{fig:required-label} is the logo for \dword{dune}. 

\begin{dunefigure}[Optional short caption for LoF]
{fig:required-label}
{Required full caption. Don't capitalize every word.}
\includegraphics[width=0.8\textwidth]{dunelogo_colorhoriz}
\end{dunefigure}

%%%%%%%%%%%%%%%%%%%%%%%%%%%%%%%
\section{My Amazing Widget}
\label{sec:chap-id:mywidget}

The string of percent signs just makes it easier to spot where new sections start.

Notice that all the main words in headings are capitalized.

Let's add a reference. I'm sure that this reference is useful somewhere, but not here~\cite{Acciarri:2016sli}.

Now let's add a ``dunetable.'' See Table~\ref{tab:table-label}.

\begin{dunetable}
[The LoT caption]
{cc}
{tab:table-label}
{The full caption that appears above the table.}
Rows & Counts \\ \toprowrule
Row 1 & First \\ \colhline
Row 2 & Second \\ \colhline
Row 3 & Third \\ % no \colhline on final row
\end{dunetable}

%%%%%%%%%%%%%%%%
\subsection{Numbers for my Widget}
\label{sec:chap-id:mywidget:num}

The following shows (1) how to do a bulleted list (notice the commas, the ''and'', and the period.  It also shows how to write different kinds of numbers.

\begin{itemize}
    \item 100 is written as \num{100},
    \item 1000 is written as \num{1000},
    \item 123.456 is written as \num{123.456},
    \item 1 plus or minus 2i is written as \num{1+-2i},
    \item 3 times 10 to the 45th is written as \num{3e45},
    \item 0.3 times 10 to the 45th is written as \num{.3e45} (keeps the decimal point before the 3), and 
    \item ''10, 20 and 30'' is written as \numlist{10;20;30}.
    \item 
\end{itemize}

%%%%%%%%%%%%%%%%
\subsection{Numbers with Units for my Widget}
\label{sec:chap-id:mywidget:numunit}

Here's how to do a numbered list and write numbers with units. 
\begin{enumerate}
    \item 120 GeV is written as \SI{120}{\GeV} or (more simply)  120\,GeV, and
    \item 4850 feet is written as \SI{4850}{\ft}.
\end{enumerate}

These and many others are defined in the file common/units.tex.

%%%%%%%%%%%%%%%%%%%%%%%%%%%%%%%
\section{My Second Amazing Widget}
\label{sec:chap-id:my2ndwidget}

If you have one section, you need at least two; same goes for subsections, etc. 

%%%%%%%%%%%%%%%%%%%%%%%%%%%%%%%
\section{More Information}
\label{sec:chap-id:moreinfo}

First, see Section~\ref{sec:chap-id:my2ndwidget} just to see how we do cross-sectional references.  The same goes for cross-chapter references, you just need to find the correct label for the chapter (or the section within the chapter).

More information is at \url{https://dune.bnl.gov/docs/guidance.pdf}.
