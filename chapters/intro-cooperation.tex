\documentclass[../main-00.tex]{subfiles}
\begin{document}
\chapter{Cooperation strategy \hideme{Kirby, Schellman, McNab}}
\label{ch:coop}

%%%%%%%%%%%%%%%%%%%%%%%%%%%%%%%%
%\section{xyz}
%\label{sec:coop:xyz}  %% fix label according to section

Describe the cooperation strategy, connection as stakeholder with:
\subsection{overview of strategy}
\subsection{Relation with Fermilab, CERN, Sanford}
\subsection{HSF \hideme{Laycock, David}}
\subsection{Training \hideme{ C. David}}
\subsection{WLCG/OSG \hideme{ McNab and Kirby}}
\subsection{Benchmarking \hideme{ McNab}}
\subsection{Rucio \hideme{Timm}}
\subsection{LArSoft \hideme{ Tom/Kirby}}
The \dword{larsoft} software toolkit is supported by Fermilab's \dword{scd}.  It consists largely of software and associated configuration and parametric data files contributed by the stakeholder experiments.  These are \dword{argoneut}, \dword{lariat}, \dword{microboone}, \dword{icarus}, \dword{sbnd}, and \dword{dune}.  \dword{larsoft} is based on the \dword{art} framework.  It is designed to maximize reuse of software components, which take significant effort to develop, optimize, maintain and support.  \dword{larsoft} is currently hosted on \dword{github}.  Code is contributed via a pull-request model, and members of contributing collaborations provide the personpower to review and approve code modifications.  Release support and package management is supplied by Fermilab's \dword{scd}.  Testing duties are shared between the contributing experiments and the \dword{larsoft} team.  

Bi-weekly \dword{larsoft} coordination meetings are scheduled for members of contributing collaborations to present proposed changes to \dword{larsoft}, and to provide news from the management team to the stakeholders.  Annual \dword{larsoft} steering group meetings are scheduled to review the \dword{larsoft} team's work plans for the year.  The spokespersons of the stakeholder collaborations are invited to attend and provide feedback.  Quarterly \dword{larsoft} offline leads meetings gather together software and computing experts from the stakeholder experiments to discuss work plans and issues arising from changes to the software and its management procedures.

\subsection{Art / ... \hideme{ Tom} }

The \dword{art} framework is developed, maintained, and supported by Fermilab's \dword{scd}.  In addition to the \dword{larsoft} stakeholder collaborations, which are also \dword{art} stakeholders, the \dword{nova} and \dword{mu2e} collaborations also use \dword{art}.  Unlike \dword{larsoft}, code is not contributed by stakeholders, but rather is developed by the \dword{art} team.  Bi-weekly \dword{art} stakeholder meetings are held where news of upcoming \dword{art} features, breaking changes, and bug-fixes is disseminated, as well as upcoming release plans and discussions regarding supported platforms.  Members of stakeholder collaborations report concerns that affect their use of \dword{art}.  Representatives of stakeholder collaborations are asked to approve changes to \dword{art}'s behavior that could impact experiment workflows.  Feature requests from one experiment can sometimes cause issues with other experiments' workflows, and changes to \dword{art} have only been made when consensus is reached on an appropriate course of action.

\subsection{Networking \hideme{Kirby}}
\subsection{Authentication \hideme{ Timm}}


\todo{ For each of the above, lay out current plan for engagement and resources needed}

\end{document}