\documentclass{dune} % Duke of English - Kirby

\input{common/preamble}
%\renewcommand\thedoctitle{\voltitletc} % defined in common/defs.tex
%\newcommand\thevolumenumber{\volnumbertc} 

\begin{document}

\pagestyle{titlepage}
%\includepdf[pages={-}]{name of cover file.pdf}
\cleardoublepage

% use this for things you want to only show up in some cases
%\newcommand{\hideme}[1]{{\it( #1)}}  moved to defs
% comment next line out to see assigned authors
%\renewcommand{\hideme}[1]{}
% This should be \input first thing after \begin{document}



\pagestyle{titlepage}

\begin{center}
   {\Huge  DUNE Computing Consortium}  %Yes, I know title and subtitle are reversed!

  \vspace{5mm}

  {\Huge  Conceptual Design Report}  

  \vspace{10mm}

 

\titleextra  %--- add back in if you want a picture here

\includegraphics[width=\textwidth]{graphics/dune-firstVersionNotBad-show.jpg}

  \vspace{10mm}
  \today
    \vspace{15mm}
    
    {\large{The DUNE Collaboration}}
\end{center}

\cleardoublepage
\vspace*{16cm} 
  {\small  This document was prepared by the DUNE collaboration using the resources of the Fermi National Accelerator Laboratory (Fermilab), a U.S. Department of Energy, Office of Science, HEP User Facility. Fermilab is managed by Fermi Research Alliance, LLC (FRA), acting under Contract No. DE-AC02-07CH11359.
  
The DUNE collaboration also acknowledges the international, national, and regional funding agencies supporting the institutions who have contributed to completing this Conceptual Design Report.  
  }
%\includepdf[pages={-}]{tdr-authors.pdf}              add back in later    



\renewcommand{\familydefault}{\sfdefault}
\renewcommand{\thepage}{\roman{page}}
\setcounter{page}{0}

\pagestyle{plain} 


\textsf{\tableofcontents}


\textsf{\listoffigures}

\textsf{\listoftables}
  \vspace{4mm}


\iffinal\else
\textsf{\listoftodos}
\clearpage
\fi

\renewcommand{\thepage}{\arabic{page}}
\setcounter{page}{1}

\pagestyle{fancy}

% Set how header/footers look
\renewcommand{\chaptermark}[1]{%
\markboth{Chapter \thechapter:\ #1}{}}
\fancyhead{}
\fancyhead[RO,L]{\textsf{\footnotesize \thechapter--\thepage}}
\fancyhead[LO,R]{\textsf{\footnotesize \leftmark}}

\fancyfoot{}
\fancyfoot[RO]{\textsf{\footnotesize Conceptual Design Report}}
\fancyfoot[LO]{\textsf{\footnotesize DUNE Computing Consortium}}
\fancypagestyle{plain}{}

\renewcommand{\headrule}{\vspace{-4mm}\color[gray]{0.5}{\rule{\headwidth}{0.5pt}}}

% Not all main documents have any citations.
% When not built in "final" mode, add in one citation just to let the
% document build.
% If, after substantial editing a main document still lacks any
% citations then it should have its whole bibliography removed.
%\ifdefined\isfinal\nocite{}\else\nocite{CD0}\fi
%\nocite{CD0} % REmoved 12/30/19


% see also preamble.tex
%\input{common/acronyms}

\cleardoublepage

% comment these lines out once we no longer need the example
% \setcounter{chapter}{-1}
% \chapter{Example Chapter}
\label{ch:chap-id}

%%%%%%%%%%%%%%%%%%%%%%%%%%%%%%%
\section{Introduction}
\label{sec:chap-id:intro}

Sections and subsections should have labels for cross-referencing. Labels must be unique. See the suggested format.

See Figure~\ref{fig:required-label}. Notice that only the first word is capitalized in both the short and full captions.  

%%%%%%%%%%%%%%%%%%
\subsection{Dwords and the Glossary}
\label{sec:chap-id:intro}

The dwords are defined in common/glossary.tex. You use \verb|\dword{}| the same way, whether the term is defined as a ``newduneword'' or a ``newduneabbrev.''  A term defined as an abbreviation will show the full term the first time it's used in a chapter and just the abbreviation thereafter.  For instance:

As significant as this information is, it is the \dword{lsb}. In fact, compared to many things it is the \dword{lsb}.

You can also use \verb|\dshort{}| which doesn't hyperlink, but is useful in captions and headings to use the standardized rendering of a term.

%%%%%%%%%%%%%%%%%%
\subsection{Figures}
\label{sec:chap-id:intro}


Figure~\ref{fig:required-label} is the logo for \dword{dune}. 

\begin{dunefigure}[Optional short caption for LoF]
{fig:required-label}
{Required full caption. Don't capitalize every word.}
\includegraphics[width=0.8\textwidth]{dunelogo_colorhoriz}
\end{dunefigure}

%%%%%%%%%%%%%%%%%%%%%%%%%%%%%%%
\section{My Amazing Widget}
\label{sec:chap-id:mywidget}

The string of percent signs just makes it easier to spot where new sections start.

Notice that all the main words in headings are capitalized.

Let's add a reference. I'm sure that this reference is useful somewhere, but not here~\cite{Acciarri:2016sli}.

Now let's add a ``dunetable.'' See Table~\ref{tab:table-label}.

\begin{dunetable}
[The LoT caption]
{cc}
{tab:table-label}
{The full caption that appears above the table.}
Rows & Counts \\ \toprowrule
Row 1 & First \\ \colhline
Row 2 & Second \\ \colhline
Row 3 & Third \\ % no \colhline on final row
\end{dunetable}

%%%%%%%%%%%%%%%%
\subsection{Numbers for my Widget}
\label{sec:chap-id:mywidget:num}

The following shows (1) how to do a bulleted list (notice the commas, the ''and'', and the period.  It also shows how to write different kinds of numbers.

\begin{itemize}
    \item 100 is written as \num{100},
    \item 1000 is written as \num{1000},
    \item 123.456 is written as \num{123.456},
    \item 1 plus or minus 2i is written as \num{1+-2i},
    \item 3 times 10 to the 45th is written as \num{3e45},
    \item 0.3 times 10 to the 45th is written as \num{.3e45} (keeps the decimal point before the 3), and 
    \item ''10, 20 and 30'' is written as \numlist{10;20;30}.
    \item 
\end{itemize}

%%%%%%%%%%%%%%%%
\subsection{Numbers with Units for my Widget}
\label{sec:chap-id:mywidget:numunit}

Here's how to do a numbered list and write numbers with units. 
\begin{enumerate}
    \item 120 GeV is written as \SI{120}{\GeV} or (more simply)  120\,GeV, and
    \item 4850 feet is written as \SI{4850}{\ft}.
\end{enumerate}

These and many others are defined in the file common/units.tex.

%%%%%%%%%%%%%%%%%%%%%%%%%%%%%%%
\section{My Second Amazing Widget}
\label{sec:chap-id:my2ndwidget}

If you have one section, you need at least two; same goes for subsections, etc. 

%%%%%%%%%%%%%%%%%%%%%%%%%%%%%%%
\section{More Information}
\label{sec:chap-id:moreinfo}

First, see Section~\ref{sec:chap-id:my2ndwidget} just to see how we do cross-sectional references.  The same goes for cross-chapter references, you just need to find the correct label for the chapter (or the section within the chapter).

More information is at \url{https://dune.bnl.gov/docs/guidance.pdf}.

% \cleardoublepage


% We have decided to utilize the FP package for doing calculations and tracking constants in the CDR
% (e.g. the limit of annual data from ll active FD modules to permanent storage at FNAL is 30 GB/year)
% The template for https://www.overleaf.com/project/5f5a7bb23f29aa0001f211d2a variable is this:
% "chapter""section"variable where both "chapter" and "section" are abbreviations
% with the Camel Caps used for readability
% Note that you should not override the use of a variable defined in generated/parameters.tex
https://www.overleaf.com/project/5f5a7bb23f29aa0001f211d2
% e.g. MonEtfDevPeople is the amount of effort for development in the ETF section of the Monitoring chapter
%  \FPset{MonEtfOpsPeople}{3.2}
%  \FPset{MonEtfDevPeople}{1.0}

% You can do floating point operations on constants defined in this manner:
% \FPadd\MonEtfTotalPeople\MonEtfOpsPeople\MonEtfDevPeople

% But note that floating point operations generate numbers with large precision and so you need to make
% sure to format numbers with printing them. In this example:

%\num[round-mode=places,round-precision=1]{\MonEtfTotalPeople}

% For FP set directly with \FPSet, then input precision will be maintained when printed.

%Global numbers for the entire document

\FPset{GlobalFDDataStorage}{30} % this is petabytes per year



\subfile{Algorithms and Frameworks} %  Tom Junk, David Adams

In order to accomplish the goals outlined in Part~\ref{part:overview}, a large number of algorithms must be applied to a large sample of data and also large samples of simulated data.  Care must be taken to apply the right calibrations and catalog the outputs so that the computational steps are unambiguous, complete and reproducible.  This part describes the frameworks in Chapter~\ref{ch:fworks}, databases in Chapter~\ref{ch:db}, applications in Chapter~\ref{ch:appl}, including simulation, reconstruction, and analysis tasks.

\documentclass[../main-v1.tex]{subfiles}
\begin{document}

\chapter{Frameworks \hideme{Norman,  Laycock,  Muether}}
\label{ch:fworks}
\todo{Define all the fun data concepts }
%%%%%%%%%%%%%%%%%%%%%%%%%%%%%%%%
%\section{xyz}
%\label{sec:fworks:xyz}  %% fix label according to section

The unique challenges of reconstructing time-series based objects from the DUNE far detectors and of adapting code to run on new generations of computing resources such as \dwords{hpc} requires that we reoptimize our main computing frameworks.  In this Chapter we provide a brief overview or our existing codes followed by a description of our process for designing the optimal framework needed when full \dword{dune} begins operations.  


\section{Current status \hideme{ Junk/Muether - draft}}

The data processing framework in use by the \dword{fd} simulation and reconstruction efforts, the \dword{protodune} detectors and \dword{ndgar} is \dword{art} developed at Fermilab.  The \dword{art} framework defines the data processing loop, manages memory, interfaces to I/O tools, defines uniform mechanisms for defining, associating, and persisting data products, provides a uniform mechanism for job configuration, stores job configuration information in its output files, and manages messages, random numbers and exceptions.  The \dword{art} framework runs user code that is provided as {\it plug-ins}, which are built as separate shared object libraries that can be dynamically selected and loaded at runtime.  The \dword{art} framework is also used by the \dword{nova}, Muon g-2, MicroBooNE, ICARUS, LArIAT and ArgoNeuT collaborations.  The \dword{art} framework evolved from CMS's software framework and started being used by intensity-frontier experiments in 2011~\cite{Green:2012gv}. It is developed, maintained and supported by Fermilab's \dword{scd}.

The \dword{larsoft} toolkit is a collection of \dword{art} plug-ins and associated algorithm code, configuration files, and static data such as geometry specification files and photon visibility maps.  \dword{larsoft} provides the interface to event generators such as \dword{genie} and \dword{corsika}, detector simulation via \dword{geant4}, custom simulation and reconstruction software, event displays and tutorials.  Experiment-specific metadata plug-ins assist in batch workflow organization.  \dword{larsoft} is supported by Fermilab's \dword{scd}, though much of the software has been contributed by participating experimental collaborations.

% paragraph on SSRI/\dword{tms}

The gaseus detector simulation, \dword{garsoft} is patterned on \dword{larsoft}.  It is a software toolkit for simulating and reconstructing data from \dword{ndgar}.  It is also based on the \dword{art} framework.  Like \dword{larsoft}, it provides interfaces to event generators and \dword{geant4}, custom simulation and reconstruction software, and event displays.  It also provides simulation and reconstruction for \dword{ndgarlite}.  \dword{garsoft} is written, maintained, and supported by the DUNE collaboration.

The \dword{ndlar} software effort is currently being developed as a series of standalone tools for simulating and reconstructing pixel-based \dword{lartpc} data.  Toolchains have been developed for analyzing \dword{singlecube} prototype data.

The \dword{sand} software effort benefits from long experience with the \dword{kloe} detector, which provides the magnet and calorimeter.  New software is developed for the \dword{3dst} and other components of SAND.  Flexibility is important at this stage in order to allow studies of different detector designs to fill in between the \dword{3dst} and the calorimeter, such as a gaseous \dword{tpc}.  Because \dword{sand} will not move off axis with \dword{ndlar} and \dword{ndgar}.


\section{Framework Requirements \hideme{Laycock/Norman - needs update}}
\subsection{Requirements Process} %-\hideme{Norman/Laycock in progress}}
% Taken from vCHEP paper
Modern \dword{hep} software frameworks are in the process of addressing the increasing heterogeneity of computing architectures that are redefining the computing landscape, work that is closely followed by the HEP Software Foundation~\cite{Alves:2017she, Calafiura:2018rwe}. The \dword{dune} collaboration is keen to participate in these efforts to minimize unnecessary development effort in improving our own framework. 

Given the unique challenges that \dword{dune} data pose, in 2020, the collaboration assembled a task force to define the requirements for its software framework based largely on physics use cases \cite{bib:docdb21934}.  The collaboration then approached the \dword{hsf} who assembled a panel of software framework experts from various experimental backgrounds to review these requirements. The report \cite{bib:docdb24423} produced by this panel was discussed in a workshop involving the panelists which resulted in a followup workshop to address the outstanding issues, mostly around the topic of concurrency.  The findings of the panelists and the summary of the concurrency workshop \cite{bib:docdb24426} have been incorporated into the \dword{dune} software framework requirements presented in the following.

\subsection{Summary of Use Cases}

\todo{Summarise the use cases in section 2 in a table.}

The use cases for \dword{dune} are summarized in table BLAH.

THIS IS TABLE BLAH
%Would be useful to identify "production" use cases and "analysis" use cases.


One striking feature of the use cases that \dword{dune} needs to support is the diversity in scale of the data processing atom, the smallest unit of data that can be processed independently.  Somewhat counter-intuitively, although the raw trigger records are themselves very large, the data processing atom (a single APA) is relatively small and compatible with the same high-throughput computing workflows adopted by most \dword{hep} experiments.  Meanwhile at the analysis level although the quantity of data related to a single trigger record is small, nuisance parameter extraction requires correlations across all trigger records in an analysis dataset, which is then the data processing atom.  This is a very good fit to high performance computing, particularly if the analysis framework can take advantage of many \dword{hpc} nodes at the same time. Experience from the \dword{hsf} panelists encouraged \dword{dune} to separate the analysis and production use cases when considering the software framework design.

\subsection{Analysis\hideme{Norman/Laycock - still needs more}}

Based on the recommendations of the \dword{hsf} panel experts, the analysis use case is not required to be supported by the same software framework as that used for production.

\todo{Needs much more on the actual analysis requirements, including ML.}

Machine learning is already heavily used in analysis of ProtoDUNE data and the framework should give special attention to machine learning inference in the design, both to allow simple exchanges of inference backends and to record the provenance of those backends and all necessary versioning information.  Finally, the framework should be able to work with both Near Detector and Far Detector data on an equal footing, and within the same job.

%Now move on to discussing "the" framework requirements, separate section? 

\subsection{Data and I/O layer} % Needs some updating but mostly ok
As noted the discussion on raw data processing, the DUNE software framework must have the {\bf ability to  change its unit of data processing} from single planes to supernova readouts with extreme flexibility and ensure that memory is treated as a precious commodity.  
It must be possible to correlate these units to DUNE data-taking "events" for exposure accounting, and experiment conditions.
{\bf Partial reading of data} is another strong requirement needed to keep memory usage under control.  There must be {\bf no assumptions on the data model}, {\bf the framework must separate data and algorithms} and,  further, {\bf separate the persistent data representation from the in-memory representation} seen by those algorithms.  DUNE data is well suited to massively parallel processing compute architectures, so the framework will need to {\bf have a flexible I/O layer supporting multiple persistent data formats} as different formats may be preferred at different stages of data processing.  The sparsity of DUNE data also imply that compression will play an important role in controlling the storage footprint, especially for raw data, and the sparsity of physics signals further emphasize support for data reduction (skimming, slimming and thinning) in the framework.  The framework {\bf needs to be able to handle the reading and writing of parallel data streams}, and navigate the association (if any) between these streams with a {\bf critical need to allow experiment code to  mix simulation and (overlay) data}.

\subsection{Concurrency} % Need to distinguish production and analysis
Many aspects of DUNE data processing are well suited to concurrency solutions and {\bf the framework should be able to offload work to different compute architectures efficiently}, facilitate access to co-processors for developers and {\bf schedule thread-safe and non-thread-safe work}.  {\bf It must be possible to derive the input data requirements for any algorithm in order to define the sequence of all algorithms needed for a particular use case, and it must be possible to only configure those components required for that use case.}

\subsection{Reproducibility and provenance} % HSF had lots of comments
As previously stated, the framework must ensure that memory is treated as a precious commodity, implying that {\bf intermediate data products cannot occupy memory beyond their useful lifetime}.  Nevertheless, reproducibility is a key requirement of any scientific tool and {\bf the framework must provide a full provenance chain} for any and all data products which must include enough information to reproduce identically every persistent data product. By definition, the chain will also need to include sufficient information to reproduce the transient data passed between algorithm modules, even though the product is not persisted in the final output.
It is also highly desirable that the framework broker access to random number generators and seeds in order to guarantee reproducibility.  All of the preceding considerations imply that {\bf the software framework will need a very robust configuration system} capable of handling the requirements in a consistent, coherent, and systematically reproducible way.


\section{Timeline \hideme{Norman/Laycock needs a lot more}}

Near term vs long-term plans.  As the production use case can be taken care of by a traditional software framework, the focus in the near term should be to continue as we are and thereby learn more about the physics use cases and ultimately the needs of DUNE.  The long-term plan should not be saddled with supporting short-term needs for ProtoDUNE.  Given the timeline for DUNE, implementation of its framework is envisioned to start at (DISCUSS !!) the end of ProtoDUNE.

\subsection{Missing functionality \hideme{Norman/Laycock - needs more}}

The DUNE production framework will ultimately need to handle much more demanding trigger record sizes than ProtoDUNE and this will be the main development challenge.  In addition, major development work will be needed for:

\begin{itemize}
    \item one
    \item two
\end{itemize}

For analysis, early investigations into an MPI-based framework (CITATION NEEDED) already showed promising results.  The compatibility of DUNE analysis to HPCs suggests that this would be an efficient way for analysis computing resources to be provisioned to DUNE.

\subsection{Plan to go forward \hideme{ Norman/Laycock -needed }}

Development plan.

%This plan was further reviewed by the HSF, maybe.
%
%\subsection{HSF recommendation for a path forward}
\end{document}  % Paul and Andrew
\cleardoublepage

\documentclass[../main-v1.tex]{subfiles}
\begin{document}
\chapter{Databases \hideme{Buchanan and Laycock - draft}}
\label{ch:db}

%%%%%%%%%%%%%%%%%%%%%%%%%%%%%%%%
\section{Introduction  \hideme{Buchanan and Laycock - draft} }
\label{sec:db:intro} 

In order to accommodate the large range of metadata that will be tracked by \dword{dune}, the \dword{dune} \dword{db} structure comprises several databases specific to the information, or metadata, that they contain. 
The subset of all \dword{dune} metadata that is required, in the strict sense of being absolutely necessary, for data processing and analysis needs to be carefully identified and assessed.  We refer to this subset of metadata as "conditions metadata", see section \ref{subsec:db:conditions_metadata}.
It is critical that users be able to access conditions metadata throughout the full data processing and analysis chain with as little burden as possible. To achieve this, users will interact with a centralized high-level interface   described in Section~\ref{sec:db:conditions}.

The \dword{dune} experiment is expected to operate for 20-30 years and the \dword{dune} databases need to be reliably maintained and operated for that entire period. In order to accommodate this requirement, the database system should not rely on implementation solutions that possess the risk of becoming unavailable during the operation period. Open-source, non-proprietary solutions will therefore be used. Currently the databases housed at \dword{fnal} use the open-source PostgreSQL (Postgres) relational database management system. Postgres is supported by the \dword{scd}.       
\dword{dune} would also like to benefit from the close collaboration between \dword{doe} laboratories to improve service availability and mitigate single-point-of-failure risk by having secondary databases at other \dword{doe} labs.  \dword{bnl} is a good example of a lab that already provides database services to international experiments using the same Postgres technology.

It is expected that there will be reconstruction and analysis jobs distributed across large numbers of traditional grid-based and high-performance computing (\dword{hpc}) systems and that database access will need to be able to scale appropriately. Additionally, it is important to ensure that users are able to work on analysis tasks when unable to access the database directly through a network connection. 

Some of the database solutions outlined in this document have been deployed and tested to some degree during Run I of the \dword{protodune} experiment. Experience coming from %anne\dword{protodune} Run I 
this will be briefly described in the sections below, when relevant. %anneRun II of the 
The \dword{protodune2} experiment will provide a further testbed for the database systems proposed for DUNE.  

\subsection{Conditions Metadata}
\label{subsec:db:conditions_metadata} 

\dword{condmeta}
%Conditions Metadata 
is defined as the information necessary to understand the context of physics data, e.g., beam data or calibrations.
%special run data
Metadata can be indexed by %anne either 
time, run, or fraction of a run (subrun). 
%Additionally, 
An \dword{iov}
%\dwords{iov2} 
defines the period, indexed by any of the three options, for which a given metadata is valid.
%may be used to index metadata that is used over periods not directly corresponding to run or subrun boundaries. 
Alignment constants are an example of conditions metadata that will likely be valid for several runs. 
The DUNE conditions database will have APIs that provide easy and transparent access to all conditions metadata independently of technical details.
%be indexed in a fashion that makes access by users transparent, or APIs will provided to serve the same purpose.

Time-indexed metadata will, in some cases, be sampled at a rate higher than typically needed by offline users. In these instances the metadata will be filtered down, or interpolated, to a lower rate for inclusion in the conditions database. In general, metadata falls into two categories, interpolated and non-interpolated, where an example of the latter would be run-indexed values pertaining to run configuration. Interpolated metadata, e.g., values read back from the slow control system, can be interpolated through a rolling average or updated on changes to the values. The method used will depend on the natural variation of the value being recorded and the physics use case. 
%I removed the user defining this, to stress that it should be a physics motivation
The database group will provide interpolated values to match the physics requirements of the experiment.  

Conditions metadata will in general be stored in appropriate databases but there will be some cases where it is more reasonable to include the metadata with the raw trigger records instead. An obvious example of this is metadata that changes at the individual trigger record level, i.e., trigger bit information. This metadata will be extracted from the data during the reconstruction for inclusion in the appropriate database. 

The following table contains classes of metadata: 

\begin{comment} REDID BELOW - WAS TOO WIDE (ANNE) %%%%%%%%%%%
\begin{dunetable}
[Run Configuration Database Content]
{l l l l} 
%\begin{table}[ht!]
%\centering
 %\begin{tabular}{l l l l} 
 {table:metadata}
 {Example metadata values and types stored in the run configuration database.}
 Metadata  & Example(s) & Database &  Interpolated? \\ [0.5ex] 
 %
Run Configuration   &  Start Time, config file & Run Configuration & No\\  
Detector Conditions  & TPC high voltages & Slow Control & Yes  \\ 
Beam Conditions  &  Horn polarities, beam current & IFBeam & Yes \\  
Hardware Information & Component history &  Hardware/QC  & No \\  
Calibration Constants & Channel gains  & Calibration & No \\ 
Physics/Hardware Locations & Channel maps & Geometry & No \\  
Data Quality & Good runs list & \Dword{dqm} & No \\  
%
%\end{tabular}
%{Example metadata values and types stored in the run configuration database.}
%\label{table:metadata}
\end{dunetable}
\end{comment}
%%%%%%%%%%%%%%%%%%%%%%%%%%%%%%%%%%%

\begin{dunetable}
[Run Configuration Database Content]
{l l l} 
 {table:metadata}
 {Example metadata values and types stored in the run configuration database. (I) indicates interpolated metadata.}
 Metadata  & Example(s) & Database  \\ 

Run Configuration   &  Start Time, config file & Run Configuration \\   \toprowrule
Detector Conditions (I)  & TPC high voltages & Slow Control   \\ \colhline
Beam Conditions (I)  &  Horn polarities, beam current & IFBeam  \\ \colhline  
Hardware Information & Component history &  Hardware/QC   \\ \colhline  
Calibration Constants & Channel gains  & Calibration  \\ \colhline 
Physics/Hardware Locations & Channel maps & Geometry  \\ \colhline  
Data Quality & Good runs list & \Dword{dqm}  \\  
\end{dunetable}

A DUNE metadata task force was assembled in 2020 and a resulting report discussing the interfaces between online and offline systems can be found in reference~\cite{bib:docdb22983}.

\section{Conditions Database  \hideme{Buchanan and Laycock - draft}}
\label{sec:db:conditions} 

The conditions database is a high-level database that provides an interface to offline users and processes. The motivation for such a centralized database is to provide an easy-to-use interface for users and to reduce the number of database connections required by offline processes. This will ensure that jobs will be ``lightweight'' and processing time will not be extended due to database accesses.  The granularity and content of the conditions database will be well-matched to offline needs by design. which will allow the heavy use of caching technologies to optimize resource usage. The design of the interface between the data processing software framework (see chapter \ref{ch:fworks}) and the conditions database will play an important role if all of the access patterns needed by \dword{dune} are to be well supported. Experience from other \dword{hep} experiments with distributed computing resources show that careful design of this interface is crucial to the success of the computing model \footnote{Overlays (see section \ref{sec:usecases_overlays} are an example of a workflow that can cause issues for conditions database access if this interface is not well designed.}.
Figure~\ref{fig:dbmap} shows the relationship between the conditions \dword{db} and the other \dword{dune} databases. For \dword{protodune} the "Master Metadata Store" is an intermediate database that allows for potentially heavy data interpolation tasks to run without creating extra load on the other, often critical, database services.  It provides separation between the online and offline worlds at the cost of partial data duplication. A design utilizing the Master Store also allows for maximum flexibility without the need for a predetermined schema. The final database system design for \dword{dune} will benefit from experience with this design.

\begin{dunefigure}
[Map of DUNE databases]
{fig:dbmap} 
{Map of DUNE databases showing conditions database (combination of Master Store and Run Conditions DB). The arrows illustrate the flow of metadata. Offline access to conditions metadata is made through the Run Conditions DB, which is a relational DB containing a small subset of the metadata from the Master Store DB. }
\includegraphics[width=.9\columnwidth]{graphics/Databases/DBSystem-cartoon.png}
\end{dunefigure}

The conditions database will contain interpolated (e.g., slow controls) and non-interpolated (e.g., run configurations) information. Interpolated information may more naturally be keyed by time-stamps, while configurations are more naturally keyed by run number. Tools will need to be developed to handle these two types of database content. 

\subsection{Conditions Database for \dword{protodune}}

In order to provide a balance between the availability of the largest set of metadata possible  and allowing schema evolution,  the \dword{protodune} conditions database will employ an unstructured approach utilizing a No-SQL database \dword{ucondb}~\cite{bib:ucondb}. The \dword{ucondb} where metadata from several specific databases and sources, like the \dword{daq} system, will be stored in ``blobs'' corresponding to temporal periods (run, time blocks, or \dwords{iov2}). Each blob will contain a \dword{json}-formatted record of metadata. Folders will be used to hold metadata corresponding to time and run keys. Tools will be provided to correlate between the two. 

The following is an example fragment of DAQ metadata from the first run of  \dword{protodune}. A typical file from this run was on the order of 10  

\begin{verbatim}
{
Start of Record
Run Number: 5185
Packed on Oct 11 22:19 UTC
#####
boot.fcl:
#####
DAQ_setup_script: "/nfs/sw/work_dirs/dune-artdaq_artdaq_v3_03_00_beta/setupDUNEARTDAQ" 
PMT_host: "localhost" 
PMT_port: 5400 
debug_level: 1 
partition_number: 0 
tcp_base_port: 15000 
request_port: 3000 
request_address: "227.128.12.25" 
table_update_address: "227.129.1.128" 
routing_base_port: 10010 
zmq_fragment_connection_out: 17437 
...
}
\end{verbatim}

%A smaller relational database will contain a subset of the metadata most often queried by users. This will enable much faster queries than accessing the unstructured conditions database directly.

\subsection{Conditions Database for DUNE}

For \dword{dune}, the interface between the conditions database and the software framework will become particularly important.  The consensus among several \dword{hep} experiments was reported in an \dword{hsf} whitepaper \cite{Laycock:2019ynk}. The use of a lightweight relational database schema allows a single point of entry, a "global tag", to configure conditions data access for the framework.  Evolving the \dword{protodune} solution to benefit from that experience is foreseen by the database group. Further details on the development plan can be found in section \ref{sec:db:conddbdev}.

\section{Run Configuration Database  \hideme{Buchanan and Laycock - draft}}
\label{sec:db:config}  

The run configuration database contains the intended configuration of the detectors and \dword{daq} during data collection -- physics or otherwise. 

Metadata contained in the run configuration database includes hardware settings, run type, and run start and end times. Table~\ref{table:runconfig} contains some examples of typical metadata that will be contained in the run configuration database. 

\begin{dunetable}[Run configuration database example]
{l  l } 
{table:runconfig}
{Example metadata values and types stored in the run configuration database.}
%\begin{table}[h!]
%\centering
% \begin{tabular}{l  l } 
% 
 Metadata Value & Type  \\ [0.5ex] 
 
Start of run   &  Time \\ \toprowrule
Readout window size  & Integer  \\ \colhline
Readout trigger type  &  Integer \\  \colhline
Readout firmware version &  Integer \\  \colhline
Baseline start &  Integer \\  \colhline
Shifter comments &  Text \\  \colhline
Run end status & Integer \\  
%
%\end{tabular}
%\caption{Example metadata values and types stored in the run configuration database.}
%\label{table:runconfig}
%\end{table}
\end{dunetable}

The majority of run configuration metadata comes from the configuration files used by the \dword{daq} system during run execution. Some additional metadata collected at the end of the run, or shortly thereafter, may also be included. Examples are run completion status and comments made by the shifter during the run or in run-related checklists.

Parameters used to configure the run will be collected and packed into \dword{json}-formatted blocks in a single blob corresponding to a \dword{daq} run.   

\subsection{Run Configuration Database for ProtoDUNE}
\label{sec:runconfigPD}

 Following the completion of a run, the run configuration parameters corresponding to the run are read from the mongoDB and packed into a single %anne"blob" 
 blob of key-value pairs in \dword{json} format. Any additional information, such as end-of-run time, are added to the blob, which is then transferred to the \dword{ucondb} at \dword{fnal}. A typical metadata blob is on order 10\,MB in and contains more information than most users will want to use. An additional step of reducing the metadata is performed to produce a subset of metadata needed by offline users. The reduced set of metadata is stored in a single table in a relational database referred to as the ``run history database.'' An interface is provided to users, enabling them to retrieve run numbers and file locations based on queries of the history database. 

For %anne Run II of 
\dword{protodune2} the conditions \dword{db} will include metadata from multiple sources, as will be the case for DUNE. Metadata from the beam conditions database (IFBeam), the slow controls database, and data quality monitoring database (DQMDB), will be included in addition to the run configuration information. The \dword{ucondb} is able to store data keyed by either non-interpolated values (run number) or interpolated values (timestamps). Users are then able to access information using run numbers or dates and times.

\begin{dunefigure}
[Flow of metadata from \dshort{protodune} DAQ to user interface]
{fig:protoconditions} 
{Flow of metadata from \dword{protodune} \dword{daq} to user interface.}
\includegraphics[width=.9\columnwidth]{graphics/Databases/Conditions_ProtoDUNE.png}
\end{dunefigure}

\section{Data Quality and Monitoring Database  \hideme{Buchanan and Laycock - draft}}
\label{sec:db:dqm}  

The \dword{dqmdb} contains monitoring histograms and metadata derived from data collected during operation of the DUNE detectors. The \dword{dqmdb}  is an online database and
the histograms it stores to assess and monitor data quality are critical for the operations of \dword{dune}, but not relevant for offline data processing and analysis.
The derived data quality metadata, which include boolean flags indicating the results of the online data quality assessment algorithms, are relevant for offline analysis and this small subset of \dword{dqmdb} data will require an interface with the conditions database either directly or via an offline replica of the data quality database.

\section{Offline Calibration Database  \hideme{Buchanan and Laycock - draft}}
\label{sec:db:calib} 

The calibration database contains calibration constants determined from collected data corresponding to  \dwords{iov2}. The conditions metadata from the calibration system will result from offline calculations using data collected from the DUNE detectors.
There will generally be multiple versions of calibration constants corresponding to the same \dword{iov}, and the conditions database will provide coherent access to the appropriate version of these calibrations via e.g., the global tag mechanism.
%These versions will be contained within the database and accessible to users.  

\section{Slow Control Database \hideme{Buchanan/Laycock - draft}}
\label{sec:db:slowcontrol}  

The slow control \dword{db} contains metadata specific to the state of detectors during the time data were collected as well as before and after. Examples of slow control metadata are measurements of power supply voltages and currents, and temperatures. Each slow control quantity corresponds to a particular device. The slow control \dword{db} metadata is time-indexed and hence must be interpolated. Additionally, different devices will be sampled at different rates.

The slow control metadata is captured via a Supervisory Control and Data Acquisition (\dword{scada}) system that is the responsibility of the Slow Control and Monitoring group. The \dword{scada}  system pushes values to a back-end database, where the \dword{db} flavor is tied to the \dword{scada} solution. 
The \dword{scada}  system can provide data reduction through filtering prior to insertion of metadata into the back-end \dword{db}, which reduces the workload on any API used to move the metadata to the conditions database. 

\subsection{Slow Control Database for ProtoDUNE}
\label{sec:slowcontrolPD}

The \dword{protodune} experiment has been using an Oracle~\footnote{Oracle\textcopyright, \url{https://www.oracle.com}} back-end \dword{db} for the slow control system. As 
\dword{scd}
%Fermilab Scientific \todo{(anne) did you mean Core? }Computing Division 
does not support Oracle, the information from the Oracle database must be extracted and moved into a Postgres \dword{db} at Fermilab. Any filtering of the metadata not handled by the \dword{scada}  system when populating the Oracle database can be handled by the API that transfers the Oracle records to Postgres.

No data filtering was provided by the \dword{scada} system for \dword{protodune} Run I but for Run II it is expected that the slow controls conditions metadata will be filtered based on the physics needs.

\section{Beam Conditions Database - IFBeam  \hideme{Buchanan and Laycock - draft}}
\label{sec:db:ifbeam}  

The beam conditions database, \dword{ifbeam}~\cite{ifbeam} will contain metadata related to the condition extracted beam and corresponding diagnostics.  The functional form of this database is essentially the same as that of the slow control database. A large number of devices are sampled into the \dword{ifbeam} \dword{db}. The \dword{ifbeam} metadata transferred to the conditions \dword{db} will be a coarser subset of the original set.

Quantities contained in the \dword{ifbeam} \dword{db} include beam currents, horn currents and polarities, and beam monitoring instrument metadata.

\begin{dunetable}
[Example IFBeam metadata]
{l  l } 
{table:ifbeam}
{Example metadata values and types stored in the \dword{ifbeam} database.}
%\begin{table}[h!]
%\centering
% \begin{tabular}{l  l } 
% 
 Metadata Value & Type  \\ \toprowrule [0.5ex] 
% 
Horn 1 Polarity &  Integer \\ \colhline
Horn 2 Polarity  & Integer  \\ \colhline
Beam current & Float \\  
%
%\end{tabular}
%\caption{Example metadata values and types stored in the run configuration database.}
%\label{table:ifbeam}
%\end{table}
\end{dunetable}


\section{Hardware Database  \hideme{Buchanan and Laycock - draft}}
\label{sec:db:hwdb}  

The principle purpose of the hardware database (\dword{hwdb}) is to track the lineage of hardware components and record the results of their \dword{qc} tests. In this context a component can be a sub-detector module or any of the individual parts comprising it. For example, a readout board is a component as is a mezzanine daughter board or programmable logic chip mounted on the readout board. The lowest level component tracked within the \dword{hwdb} will be unique to the corresponding hardware system. 

A requirement of the \dword{hwdb} is that any component, or part, stored in the database must have a unique identification number assigned to it, coordinated by the \dword{dune} Integration group. % Jim Stewart 
%A separate database, under the responsibility of the DUNE Integration group, 
%\todo{(anne) is this a formal group, and is this the title?} will be the source of the unique part ID numbers assigned to each component. 
The part numbers will be designated as shown in Table~\ref{table:partsid}. The project field corresponds to DUNE detectors (D), integration (I), LBNF (L), and future project (P). The project identifier is allocated by the project management team while the other identifiers are left for the various hardware consortia to assign. There are additional fields not listed as they are not relevant to the \dword{hwdb}.  More details of the parts identification number can be founds in~\cite{bib:cernedms2505353}.

\begin{dunetable}
[Hardware database component IDs]
{l l l l l l} 
{table:partsid}
{Unique parts identification number assigned to each component stored in the hardware database.}
Project & System ID & Subsystem ID & Item Type ID & Dash & Item Number  \\ \toprowrule 
D/I/L/P & 01-99 & 001-999 & 00001-99999 & - & 00001-99999 \\  
\end{dunetable}

Hardware \dword{db} metadata will reflect the complete lifetime of the detector component, including the following:

\begin{itemize}
\item Procurement, 
\item Fabrication,
\item Quality control testing,
\item Shipping and storage,
\item Installation, and
\item Maintenance. 
\end{itemize}

The relationships between components will be reflected in the \dword{hwdb}. Metadata corresponding to multiple instances of events such as \dword{qc} tests will be handled using time series within the database. 

The database group will provide an interface to the \dword{hwdb} and each hardware consortium will be required to ensure that their metadata is inserted into the database. Given the wide range of hardware consortia, the international nature of the experiment and the fact that individual consortia must manage their own construction projects, it is 
beyond the scope of the database group to dictate how the consortia will handle data entry into the \dword{hwdb}.  Instead, the database group will consult with the consortia, and in particular their database group liaisons, providing documentation for the \dword{hwdb} and advising on efficient methods for working with it.  Sharing of tools will be very strongly encouraged to reduce the duplication of effort as far as possible.
%expected that the various consortia will develop different metadata capture and temporary storage solutions. The database group will consult with the consortia but ultimately the consortia will need to provide their own APIs for record insertion into the \dword{hwdb}.  


\section{Service and Maintenance  \hideme{Buchanan and Laycock - draft}}
\label{sec:db:service}  

Most, if not all, of the DUNE databases will operate in advance of the full DUNE experiment coming online and these databases will need to be maintained and serviced once they are operational. 

The second run of the \dword{protodune} experiment (\dword{protodune2}) will employ a suite of databases that will be the precursors to the full database system that will be in place for DUNE. Each of these databases (run configuration, beam instrumentation, conditions, slow controls, and hardware) will require stable monitoring, maintenance, and service to address operational issues that will arise in the lead up to and during %anne the running of the \dword{protodune} II experiment. 
\dword{protodune2} operations.

Monitoring will be achieved using automated web-based tools \todo{[ref needed]|} and responses to offline database issues will be made within an 8-hour period corresponding to a typical operation or production ``shift'', i.e. 5 days a week
\footnote{The assumption is that offline database services will not require 7 days a week coverage.}
. For \dword{protodune2} databases will be located at both \dword{fnal} and \dword{cern}, both of which have a long history of database support.  Moving on to \dword{dune}, the expectation is that another \dword{doe} lab like \dword{bnl} would share database operations with \dword{fnal}.

\section{Development Plans  \hideme{Buchanan and Laycock - draft}}

There are a number of database-related projects where R\&D is needed or underway, utilizing effort \todo{how to credit the FOA funding???}
and expertise from both \dword{doe} labs and universities.  Coordination is provided by the Database group and attempts to balance the short-term needs of \dword{protodune} with the longer term needs of \dword{dune}.

\subsection{Conditions Database Development}
\label{sec:db:conddbdev}

The Conditions Database is the primary interface to conditions metadata for all \dword{dune} distributed computing resources and this task includes designing and developing caching strategies.  The \dword{dune} Database group will collaborate with the \dword{hsf} to benefit from the experience of other \dword{hep} experiments, as well as the experience gained from \dword{protodune}, to develop a robust system while optimizing development effort.

\subsubsection{Conditions Database Core Design}

The \dword{hsf} Conditions Databases group identifies the following points as key features of a good Conditions Database design:

\begin{itemize}
    \item    Loose coupling between client and server using RESTful interfaces
    \item    The ability to cache queries as well as payloads
    \item    Separation of payload queries from metadata queries
\end{itemize}

These guiding design principles are likely to remain valid throughout the lifetime of \dword{dune}, while implementation will need to evolve with technology.

\subsubsection{Conditions Database and Software Framework}

The interface between the Conditions Database and the Software Framework is defined by the Application Programming Interface (API) % let's see if we reuse this and it motivates a dword!
of the Conditions Database.  The \dword{hsf} Conditions Database group is in the process of defining a generic API, again using the experience of existing \dword{hep} experiments to understand best practice for supporting all read and write use cases.  \dword{dune} will collaborate with the \dword{hsf} on this API definition and share experience on implementations.  It is noted that, in line with the \dword{hsf}, given the heterogeneity of compute hardware that \dword{dune} needs to use, a common implementation is considered to be less important than a common API and sharing experience.

\subsubsection{Conditions Database Service Robustness and Distributed Computing}

In addition to the cache-friendly design of the central Conditions Database service itself, a major issue for a critical (for offline) service like the Conditions Database is resilience such that a robust and reliable service can be provided.  Here, \dword{dune} plans to take advantage of synergies with other \dword{doe} labs, particularly at \dword{bnl} which hosts Conditions Database services for Belle II and other experiments.  The backend database technology (Postgres) is common between \dword{fnal} and \dword{bnl}, and generally Conditions Database services, under the umbrella of the \dword{hsf}, are expected to evolve to look more and more similar.  \dword{dune} will investigate the potential for having a first class backup service (that would support writing as well as reading functionality), in addition to having redundancy of read functionality via intelligent use of caches at sites.


\subsubsection{Database Access for High-Performance Computing Facilities}

As DUNE will utilize \dword{hpc}   facilities for some analysis tasks it is important that \dword{db} access is manageable when tens, or hundreds, of thousands of processes are distributed across an \dword{hpc}   cluster. Studies on how scaling on such systems can be handled without overwhelming the conditions \dword{db} with an enormous number of simultaneous queries are required. %anne need to be undertaken. 
Here again, \dword{dune} plans to leverage the experience of other \dword{hep} experiments with experience of using \dword{hpc}'s at scale.

\subsection{Payload Serialization}

Similarly to the discussion in chapter \ref{ch:format}, the data format of conditions metadata deserves careful attention to avoid problems including technology lock-in.  The need to interact with many different groups, essentially all providers of conditions (and other persistent) metadata, is the main driver of the amount of effort required for this task.

\subsection{Slow Control Database Development}

The \dword{scada}  system chosen for Run I and II of the \dword{protodune} experiment was WinCC with Oracle as the back-end \dword{db}. This system is well tested and it is relatively trivial to transfer information from the Oracle \dword{db} to a \dword{postgres} \dword{db} located at \dword{fnal}. Given the fact that Oracle-based \dwords{db} are not guaranteed to be supported for the life of DUNE it would be beneficial to use a solution that enables a \dword{postgres} back-end.  The \dword{scada} system for \dword{dune} may also bring new challenges and the Database group are investigating potential solutions in collaboration with Slow Controls experts. 


\subsection{Hardware Database Development}

The core functionality of the Hardware Database is described in section \ref{sec:db:hwdb} and the first version is expected to be complete in FY22.  Experience gained as the consortia use the database are expected to motivate feature requests, many of which will require urgent attention to support the construction phase of \dword{dune}.  Looking further ahead, some attention will be needed to ensure the maintenance of this crucial database for the long lifetime of the collaboration.  Some information in the Hardware Database will correspond to Conditions Metadata (calibrations), thus requiring the transfer of data while retaining provenance.


\subsection{Run Configuration Database Development}

Transfer of Run Configuration information for offline use was a major problem for \dword{protodune} Run I, and significant effort will be used to improve the situation for Run II.  The evolution of both the \dword{dune} Conditions Database and \dword{daq} system could imply a similar amount of effort moving to \dword{dune}.


\subsection{Other Database Development}

The \dword{dune} Calibration system is expected to generate very large datasets that will be processed to create Conditions Metadata that needs to be transferred to the Conditions Database.  Meanwhile the Beam Conditions Database stores far more data than needed for the data processing Conditions Metadata, and effort will be needed to find efficient solutions for data reduction.  Finally, the Data Quality Monitoring Database is under the control of the \dword{dqm} group, with some amount of data needing to be transferred to the Conditions Database.  All of these connections could imply additional development work which, if arising from offline-only constraints, would should be supported by Database group effort.


\subsection{Database Access Tool Development and Documentation}

Guided by use cases, tools will be developed for all of the \dword{dune} Databases to enable to access of metadata for inclusion in the conditions database as well as direct access for studies\footnote{Most offline DB access will be made via the Conditions DB but there will be cases where expert users will need direct access to the various databases.}. An interface within the ART framework is already part of the DUNE software stack and has been used to access the IFBeam beam instrumentation database. This ART database service has been used, and is currently used, by the NOvA experiment. This interface will be the baseline for any other interfaces used by users interacting with the DUNE databases, including with the Conditions DB.

Additionally, APIs will be developed to provide the interfaces between the various databases and the Master Store DB, as well as between the Master Store and the Run Condition database. These will likely be light-weight Python-based applications. 

%As more use cases and workflows are discovered, all of the \dword{dune} Databases will likely need to have tools to allow efficient and easy access to them. 

Documentation is another critical aspect of the database system. This includes descriptions of the DB system components and training materials, that will need to be updated as development work proceeds. The number of use cases, databases and groups involved drive the significant amount of effort required here.


\subsection{Person Power Estimates}

The personnel needs will be largely front-loaded as the database systems are researched, implemented, and tested. The databases requiring the most effort will be the slow control, run configuration, calibration, and conditions databases.  

Estimates of the personnel needs over the next three years are given in Table~\ref{table:dbneeds}. 

\todo{PJL 03/15 - check these effort levels vs tasks looks somewhat sane}
\begin{dunetable}
[Person-power estimates for database development]
{l l}
{table:dbneeds}
{Database group R\&D person-power needs (FTE years) from FY22 to FY27.}
% \begin{table}[h!]
% \centering
%  \begin{tabular}{l l l l l} 
 
 % Assume we will mainly need to justify out to 2027 when there is a ramp down
 Task & Effort (FTE years) \\ \toprowrule
 
Conditions \dword{db}             & 6 \\ \colhline
Payload Serialization             & 4 \\ \colhline
Slow control \dword{db}           & 2 \\ \colhline
Hardware \dword{db}               & 2 \\ \colhline
Run Configuration \dword{db}      & 1 \\ \colhline
Calibration \dword{db}            & 1 \\ \colhline
Tools and Documentation           & 2 \\ \colhline
\colhline
Total effort \dword{db}           & 18.0 \\ 
%%\end{tabular}
%\caption{Database group person power needs over next 3 years.}
%\label{table:dbneeds}
%\end{table}
\end{dunetable}
\end{document} % Norm and Paul 
\cleardoublepage

\chapter{Applications overview}
\label{ch:appl}

This chapter describes the current state of the simulation, reconstruction, and analysis applications, the resource usge and the plans for future development.  Algorithms for ProtoDUNE-SP and the single-phase far detector modules are fairly mature, while algorithms for the near detector, the dual-phase far-detector module(s), and the Module of Opportunity are in an earlier stage of development.

\cleardoublepage

% these are now sections and so include them here

\documentclass[../main-v1.tex]{subfiles}
\begin{document}
\section{Simulation Algorithms \hideme{TRJ - draft}}
\label{sec:algo:sim}

\subsection{Beam simulation}
\label{sec:beamsim}

The future \dword{lbnf} beamline is simulated via the \dword{g4lbnf} package, which is based on the systems created for and validated on the \dword{numi} beamlines.  The \dword{proto} beamlines were designed and simulated using the MAD-X framework used at CERN \cite{PhysRevAccelBeams.20.111001}.

\subsubsection{Event Generators}
\label{sec:eventgen}

Extracting physics results from the DUNE experiment requires comparing the observed data with simulations which include detailed simulations of the physics processes under study as well as the response of the detectors.  The physics simulation is performed by the neutrino generators \dword{genie}~\cite{Andreopoulos:2009rq}, NuWRO~\cite{NuWro2012}, GIBUU~\cite{Gallmeister:2016dnq}, NEUT~\cite{Hayato:2009zz}, and others.  Cosmic-ray simulations are performed with \dword{corsika}~\cite{Wentz:2003bp,Dembinski:2020wrp} for detectors on the surface, and MUSUN/MUSIC for detectors deep underground~\cite{Kudryavtsev:2008qh,LBNEDOCDB9673}.  Radiological decays are modeled with BXDECAY0~\cite{Ponkratenko:2000um} and the DUNE-specific RadioGen.

\subsubsection{Detector Simulation}
\label{sec:detsim}

Factorization of the simulation into a generation stage and a detector simulation stage is a common situation in collider experiments, such as ATLAS and CMS.  The fact that the interactions simulated by generators for collider physics happen inside an evacuated beampipe means that the details of the detector geometry and materials are not relevant for most event generation.  Lists of four-vectors of particles emerging from a primary vertex will suffice.  In a neutrino experiment, however, the detector material is the target material, and hence the generators must be aware of the detector geometry and materials, which affects the structure and performance of the generator code.  Currently, \dword{genie}, \dword{corsika}, MUSUN/MUSIC, BXDECAY0 and RadioGen are integrated with \dword{larsoft}.  \dword{genie} is integrated with \dword{garsoft}.


Two classes of simulation of DUNE's detectors exist at the time of writing. 

Parameterized, or ``fast'' detector simulations involve smearing truth-level physics quantities based on expected detector performance metrics, such as acceptance and energy resolution.  These fast simulations are useful when optimizing detector designs, and for engaging physicists outside of the DUNE collaboration.  

Full simulations, on the other hand, are based on detailed geometry models and \dword{geant4}~\cite{Agostinelli:2002hh,Allison:2016lfl}, and are needed for extraction of publication-level results.

In a \dword{geant4}-based simulation of a DUNE detector, \dword{geant4} is used only to simulate the interacting particles from neutrino scatters and other processes of interest, and energy depositions in the active detector material are stored as a distinct data product. 

The propagation of low-energy drifting electrons and scintillation photons, signal induction  and electronics response are then simulated in a separate step.  

Drifting electrons are simulated parametrically using a model based on the measured drift velocity, longitudinal and transverse diffusion coefficients, and a parameterized model of space charge.  This last effect is particularly pronounced at \dword{pdsp} due to the large number of cosmic rays crossing the detector volume, giving rise to distortions in the apparent positions of particles of up to 30~cm.  There are also field distortions due to external imperfections such as the grounded electron diverters in \dword{pdsp}.

Once the electrons drift to the anode plane in wire-based \dwords{lartpc} in the simulation, a detailed two-dimensional model of the wire responses is applied~\cite{Abi:2020mwi}.  The two dimensions are wire number and time, and the effects of induced currents on neighboring wires are included in the simulation.  The electronics response function is folded in to a final model of the observed waveforms.  Simulated waveforms have been compared with real ones and are found to be very similar in \dword{pdsp}.
In the pixel-based \dword{ndlar} and \dword{ndgar}, the electronics simulation is at a simpler level, as the electronics have not been fully designed.


Photon propagation, scattering, absorption and detection are modeled in \dword{larsoft}'s simulation via a photon visibility lookup table, which gives the probability that a photon, emitted in a random direction at a specific location in the active volume of the detector, is detected by a specific photon detector.  The spatial granularity of the lookup table is a few cm.  This lookup table is populated with values that are computed from a \dword{geant4} simulation of scintillation photons in \dword{lar}, with the assumed values of the light attenuation length, the Rayleigh scattering length, the reflectivities of the surfaces inside the detector, and the transparency of the wire planes.  Scintillation photons are not yet simulated in the \dword{ndgar}, where photon detection systems are still under consideration. 

\dword{ndgar} also has a calorimeter and a muon system.  The calorimeter is currently simulated via parameterized responses to the \dword{geant4}-simulated energy deposits, as are the responses to tracks in \dword{ndgarlite} and \dword{sand}.

A large amount of code re-use and sharing via the design of \dword{larsoft} allows for the development of simulation algorithms for \dword{pddp}, the dual-phase far detector, and the Vertical Drift detector proposals.  Only the geometry, the field description, and the anode-plane models need to be updated; the rest of the simulation chain is re-used.


\subsection{ProtoDUNE simulation \hideme{Tingjun? -draft}}

The simulation framework has already been tested successfully in protoDUNE. 

The \dword{pdsp} simulation includes beam particles, cosmic ray interactions and radiological backgrounds. The beam particle species and momentum distributions are from the \dword{geant4} simulation of the H4-VLE beam line at CERN, which consists of $e^{+}$, $\pi^{+}$, $p$, and $K^{+}$ particles at 0.3, 0.5, 1, 2, 3, 6, and 7 GeV/$c$. The cosmic ray interactions are produced with the Corsika generator. Radiological backgrounds, including $^{39}$Ar, $^{42}$Ar, $^{222}$Rn, and $^{85}$Kr, are also simulated using the RadioGen module in \dword{larsoft}. The primary particles are tracked in the liquid argon using the \dword{geant4} package. The ionization electrons are drifted towards the wire planes. The effects of recombination, attenuation and diffusion are simulated. The accumulation of positive ions in the detector modifies the trajectory of ionization electrons and the strength of electric field, which is known as space charge effects and is simulated using the measured distortion and electric field maps. The electronic signal is simulated by convolving the number of electrons reaching each wire plane with the field response and electronics response. The field response is modeled using the GARFIELD package. The electronics response uses the parameterization from the SPICE simulation with the average gain and shaping time measured in the ProtoDUNE charge injection system. %Each event corresponds to a 3 ms readout window with a sampling rate of 0.5$\mu$s per time tick.  The total number of electronic channels is 15,360. 


\section{Photon Detector Simulation}

Photon simulation in large detectors is known to be highly computationally intensive due to the need to trace large number of photons over large distances.  \dword{dune} has several approaches to this problem. 

The sequence for photons simulation is:
\begin{enumerate}
\item particle track simulation in \dword{geant4} to produce the energy deposits along the track
\item  calculate the number of photon/electron emission at each vertex where energy deposits
\item 
simulate the photon transport in the detector (either full simulation or fast simulation)
\item  reconstruct the photons, including the hits and flashes reconstruction
\end{enumerate}
 
In LArSoft, after particles are propagated using GEANT4, energy deposits along tracks are recorded so that the number of electrons and photons generated at each step can be estimated using one of the available  ionization and scintillation methods. Once the number of photons is determined, the fraction of those photons that will actually reach a given photon detector is usually estimated using one of several fast light simulation methods. This procedure is followed for both 128 nm photons (Ar scintillation) and 176 nm photons (Xe scintillation) to account for the wavelength-dependent Rayleigh scattering in the simulation of Xe-doped liquid argon. 
Since a copious number of scintillation photons (25000~ph/MeV at 500~V/cm) is produced in LAr, it is very computationally demanding to individually propagate all photons using GEANT4. Instead,  fast simulations for photons are implemented.
%The photon simulation CPU time per event which depends on what kind of events are simulated and the simulation methods. 
The full \dword{geant4}  photon simulation CPU time per event depends on the energy deposited. 
In the fast simulation methods, using a library,  semi-analytic methods or machine-learning, the CPU time depends on the granularity of the detector. Generally speaking, the performance of the three fast simulations methods is at the same level. 
 
\subsection{Optical Library Method}
This method consists of dividing the cryostat volume into smaller parallelepiped shaped regions called voxels and creating a lookup table that can store the visibility of each photon detector to photons being generated inside a given voxel.
This optical library is created using the full \dword{geant4} simulation to generate photons anywhere inside a given voxel, with random direction and polarization, and then store the fraction of those photons which land on the optically sensitive region (visibility) of a given photon detector, identified within \dword{larsoft}  by its optical channel. When using the fast simulation, \dword{larsoft}  will retrieve the Optical Library and store its information to directly transform the amount of photons generated in a given step along a particle's track into the number of photons landing on each optical channel.
This method can satisfactorily be used as a fast simulation method. Nevertheless, its performance greatly depends on the size of the voxel and the number of photons being generated per voxel. Increasing the number of voxels in a library will improve the description and reduce the bias at the cost of a large increase in memory consumption. Increasing the number of photons per voxel will provide much better statistics and also largely increase the amount of time dedicated to generating the optical library. Special care should be taken for regions with smaller visibilities.

\subsection{Generative Neural Networks}
%Fast simulation with Generative Neural Network
This method relies on a generative neural network  trained on the photon detection system of a detector. The input to the network is the vertex where the photons are emitted, and the output is the visibility of each photon detector.
The generative model can be trained ahead of time using a full  \dword{geant4} optical photon simulation with photons, emitted from random vertices in the detector, and then be frozen to a computable graph and deployed to the production environment (\dword{larsoft}  framework).

When the computable graph is loaded in \dword{larsoft}, it quickly emulates photon transport by  computing the visibility of each photon detector according to the photon emission vertex along the particle’s track.
This method is 20 to 50 times faster than the \dword{geant4} simulation while keeping the same level of detail for particle tracks, such as number of energy depositions, and precision.
The model inference also requires a relatively small amount of memory. The samples for ProtoDUNE-like and DUNE-like geometries show the required memory for the model inference is around 15\% of the \dword{geant4} simulation. Further, this memory use is not directly correlated to the volume of the detectors.

\subsection{Semi-analytical models}
A large number of photons is generated at different points within the cryostat volume and propagated using \dword{geant4}. Gaisser-Hillas functions are fitted to the number of photons reaching the photon detectors as a function of source-detector distance and relative angle, and the resulting parameters are used during event simulation in order to extract the fraction of photons produced at a certain point that arrives at a given sensor.
 
\subsection{Comparison between fast simulation methods}
The semi-analytic model and the use of  optical library models for fast simulation have been compared to the full light simulation in \dword{geant4} for the \dword{sbnd}   experiment. This used a highly segmented and large photon count optical library, created with $\sim$1.6M voxels, with 0.5M photons being generated in each voxel (total of $7.9 \times  10^{11}$ photons), resulting in a file of size 1.2 GB. A similar optical library for the DUNE 1x2x6 geometry (volume of $7 \times 12 times 13.9$ m$^3$) would be prohibitively large.
All DUNE optical libraries produced so far have larger voxels and less photons per voxel being simulated in comparison with the \dword{sbnd}   one which was used as reference for the comparison of the two modes. It has been reported that the optical library struggles to properly describe light signals generated closer to the detectors and more on-axis (up to ~50º). This is a known issue caused by the intrinsic discontinuity of the voxelization schemes. In the protodune optical library, a smoothing of visibilities using neighboring voxels was used trying to minimize this effect.
The semi-analytic model, on the other hand, presents a reduced resolution when going off-axis due to shadowing effects. The influence of shadowing should be minimal for the case of DUNE FD as there are no PMTs in the geometry (in \dword{sbnd}  , the PMTs’ windows reach out to about $\sim$10cm beyond that the X-Arapuca windows), so in the DUNE case we expect  good performance of the semi-analytic model even in the far-off-axis cases.
Specifically for the X-Arapucas in \dword{sbnd}  ,  better performance of the semi-analytic model is expected: better resolution close-on axis (3.6\% vs 5.6\%), with no bias (less than 1\%) in any case, while the optical library is systematically biased (2.5-4.9\%), in particular for the larger/closer signals. This together with the very high memory consumption (of several extra GB) during simulations when using an optical library justifies the current choice of the semi-analytic model as the default for fast simulations in the DUNE FD.


\end{document}

\documentclass[../main-v1.tex]{subfiles}
\begin{document}
\section{Reconstruction Algorithms\hideme{ TRJ - draft}}
\label{sec:algo:reco}

This section gives brief outlines of reconstruction algorithms in use in DUNE's detectors.  The \dword{fd} will consist of four modules, at least three of which will be \dwords{lartpc}, and the fourth module, the module of opportunity, has yet to be designed.  The near detector will have a \dword{lartpc}, in addition to a muon spectrometer and/or \dword{ndgar}, and \dword{sand}, each of which consists of multiple subdetectors.  \dword{ndlar} will have a pixel-based readout, which enables a much easier reconstruction of densely-packed objects.  The \dword{pdsp} detector is also a \dword{lartpc}, and a vertical-drift prototype is planned.

\subsection{Liquid Argon TPC Reconstruction}
\label{sec:algo:reco:lartpc}

\subsubsection{Wire- and Strip-Based Liquid Argon TPC Reconstruction}
\label{sec:algo:reco:lartpc:wirestrip}

Reconstructing events in a \dword{lartpc} is challenging.  Each trigger record contains a large number of waveforms, one for each readout channel.  In \dword{pdsp}, the waveforms typically were 6000 time ticks long, with one \dword{adc} sample per time tick.  A detailed description of the reconstruction procedures, starting with these waveforms, is given in~\cite{Abi:2020mwi}, and summarized briefly here.

Typically, data from a single-phase \dword{lartpc} \dword{daq} are stored as compressed sequences of data frames, containing 128 or 256 channels' worth of data on each time sample.  These must be decompressed and rearranged into sequences of \dword{adc} values corresponding to the original waveforms.  Additional decoder modules are run to unpack \dword{pds}, \dword{crt} trigger and timing \dword{daq} fragments.
From there, mitigations are applied to correct, as best as possible, known failures of the front-end electronics.  As yet, these are only needed for the TPC readout electronics.  A particular failure mode in \dword{pdsp} is the sticky-code problem, which is mitigated by interpolating neighboring ADC samples in time.  Channel-to-channel gain corrections are applied, correlated noise is filtered out, the pedestal is found, and a correction is made for the \dword{ac} coupling between the preamp and the ADC.  The ADC mitigation, correlated noise removal, gain correction, pedestal finding and \dword{ac} coupling corrections are grouped into the data preparation module.  The \dword{wirecell} toolkit provides a two-dimensional deconvolution in wire index and charge arrival time, and it associates charge in the three views to produce a three-dimensional image of the charge locations where it was produced.  

The calibrated, filtered, deconvolved waveforms are then used as input to a hit-finding algorithm, which identifies peaks in the waveforms and fits Gaussians to each proposed peak.  These hits are then used as inputs to general reconstruction algorithms, such as the SpacePointSolver Pandora, TrajCluster, and PMA, which identify clusters, tracks and showers in three dimensions by associating objects in the three two-dimensional views.  The calorimetry modules sum up and calibrate the charge deposits for use in energy reconstruction and \dword{pid}.  The parameters of the clusters, tracks and showers are stored in ROOT trees that end users can analyze rapidly and repeatedly.

Additional modules find hits in the photon detector waveforms and group them into clusters called flashes.  The \dword{crt} data are analyzed in dedicated modules that produce associations between hits in the upstream and downstream \dword{crt} modules for use in analyses.

Separating track-like energy deposits from shower-like deposits is a key part of many DUNE analyses.  This is accomplished with a \dword{cvn} called {\tt EmTrackMichelID} in Table~\ref{tab:protodune_cpu_reco_by_module}.  It is one of the most CPU-intensive operations in the \dword{pdsp} reconstruction chain, when the algorithm is run on a grid node lacking a GPU.  Recently, however, a \dword{gpuaas} technique has been developed~\cite{Wang:2020fjr}, enabling a speedup of the order of a factor of ten, though it depends on the ratio of CPU-only nodes to GPU resources.


\subsubsection{Pixel-Based Liquid Argon TPC Recontruction}
\label{sec:algo:reco:lartpc:pixels}



\begin{longtable}
{l r}
\caption[Processing time for reconstruction modules for a \dword{pdsp} event]{Wall-clock module execution times for the reconstruction of a typical ProtoDUNE-SP event, in seconds.  The event is a data event from Run 5809, a 1 GeV beam run.} \\ \toprowrule
  \rowcolor{dunesky}
Module Label & time/event (sec)\\ \toprowrule
RootInput(read)                          &     0.147283          \\
timingrawdecoder:TimingRawDecoder        &     0.0095498         \\
ssprawdecoder:SSPRawDecoder              &     0.126099          \\
crtrawdecoder:CRTRawDecoder              &     0.0181903         \\
ctbrawdecoder:PDSPCTBRawDecoder          &     0.0204753         \\
beamevent:BeamEvent                      &      1.6154           \\
caldata:DataPrepByApaModule              &      83.0324          \\
wclsdatasp:WireCellToolkit               &      79.8616          \\
gaushit:GausHitFinder                    &      1.56342          \\
nhitsfilter:NumberOfHitsFilter           &    0.00177549         \\
reco3d:SpacePointSolver                  &      9.44157          \\
hitpdune:DisambigFromSpacePoints         &      1.52541          \\
pandora:StandardPandora                  &      39.8787          \\
pandoraWriter:StandardPandora            &     0.370009          \\
pandoraTrack:LArPandoraTrackCreation     &      4.57221          \\
pandoraShower:LArPandoraShowerCreation   &      3.73432          \\
pandoracalo:Calorimetry                  &      2.11152          \\
pandoracalonosce:Calorimetry             &      1.90852          \\
pandorapid:Chi2ParticleID                &     0.0115046         \\
pandoracali:CalibrationdEdXPDSP          &     0.106919          \\
pandoracalipid:Chi2ParticleID            &    0.00851985         \\
pandoraShowercalo:ShowerCalorimetry      &      3.26953          \\
pandoraShowercalonosce:ShowerCalorimetry &      3.17698          \\
emtrkmichelid:EmTrackMichelId            &      233.794          \\
ophitInternal:OpHitFinder                &     0.0190084         \\
ophitExternal:OpHitFinder                &    0.00828164         \\
opflashInternal:OpFlashFinder            &     0.0172739         \\
opflashExternal:OpFlashFinder            &    0.000597182        \\
opslicerInternal:OpSlicer                &     0.0184422         \\
opslicerExternal:OpSlicer                &    0.00538796         \\
crttag:SingleCRTMatchingProducer         &     0.0231673         \\
crtreco:TwoCRTMatchingProducer           &     0.0128138         \\
anodepiercerst0:T0RecoAnodePiercers      &      1.07673          \\
pandora2Track:LArPandoraTrackCreation    &      11.6525          \\
pandora2calo:Calorimetry                 &      4.95932          \\
pandora2calonosce:Calorimetry            &      4.55118          \\
pandora2pid:Chi2ParticleID               &     0.0211641         \\
pandora2cali:CalibrationdEdXPDSP         &     0.0867829         \\
pandora2calipid:Chi2ParticleID           &     0.0206842         \\
pandora2Shower:LArPandoraShowerCreation  &      4.17815          \\
pandora2Showercalo:ShowerCalorimetry     &      4.1729           \\
pandora2Showercalonosce:ShowerCalorimetry&      3.83562          \\
TriggerResults:TriggerResultInserter     &     0.000349372        \\
RootOutput                               &    2.8912e-05         \\
RootOutput(write)                        &     2.79458               \\
{\bf Total:}                             &     {\bf 507.881}      \\ \colhline
\label{tab:protodune_cpu_reco_by_module}
\end{longtable}



\subsubsection{Pixel-Based Gaseous Argon TPC Recontruction}
\label{sec:algo:reco:gartpc:pixels}

The \dword{ndgar} consists of two primary detectors -- a copy of the ALICE pixel-based \dword{tpc} in a 10-bar gas consisting predominantly of argon, surrounded by a calorimeter and a superconducting magneti coil.  A muon system is envisaged outside of the calorimeter but is not included in the simulation or the reconstruction at the time of writing.  Data are unpacked and hits are found on the per-readout-pad waveforms, similarly to how the initial stages of reconstruction for a \dword{lartpc} are followed.  Hits are then clustered into TPC clusters, which reduces the memory and CPU usage of subsequent steps, and also increases the spatial resolution of individual clusters.  Vector hits are found by grouping TPC clusters into short line segments, and the vector hits themselves are grouped into track candidates by a pattern recognition module.  A track fit based on a Kalman filter then finds the best estimates of the track parameters.  Vertices are then found using tracks with nearby endpoints.  Because there is a cathode in the middle of the drift volume in the nominal design, a cathode stitch module is then run to associate track segments on either side of the cathode, bring them together by solving for the best interaction time that lines the segments up, and also moves the associated tracks and vertices.  Vees ($K^0_s$ and
$\Lambda^0$ candidates) are found by pairing nearby tracks together, optionally requiring a displacement from another vertex, and forming the invariant mass.  The calorimeter consists of a mixture of strips and pads.  Calorimeter hits are found in the \dword{sipm} waveforms provided by the calomrimeter \dword{daq}, and they are clustered together to form reconstructed three-dimensional energy deposit objects with positions, directions, and energies.




\begin{dunetable}
[Average \dshort{ndgar} event wall-clock module execution times]
{l r}
{tab:garsoft_cpu_reco_by_module}
{Average wall-clock processing time for reconstruction modules for GArSoft events consisting of just one interaction in the gas.  An actual spill will contain of order 60 interactions, mostly  in the calorimeter, though many tracks will pass through the gas.}
Module Label & time/event (sec)\\ \toprowrule
RootInput(read)                             &    0.000353765       \\
init:EventInit                         &    1.31275e-05       \\
hit:CompressedHitFinder                &    0.00488308        \\
tpcclusterpass1:TPCHitCluster          &     0.0091922        \\
vechit:tpcvechitfinder2                &     0.0103787        \\
patrec:tpcpatrec2                      &     0.0130245        \\
trackpass1:tpctrackfit2                &     0.014211         \\
vertexpass1:vertexfinder1              &    0.000851081       \\
tpccluster:tpccathodestitch            &     0.0269436        \\
track:tpctrackfit2                     &     0.0135842        \\
vertex:vertexfinder1                   &    6.19847e-05       \\
veefinder1:veefinder1                  &    9.96417e-05       \\
sipmhit:SiPMHitFinder                  &    0.00153375        \\
sscalohit:CaloStripSplitter            &     0.0547754        \\
calocluster:CaloClustering             &    0.00492599        \\
trkecalassn:TPCECALAssociation         &    0.000237503       \\
TriggerResults:TriggerResultInserter        &    2.36397e-05       \\
RootOutput                                  &    3.6783e-06        \\
RootOutput(write)                           &    0.214077         \\
{\bf Total}                                  &     {\bf 0.369802}       \\ 
\end{dunetable}



\section {Anything missing? \hideme{??? -needed}}
\end{document}

\section{Analysis Algorithms}
\label{sec:algo:an}

 % Tom and Mathew and Elisabetta

\cleardoublepage


\input{common/final}
\end{document}
